\chapter{Considerações finais}
Após a análise de todas as equipes descritas anteriormente, é possível concluir que todas fizeram excelentes trabalhos tendo em vista suas circunstâncias individuais e o curto período de tempo para preparar suas apresentações e documentações para a exposição. Cada uma das equipes trouxera elementos ricos e diferentes de acordo com suas propostas e todas se fizeram entendíveis sobre seus objetivos para o ano.

Por outro lado, após tais leituras e análises, foi possível perceber também algumas melhorias para nosso projeto, sendo elas: ajustar a explicação sobre as tecnologias, trazendo a especificação de um \acs{sgbd}, \textsl{frameworks} para o \textsl{front-end} da aplicação e a explicação da nossa arquitetura de forma visual e simplificada, visto que o \acs{mvc} pode ser confundido com um padrão de código; ajustar o detalhamento visual do que cada integrante irá fazer dentro do projeto e criar um quadro de distribuição de tarefas; partir para especificações mais detalhadas sobre as funcionalidades utilizando as histórias de usuário; além de trabalhar para que as futuras apresentações não ultrapassem o tempo estipulado e não fujam mais da padronização de páginas feita pelos professores.

Ademais, é importante que as equipes estejam cientes em suas documentações sobre os seguintes pontos: utilização de palavras em inglês sem estarem em \textsl{itálico}; falta de citações e referências suficientes para corroborar suas afirmações durante os textos; formatação de quadros estar diferente em cada coluna e capítulos ou seções com apenas um parágrafo. 

Um outro ponto em comum é que nenhuma das equipes, incluindo a nossa, conseguiu validar seus arquivos "equipe.yaml" no \href{https://yamllint.readthedocs.io/en/stable/}{Yamllint}, visto que nenhum dos projetos se encontra na \href{https://dicas.ivanfm.com/aulas/blogs-de-trabalhos.html}{página de Blogs de Trabalhos no Dicas Ivan}. Desse modo, deixamos como sugestão geral para que revisemos todos nossas estruturas de pastas e arquivos gerados dentro do repositório do SVN da disciplina.

Por fim, ainda que possam ter ocorrido imprevistos e problemas até o momento da apresentação, acreditamos que todos possuem propostas com grandes potenciais para desenvolvimento e esperamos que consigam obter o devido êxito esperado para suas aplicações, assim como aguardamos avidamente o crescimento e a melhora de todos com os \textsl{feedbacks} dados.