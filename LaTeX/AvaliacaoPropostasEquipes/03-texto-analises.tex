%%%%%%%%%%%%%%%%%%%%%%%%%%%%%%%%%%%%%%%%%%%%%%%%%%%%%%%%%%%%%
\chapter{Introdução}
Este documento tem como objetivo analisar criticamente as apresentações (dos dias 5, 12 e 26 de setembro de 2022) das entregas finais das equipes constitui itens dos projetos moldados na disciplina de \acs{pds}, bem como suas documentações entregues no dia 22 de agosto de 2022.

Antes de dar início as análises, é importante ressaltar que todas as aqui presentes foram feitas com o objetivo de gerar um \textsl{feedback}, ou seja, dar um retorno para as equipes, baseadas na opinião e no consenso do que foi discutido entre os integrantes da equipe Bunka Bytes. Além disso, o presente documento não pretende fazer observações de juízo de valor ou sobre o desempenho oratório daqueles que fizeram as apresentações de seus projetos. Os fatores aqui avaliados são com base nas experiências pessoais e profissionais de cada integrante e nos requisitos apresentados para a disciplina, devidamente descritos na página de \href{https://dicas.ivanfm.com/aulas/pds.html}{Dicas e regras do professor Ivan Francolin} e especificados pelos professores.
%%%%%%%%%%%%%%%%%%%%%%%%%%%%%%%%%%%%%%%%%%%%%%%%%%%%%%%%%%%%%
\chapter{Equipes}
Este capítulo procura estabelecer uma ordem cronológica de análises das propostas descritas por cada uma das equipes dispostas na turma 413, formada de acordo com a ordem das apresentações feitas no dia 11 de abril de 2022. Elenca-se aqui, portanto, os pontos principais (positivos e sugestões de melhoria) observados pela equipe Bunka Bytes sobre as seguintes partes das propostas: introdução (exposição do problema e justificativas), especificações sobre a proposta, organização da equipe e tecnologias e ferramentas escolhidas.

%%%%%%%%%%%%%%%%%%%%%%%%%%%%%%%%%%%%%%%%%%%%%%%%%%%%%%%%%%%%
\section{Equipe Sigma}
A proposta inicial foi muito bem explicada pela equipe Sigma, com o sistema “Visita Sampa”, ao qual deixou claro o intuito da aplicação e o nicho em que se encontra, sendo um site de roteiros turísticos personalizados da capital de São Paulo. Outro ponto positivo foi que explicaram bem o funcionamento do sistema e suas principais funcionalidades, como o questionário e o roteiro. Para validar essa ideia do projeto, a equipe também realizou uma pesquisa de viabilidade, ao qual mostrou muito bem em seu documento, por meio de gráficos de pizza que corroboraram com as proposições feitas na introdução do projeto. Além disso, foi muito interessante o modo com que preferiram colocar as perguntas para a pesquisa de campo nos apêndices: ficou bem organizado e fácil de entender.

No geral, a equipe fez uma boa apresentação, com atenção especial para a padronização dos integrantes ao combinarem suas roupas, assim como seus slides não sobrecarregados de informações, possuindo cores agradáveis e seguindo o modelo proposto. Também introduziram sobre a equipe e os papéis de cada integrante, sobre a importância da cultura e como ela se relaciona com o problema proposto, a pesquisa de viabilidade, o \textsl{wireframe} e o funcionamento da aplicação; para assim entendermos melhor a proposta do projeto. Entretanto, um ponto que sentimos falta, foi sobre a explicação de uma metodologia de gerenciamento/organização a ser utilizada pela equipe, ao qual não foi falada sobre.

Quanto ao desenvolvimento, foi muito bem explicado na apresentação sobre a estrutura da arquitetura do aplicativo, porém, uma sugestão para os leitores, é que isto poderia ser descrito da mesma forma no documento, ao invés de apenas a imagem da arquitetura. Por outro lado, as tecnologias a serem utilizadas pela equipe foram muito bem explicadas, tanto no documento como na apresentação, e ficou claro o motivo de usar cada uma, ainda que tenhamos sentido falta de um \textsl{framework} no \textsl{back-end}, visto que isso talvez reduza um trabalho a mais desnecessário. Portanto, 
é algo que pode ser pensado pela equipe para um desenvolvimento mais rápido da aplicação.

Algo que nos confundiu no documento foi sobre a necessidade de incluir o resumo e \textsl{abstract}, que já foi realizado na proposta inicial do projeto, visto que a recomendação padrão é que o mesmo seja feito apenas na entrega final do documento. Outro ponto, foi sobre as ferramentas de apoio a serem utilizadas, apesar de serem citadas algumas ferramentas para a organização, no documento não foi mostrado quais seriam todas as ferramentas e o porquê de utilizar cada uma.

%%%%%%%%%%%%%%%%%%%%%%%%%%%%%%%%%%%%%%%%%%%%%%%%%%%%%%%%%%%%%
\section{Equipe TechFive}
Tendo em vista a primeira parte da apresentação e da documentação da proposta do projeto Mini-Me, foi possível entender com clareza o propósito da aplicação e qual o nicho de mercado procuram explorar, visto que apresentaram logo nos primeiros slides uma análise dos aplicativos concorrentes e a partir deles buscaram aplicar pontos diferenciais em sua proposta - como a questão do uso das aplicações por mais de um responsável ao mesmo tempo, por exemplo. Além disso, antes de tratarem sobre os concorrentes também deram visibilidade a uma dor do seu usuário final, o que foi um ponto positivo não só para o entendimento da solução, mas também para as demais partes da proposta.

Porém, sobre essa parte na sua documentação, um ponto de melhoria para o leitor seria adicionar a análise das concorrências como parte da justificativa para a construção do Mini-Me, já que durante a apresentação, a clareza do objetivo foi melhor especificada quando feita esta comparação logo de início. Além disso, seria importante para justificar com mais detalhes o problema a ser resolvido, pois o problema não recebeu muito foco na introdução e só é melhor compreendido quando apresentadas as concorrências.

A apresentação da equipe TechFive foi clara e objetiva e possibilitou um bom entendimento sobre quais serão as áreas de atuação de cada integrante. Uma única sugestão seria talvez demonstrar possíveis metodologias de gerenciamento e desenvolvimento que poderão ser utilizadas para nortear o trabalho em equipe, e possivelmente fazer uma ligação sobre isso na utilização das ferramentas de apoio.

Com relação as funcionalidades iniciais do aplicativo, foi bastante interessante elas serem justificadas com base nos erros dos concorrentes, conforme foi explicitado acima, porém para o futuro da aplicação, seria interessante procurar referências das quais demonstrem que o objetivo dito no início esteja sendo cumprido em cada uma das funcionalidades.

De repente, documentar uma pesquisa sobre a usabilidade de seu público alvo, pode ajudar a entender melhor quais são os principais pontos a serem melhorados nos quesitos de facilidade e intuição. A exposição dos gráficos, por exemplo, poderia exigir um tempo de uso da aplicação e uma quantidade de dados relativamente grandes para que comecem a trazer informações relevantes sobre o bebê. Por outro lado, a funcionalidade das notificações pode trazer uma vantagem para chamar a atenção dos pais para a aplicação, ainda que precisem parar por um tempo para preencher as informações sobre seus bebês (o que pode ser um desafio para o usuário, já que crianças recém nascidas costumam demandar muita atenção). 

Em resumo, sobre essa parte, a importância da busca de referências pode ser um artifício interessante futuramente em sua revisão de literatura, pois nela a equipe pode encontrar pesquisadores sobre o assunto que possam ser auxiliares na explicação da importância de cada uma das funcionalidades. Isto, por outro lado, pode direcionar a pesquisas de usabilidade mais certeiras e que estejam diretamente relacionadas a validação dos usuários sobre como as funcionalidades foram aplicadas, além de ajudarem a observar dores futuras.

Sobre as tecnologias escolhidas, é possível compreender bem o porquê de serem escolhidas e suas descrições durante a apresentação foram explicitadas de forma objetiva. Além disso, a escolha da arquitetura Offline First ter se baseado numa dor dos usuários foi bastante rica para o entendimento. No entanto, vale revisar a explicação feita, pois há alguns pontos que, tanto na documentação como na apresentação, podem ter ficado confusos: como a utilização do Node.js, do Redux e do RealmBB estarem na parte de arquitetura e não como tecnologias. Ademais, dentro da arquitetura poderia ser inclusa uma explicação sobre quais protocolos serão utilizados e qual API estará em uso, por exemplo. De modo geral, uma sugestão sobre esta parte seria pesquisar mais afundo sobre cada uma das tecnologias, visto que algumas definições podem ter causado confusão nos leitores.

Por último, sobre a utilização do iOS, por outro lado, é preciso explicar melhor sobre como será feita, já que a utilização de tal implementação, segundo a \href{https://reactnative.dev/docs/environment-setup}{documentação do React Native}, só é possível de ser implementada em computadores MacOS com o emulador Xcode, ao contrário do Android Studio, que funciona tanto nesse sistema operacional como no Linux e no Windows. Tal explicação pode ser importante para definirem o escopo de sua aplicação, pois algumas funções nativas do React Native são programadas de maneiras diferentes de acordo com o sistema operacional a ser utilizado.

%%%%%%%%%%%%%%%%%%%%%%%%%%%%%%%%%%%%%%%%%%%%%%%%%%%%%%%%%%%%%
\section{Equipe LibWeb}
Tendo por base a proposta apresentada pela equipe LibWeb, o projeto se baseia em suma de uma biblioteca de libras que abrange todos os símbolos da linguagem de sinais brasileiras, inclusive com as variantes características de cada região do Brasil.

A equipe foi apresentada de maneira clara e mesmo sem a participação de um de seus componentes na apresentação, foi possível entender as metodologias utilizadas pela equipe, sua forma de organização e a área específica em que cada componente irá atuar de forma mais ativa.

De semelhante modo, a proposta também foi apresentada de maneira clara e objetiva durante o momento de aula, porém a documentação não se encontra disponível no repositório SVN da disciplina, o que impossibilitou uma melhor avaliação. Por isso, também aproveitamos para deixar uma sugestão sobre as atentar-se para a disponibilização dos arquivos no repositório, pois isto pode, não só fazer com que percam pontos na disciplina, como dificultaria o trabalho dos colegas e professores. Uma outra sugestão é que conversem com os professores quando estas dificuldades surgirem, explicando a situação que não tenha permitido entregarem, já que nossa equipe sofreu com um problema de envio do documento no Moodle da disciplina, e após conversarmos com os professores, pudemos ajustar o envio conforme previsto no início.

De um modo geral, com relação aos materiais e métodos, a equipe apresentou e justificou bem as escolhas das tecnologias que pretendem utilizar, como o uso do AWS como servidor e banco de dados e também da arquitetura escolhida para o projeto.

%%%%%%%%%%%%%%%%%%%%%%%%%%%%%%%%%%%%%%%%%%%%%%%%%%%%%%%%%%%%%
\section{Equipe SpaceCode}
De acordo com a apresentação da proposta inicial e do documento de visão realizada pela equipe SpaceCode, referente ao projeto "Troca Book", foi possível entender bem o problema na qual a equipe está disposta a resolver. Ademais, a SpaceCode preparou uma pesquisa de viabilidade e, com base nos dados obtidos, formaram muito bem o objetivo do projeto.

Com relação à apresentação da equipe, de modo geral, foi possível observar sua distribuição de tarefas dos integrantes. Entretanto, pode não ter ficado interessante a parte descrita como "auxílio" no quadro “Tabela Equipe”, visto que todos os outros campos estão marcados com “X” e este não. Além disso, dois dos integrantes não possuem esta tarefa, o que deixa ainda mais vago sobre quais atividades serão desenvolvidas nessa divisão. Portanto, apesar de ter sido explicado na apresentação feita em aula, uma sugestão de melhoria seria explicar, de maneira mais detalhada, essa distribuição também no documento, principalmente no que tange a parte de “auxílio” dos integrantes.

A gestão da equipe através das metodologias ágeis escolhidas está bem definida no projeto, a explicação de terem escolhido o Scrum e XP para trabalharem em conjunto deu credibilidade no funcionamento da equipe como um todo - ainda mais com detalhamento das práticas das metodologias dentro do documento.

A proposta do projeto ofereceu um entendimento básico, mas poderia ficar melhor se houvesse um detalhamento de pelo menos duas funcionalidades principais dentro do sistema, pois seria melhor entendido a forma como o Troca Book irá funcionar para incentivar as pessoas a lerem mais.

Por último, a parte de materiais e ferramentas ficou bem detalhada e definida, principalmente com a arquitetura mostrada, tanto no documento quanto na apresentação. As linguagens, \textsl{frameworks} e ferramentas, também foram bem explicadas de acordo com a razão de serem escolhidas para trabalharem e de como serão usados no projeto. Entretanto, não foi possível entender bem o funcionamento do Blade, comentado na apresentação, visto que sua explicação gerou certa confusão entre os entendimentos dos integrantes da nossa equipe. Uma sugestão para esse problema poderia ser explicar com mais detalhes sobre a utilização desse \textsl{framework} no documento do projeto.

%%%%%%%%%%%%%%%%%%%%%%%%%%%%%%%%%%%%%%%%%%%%%%%%%%%%%%%%%%%%%
\section{Equipe Fast Solutions}
A Fast Solutions iniciou sua explicação com a apresentação dos membros da equipe. Logo, já em seus próximos slides eles se preocuparam em dar início a proposta de seu projeto, visto que uma das primeiras informações que passaram para os professores e para as outras equipes, foi o motivo do surgimento da proposta.
Antes de apresentar a ideia em si, esta equipe trouxe alguns problemas enfrentados nesse âmbito, já que um dos integrantes compartilhou uma dificuldade vivenciada a respeito do assunto.

Isso foi um ponto interessante, pois, para uma apresentação, ainda mais se tratando da exposição de uma ideia de projeto, é essencial se ter um estudo da problemática, pois, mostra que a equipe se preocupou em estudar o espaço onde gostariam de trabalhar. Porém, da forma que essa problemática foi posta, ou seja, as explicação das dificuldades vivenciadas pelo integrante; deu a entender que a ideia e a base para todo o projeto surgiu apenas desse integrante em específico. Logo, uma sugestão para tal, seria evitar falar sempre na primeira pessoa do singular, já que se trata de uma equipe e a opinião dos demais também deve ser levada em conta para a composição do projeto como um todo. Exceto por essa pequena sugestão, a problemática foi muito bem explicada.

À vista da problemática, a ideia do projeto ficou bem clara e o nome também se relacionou de maneira agradável "TinderVoluntariado", pois, retrata bem sobre o que será essa aplicação: um site que ajude pessoas voluntárias a encontrar um serviço social no qual queiram contribuir. Além disso, na apresentação também abordaram a divisão das tarefas entre os membros, assim foi possível perceber e entender o fluxo de desenvolvimento do projeto.

Seguindo adiante, a equipe também apresentou as tecnologias e ferramentas com os quais gostariam de trabalhar, dando uma introdução a respeito de como elas seriam utilizadas, o que foi ideal para o momento. Porém, a equipe não pôde apresentar a arquitetura com a qual gostariam de trabalhar, logo, seria recomendável atualizar essas informações faltantes, por exemplo, num possível vídeo de apresentação ou até mesmo no próprio documento.

Dessa forma, conclui-se que a apresentação seguiu pelo percurso proposto pelos professores e retratou bem o que querem atingir com o projeto e como pretendem desenvolvê-lo. Porém, realizando a avaliação da apresentação percebeu-se que a equipe ainda não atualizou o seu repositório no SVN, visto que existem documentos pendentes - como o da proposta inicial. Logo, não foi possível fazer uma avaliação mais detalhada e melhor desenvolvida por nossa equipe - estabelecendo comparações entre a documentação e a apresentação, como vinha sendo feita até então, já que a mesma não fora disponibilizada. 

Por último, outra sugestão bastante importante é o cuidado com a estrutura do SVN, já que as pastas estão fora do padrão proposto pelos professores, e isso dificulta a visualização dos documentos, além de deixar o projeto com uma aparência poluída e confusa.

%%%%%%%%%%%%%%%%%%%%%%%%%%%%%%%%%%%%%%%%%%%%%%%%%%%%%%%%%%%%%
\chapter{Considerações finais}
Após a análise de todas as equipes descritas anteriormente, é possível concluir que todas fizeram excelentes trabalhos tendo em vista suas circunstâncias individuais e o curto período de tempo para preparar suas apresentações e documentações para a exposição. Cada uma das equipes trouxera elementos ricos e diferentes de acordo com suas propostas e todas se fizeram entendíveis sobre seus objetivos para o ano.

Por outro lado, após tais leituras e análises, foi possível perceber também algumas melhorias para nosso projeto, sendo elas: ajustar a explicação sobre as tecnologias, trazendo a especificação de um \acs{sgbd}, \textsl{frameworks} para o \textsl{front-end} da aplicação e a explicação da nossa arquitetura de forma visual e simplificada, visto que o \acs{mvc} pode ser confundido com um padrão de código; ajustar o detalhamento visual do que cada integrante irá fazer dentro do projeto e criar um quadro de distribuição de tarefas; partir para especificações mais detalhadas sobre as funcionalidades utilizando as histórias de usuário; além de trabalhar para que as futuras apresentações não ultrapassem o tempo estipulado e não fujam mais da padronização de páginas feita pelos professores.

Ademais, é importante que as equipes estejam cientes em suas documentações sobre os seguintes pontos: utilização de palavras em inglês sem estarem em \textsl{itálico}; falta de citações e referências suficientes para corroborar suas afirmações durante os textos; formatação de quadros estar diferente em cada coluna e capítulos ou seções com apenas um parágrafo. 

Um outro ponto em comum é que nenhuma das equipes, incluindo a nossa, conseguiu validar seus arquivos "equipe.yaml" no \href{https://yamllint.readthedocs.io/en/stable/}{Yamllint}, visto que nenhum dos projetos se encontra na \href{https://dicas.ivanfm.com/aulas/blogs-de-trabalhos.html}{página de Blogs de Trabalhos no Dicas Ivan}. Desse modo, deixamos como sugestão geral para que revisemos todos nossas estruturas de pastas e arquivos gerados dentro do repositório do SVN da disciplina.

Por fim, ainda que possam ter ocorrido imprevistos e problemas até o momento da apresentação, acreditamos que todos possuem propostas com grandes potenciais para desenvolvimento e esperamos que consigam obter o devido êxito esperado para suas aplicações, assim como aguardamos avidamente o crescimento e a melhora de todos com os \textsl{feedbacks} dados.
%%%%%%%%%%%%%%%%%%%%%%%%%%%%%%%%%%%%%%%%%%%%%%%%%%%%%%%%%%%%%