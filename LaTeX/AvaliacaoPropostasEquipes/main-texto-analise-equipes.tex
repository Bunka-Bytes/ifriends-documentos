%% Adaptado a partir de :
%%    abtex2-modelo-trabalho-academico.tex, v-1.9.2 laurocesar
%% para ser um modelo para os trabalhos no IFSP-SPO

\documentclass[
    % -- opções da classe memoir --
    12pt,               % tamanho da fonte
    openright,          % capítulos começam em pág ímpar (insere página vazia caso preciso)
    %twoside,            % para impressão em verso e anverso. Oposto a oneside
    oneside,
    a4paper,            % tamanho do papel. 
    % -- opções da classe abntex2 --
    %chapter=TITLE,     % títulos de capítulos convertidos em letras maiúsculas
    %section=TITLE,     % títulos de seções convertidos em letras maiúsculas
    %subsection=TITLE,  % títulos de subseções convertidos em letras maiúsculas
    %subsubsection=TITLE,% títulos de subsubseções convertidos em letras maiúsculas
    % Opções que não devem ser utilizadas na versão final do documento
    %draft,              % para compilar mais rápido, remover na versão final
    MODELO,             % indica que é um documento modelo então precisa dos geradores de texto
    TODO,               % indica que deve apresentar lista de pendencias 
    % -- opções do pacote babel --
    english,            % idioma adicional para hifenização
    brazil              % o último idioma é o principal do documento
    ]{ifsp-spo-inf-cemi} % ajustar de acordo com o modelo desejado para o curso
        
% ---

% --- 
% CONFIGURAÇÕES DE PACOTES
% --- 
%\usepackage{etoolbox}
%\patchcmd{\thebibliography}{\chapter*}{\section*}{}{}


% ---
% Informações de dados para CAPA e FOLHA DE ROSTO
% ---
\titulo{ANÁLISE DAS PROPOSTAS DAS EQUIPES}

\renewcommand{\imprimirautor}{
\begin{tabular}{lr}
ANAÍ VILLCA ROJAS & SP3029085 \\
JAMILLI VITÓRIA GIOIELLI & SP3027473 \\
JOSÉ ROBERTO CLAUDINO FERREIRA & SP3024369 \\
JULIA ROMUALDO PEREIRA & SP3023061 \\
KAIKY MATSUMOTO SILVA & SP185075X \\
\end{tabular}
}


\tipotrabalho{Projeto da Disciplina de PDS}

\disciplina{PDS - Prática para Desenvolvimento de Sistemas}

\preambulo{Análises das propostas de projetos apresentadas pelas demais equipes na disciplina de PDS. }

\data{2022}

%---
\renewcommand{\orientadorname }{Professor:}
\coorientador{Carlos Henrique Veríssimo Pereira}
\renewcommand{\coorientadorname}{Professor:}
\orientador{Johnata Souza Santicioli}
%---

% informações do PDF
\makeatletter
\hypersetup{
        %pagebackref=true,
        pdftitle={\@title}, 
        pdfauthor={\@author},
        pdfsubject={\imprimirpreambulo},
        pdfcreator={LaTeX with abnTeX2},
        pdfkeywords={abnt}{latex}{abntex}{abntex2}{trabalho acadêmico}, 
        colorlinks=true,            % false: boxed links; true: colored links
        linkcolor=blue,             % color of internal links
        citecolor=blue,             % color of links to bibliography
        filecolor=magenta,              % color of file links
        urlcolor=blue,
        bookmarksdepth=4
}
\makeatother

% ----
% Início do documento
% ----
\begin{document}

% Retira espaço extra obsoleto entre as frases.
\frenchspacing 

\pretextual

% ---
% Capa - Para proposta a folha de rosto é suficiente pois é mais completa.
% ---
\imprimirfolhaderosto
% ---

% ---
% inserir o sumario
% ---
\pdfbookmark[0]{\contentsname}{toc}
\tableofcontents*
\cleardoublepage
% ---

% ----------------------------------------------------------
% ELEMENTOS TEXTUAIS
% ----------------------------------------------------------
\textual

% ---
% inserir lista de abreviaturas e siglas
% ATENCAO o SHARELATEX/OVERLEAF GERA O GLOSSARIO SOMENTE UMA VEZ
% CASO SEJA FEITA ALGUMA ALTERAÇÃO NA LISTA DE SIGLAS É NECESSARIO UTILIZAR A OPÇÃO :
% "Clear Cached Files" DISPONIVEL NA VISUALIZAÇÃO DOS LOGS 
% ---
% https://www.sharelatex.com/learn/Glossaries


\ifdef{\printnoidxglossary}{
    \printnoidxglossary[type=\acronymtype,title=Lista de abreviaturas e siglas,style=siglas]
    \cleardoublepage
}{
}



%%%%%%%%%%%%%%%%%%%%%%%%%%%%%%%%%%%%%%%%%%%%%%%%%%%%%%%%%%%%%
\chapter{Introdução}
Este documento tem como objetivo analisar criticamente as apresentações (dos dias 5, 12 e 26 de setembro de 2022) das entregas finais das equipes constitui itens dos projetos moldados na disciplina de \acs{pds}, bem como suas documentações entregues no dia 22 de agosto de 2022.

Antes de dar início as análises, é importante ressaltar que todas as aqui presentes foram feitas com o objetivo de gerar um \textsl{feedback}, ou seja, dar um retorno para as equipes, baseadas na opinião e no consenso do que foi discutido entre os integrantes da equipe Bunka Bytes. Além disso, o presente documento não pretende fazer observações de juízo de valor ou sobre o desempenho oratório daqueles que fizeram as apresentações de seus projetos. Os fatores aqui avaliados são com base nas experiências pessoais e profissionais de cada integrante e nos requisitos apresentados para a disciplina, devidamente descritos na página de \href{https://dicas.ivanfm.com/aulas/pds.html}{Dicas e regras do professor Ivan Francolin} e especificados pelos professores.
%%%%%%%%%%%%%%%%%%%%%%%%%%%%%%%%%%%%%%%%%%%%%%%%%%%%%%%%%%%%%
\chapter{Equipes}
Este capítulo procura estabelecer uma ordem cronológica de análises das propostas descritas por cada uma das equipes dispostas na turma 413, formada de acordo com a ordem das apresentações feitas nos dias 5, 12 e 26 de setembro de 2022. Elenca-se aqui, portanto, os pontos principais (positivos e sugestões de melhoria) observados pela equipe Bunka Bytes sobre as partes presentes na documentação e na apresentação dos projetos de um modo geral.

%%%%%%%%%%%%%%%%%%%%%%%%%%%%%%%%%%%%%%%%%%%%%%%%%%%%%%%%%%%%
\section{Equipe Sigma}
Por meio do projeto Visita Sampa, a equipe Sigma fornece aos seus usuários um roteiro de lugares para serem visitados em São Paulo, segundo preferências ou gostos pessoais, identificados na realização de um questionário que se baseia na classificação de personalidade conforme as alternativas escolhidas.

A apresentação, que ocorreu no dia 5 de setembro de 2022, teve, em suma, um resultado positivo, pois, a equipe conseguiu apresentar bem o projeto, seu valor, assim como as funcionalidades que o compõem. Os \textit{slides} usados na apresentação foram outro ponto positivo, visto que seguiram o modelo proposto pela disciplina e não ficaram poluídos. O visual da equipe também foi agradável, já que as roupas estavam combinando e isso, como mencionado pelos professores, demostra respeito à banca e aos colegas.

Embora a apresentação tenha seguido por um bom caminho, tirando alguns erros técnicos, como a falta do vídeo de demonstração, a equipe também empenhou um tempo considerável em algumas partes que não eram essências. Dessa forma, apresentamos algumas considerações: 

Na introdução, por exemplo, seria melhor já mostrar o protótipo no lugar do \textit{wireframe}, já que a explicação aplicada para ambos foi similar, também empenharam um tempo considerável na explicação das linguagens e ferramentas, portanto, seria interessante apenas apresentar como foi utilizado e quais adaptações foram feitas.

Acreditamos que dessa forma, a equipe teria se mantido no tempo estipulado, vale lembrar que com isso não queremos impor julgamento, apenas são sugestões que podem ser consideradas pelos membros em futuras apresentações.

A respeito da documentação do projeto, percebe-se que foi bem trabalhada e apresenta os requisitos exigidos pela disciplina. Dessa forma, fazendo uma varredura conseguimos identificar algumas sugestões que talvez possam ser do interesse da equipe:

Na seção ``Problema a ser solucionado'' sugerimos fazer a quebra de alguns parágrafos, já que ficaram um pouco extensos e isso, querendo ou não, acaba deixando a leitura um pouco cansativa. Outra ressalva é em relação às pesquisas descritas na seção, na apresentação a equipe mostrou esses resultados por gráficos, logo, seria interessante aplicá-lo também dessa maneira no documento, pois proporciona visualizar, de melhor forma, o que está sendo abordado.

Foi interessante por parte da equipe relatar os problemas que tiveram no decorrer do desenvolvimento do projeto, mas nem todos eles foram descritos, dessa forma seria válido especificá-los, mesmo que sejam de menor complexidade.

Já na seção ``Links do Projeto'' sugerimos reconsiderar a sua posição, visto que ela contém informações que seria interessante fornecer aos leitores já no início, ainda na seção seria legal também aplicar a padronização das figuras que vinha sendo visível ao longo do documento, logo, seria bom centralizar os \textit{QR codes}. Por último sugerimos repensar em relação à criação dos subcapítulos, percebe-se que essa ação foi tomada com fins de organização, pois, a equipe, provavelmente, fez isso para deixar os itens separados, mas talvez a criação desses subcapítulos tenha sido desnecessária, já que o uso de uma lista, por exemplo, também levaria a um mesmo resultado.

Vale reforçar, novamente, que com tais sugestões não queremos impor nenhum tipo de repreensão, nós apenas gostaríamos de ajudar na melhoria contínua do projeto.

%%%%%%%%%%%%%%%%%%%%%%%%%%%%%%%%%%%%%%%%%%%%%%%%%%%%%%%%%%%%%
\section{Equipe SpaceCode}
O projeto TrocaBook desenvolvido pela equipe SpaceCode, traz um tema importante para o âmbito social, tendo como principal objetivo estimular a leitura, o compartilhamento e indicações de livros. Por ser um tema importante e estimado socialmente, sentimos falta de uma explicação mais aprofundada da problemática, proposta e da solução trazida pelo sistema na documentação e nas apresentações. 

Após a análise da documentação, percebemos os seguintes pontos: sessões curtas, com conteúdos muito vagos que poderiam ter aprofundamento maior, visto que a equipe não recorreu a citações; imagens da pesquisa e da modelagem pequenas e com baixa resolução, dificultando a leitura; sentimos falta também da documentação dos planos de teste; além da documentação anterior não ter sido anexada nos apêndices. Dessa forma, percebe-se que a maioria dos requisitos impostos pela disciplina não foram cumpridos, ou se este não for o caso, não foram documentados. 

Com relação à apresentação, que ocorreu no dia 5 de setembro de 2022, achamos as letras das páginas pequenas e difíceis de ler, o contraste com imagem de fundo acabou não ficando muito bom, a demonstração do sistema foi um pouco lenta por conta da conexão de internet no IFSP, ponto em que o vídeo da demonstração seria muito útil. O vídeo do Gource foi mostrado durante a apresentação, entretanto o projeto ainda não havia sido alocado no repositório da disciplina \acs{svn}, todos os \textit{commits} foram realizados juntos e por apenas alguns integrantes da equipe, outro ponto que também afetou a apresentação, pois de uma equipe com cinco integrantes apenas três estavam presentes na apresentação. 

Por fim, o sistema não estava parecido com o protótipo e também não apresentava avanços significativos desde a apresentação da \acs{poc}. Os campos de formulário não estavam validados, havia pouco ou nenhuma implementação do PHP Orientado a Objetos, sendo este o principal requisito da disciplina, e acreditamos também que não existe uma funcionalidade que seja diferencial do sistema, algo que deixe uma identidade do projeto. 

%%%%%%%%%%%%%%%%%%%%%%%%%%%%%%%%%%%%%%%%%%%%%%%%%%%%%%%%%%%%%
\section{Equipe LibWeb}
A equipe LibWeb procura em seu projeto autointitulado criar um chamado dicionário léxico virtual, no qual os usuários podem adicionar e pesquisar palavras curadas pela comunidade surda e complementadas com vídeos produzidos por aqueles que disponibilizarem a inserir uma palavra no dicionário. 

A apresentação, feita no dia 12 de setembro de 2022, no geral, foi bastante objetiva e bem explicada, tocou nos pontos principais de forma clara e ainda se percebeu uma preocupação dos integrantes com o uso de roupas temáticas sobre o projeto. Foi notável que houve mudanças no projeto desde a proposta inicial, justificadas pela equipe tendo em vista as dificuldades enfrentadas ao longo do processo. Levando isso em consideração, será feita abaixo uma descrição resumida sobre alguns pontos e dúvidas que achamos mais importante ponderar, não entrando em muitos detalhes a respeito de cada parte do projeto.

Sobre a visão de negócio apresentada pela equipe, foi dito (e descrito no documento) que existe um ineditismo com relação ao sistema no mercado, porém seria interessante que se revisasse essa questão, pois, apesar de termos pouco conhecimento sobre o mercado para a comunidade surda, ao pesquisarmos sucintamente a respeito, foi possível encontrar alguns \textsl{plugins} e aplicativos de tradução de texto e voz da Web para Libras (pagos e gratuitos) que se assemelham a funcionalidades descritas no LibWeb, tais como o \href{https://www.handtalk.me/br/aplicativo/)}{Hand Talk}, o \href{https://lacom.ag/plicativo-uni-libras-e-a-comunicacao-com-o-deficiente-auditivo/)}{Uni Libras}, o \href{https://www.gov.br/governodigital/pt-br/vlibras/}{Vlibras}, entre outros - alguns encontrados \href{https://www.techtudo.com.br/noticias/2016/01/tradutor-de-libras-5-programas-e-sites-que-podem-ajudar-conversar.ghtml}{nessa matéria do TechTudo}. Esse último, em especial, chamou atenção por ser disponibilizado pelo Governo Federal gratuitamente e com código aberto. Desse modo, percebemos que talvez a inclusão de uma seção sobre análise de concorrência na documentação do sistema possa contribuir também para verificar um pouco mais sobre a posição do mercado nesse sentido.

Ainda sobre isso, também sentimos um pouco de falta de uma visão geral sobre quais são as preferências da comunidade surda e sobre quais já são as iniciativas de integração existentes dentro dela. Por exemplo, ao trazerem dados sobre a quantidade de falantes de libras na comunidade surda, qual o grau de surdez mais comum (leve, moderna, severa ou profunda), políticas de acessibilidade já existentes e quais são seus pontos de melhoria, o que pode explicar o pouco conhecimento da população brasileira sobre Libras, como as empresas de tecnologia estão se posicionando sobre a pauta da acessibilidade para surdos nesse momento, entre outras pautas. Isso pode ajudar quem não conhece o projeto e sua problemática a entender os diversos pontos de vista existentes no mesmo assunto e ainda contribuir para o embasamento na criação de funcionalidades, reforçando em conjunto a importância prática de sua realização.

Sobre a parte técnica, uma sugestão importante seria revisar a segurança a API desenvolvida, pois percebemos que seus métodos estão expostos e que não possuem autenticação para execução deles, permitindo que qualquer pessoa possa fazer requisições para a API sem controle sobre ações que podem causar perdas críticas sobre os dados armazenados.

De um modo geral, a nova versão do sistema ficou bastante interessante e contribuiu para uma melhora significativa no entendimento e na usabilidade da aplicação. Alguns dos requisitos da disciplina não foram encontrados na documentação (como testes unitários, análise estática, tabela de métricas e a análise de segurança), mas compreendemos as justificativas dadas pela equipe e esperamos que, caso queiram dar continuidade e tendo em vista sua evidente importância prática, possam passar pelo processo de melhoria contínua e aprimorar o projeto futuramente com base nos \textsl{feedbacks} e no julgamento do que acharem necessário.

%%%%%%%%%%%%%%%%%%%%%%%%%%%%%%%%%%%%%%%%%%%%%%%%%%%%%%%%%%%%%
\section{Equipe Fast Solutions}
Após uma análise do documento feito pela equipe Fast Solutions sobre seu projeto, é possível perceber uma ideia bastante interessante no ponto de vista de negócio, visto que trazem e utilizam muitos conteúdos como pesquisas e artigos em relação às ONGs com os voluntários. 

A solução na qual a equipe está propondo para este problema é, de certa forma, uma causa nobre que possui bastante potencial de crescimento, visto que não existem muitas concorrências com esse fundamento.

Em relação à parte de materiais e métodos do documento, sentimos que está faltando informações de como a equipe trabalhou com as ferramentas citadas. Por exemplo, a metodologia Scrum, seria interessante explicar quais aspectos foram utilizados  ou se houve adaptações no processo de construção do projeto, do que apenas explicar o que é a ferramenta. Além disso, não foi possível encontrar analise estática, plano de testes, dicionário de dados e métricas no desenvolvimento do próprio sistema, o que, consequentemente, não transmite muito credibilidade nessa parte em específico. 

O principal ponto positivo que encontramos lendo a documentação foi \textsl{design} dos protótipos de alta fidelidade, estão de fato bem produzidos e é possível encontrar uma identidade visual neles, aparenta ser um sistema com uma boa usabilidade e harmonia entre os seus componentes.

Sobre a apresentação, não foi permitido assisti-la no dia em que estava prevista, porém, gostaríamos de ter a oportunidade de continuar os acompanhando em seu processo de desenvolvimento e esperamos que os retornos aqui dados, ainda que estejam carentes de mais informações, possam contribuir para sua melhoria contínua.

%%%%%%%%%%%%%%%%%%%%%%%%%%%%%%%%%%%%%%%%%%%%%%%%%%%%%%%%%%%%%
\section{Equipe TechFive}

Considerando a apresentação (ocorrida no dia 26 de setembro de 2022) e a documentação do projeto Mini-Me, foi possível entender o projeto como uma aplicação móvel voltada para os pais de recém nascidos que procuram ter um maior controle sobre aspectos que compreender a vida do bebê, como vacinação, lactação, trocas de fraldas, entre outros. 

Em comparação com a análise feita anteriormente a equipe, é possível perceber diversos pontos de melhora, tanto na aplicação quanto na documentação, por exemplo, a análise das concorrências como parte da justificativa para a construção do Mini-Me, já que durante a apresentação, a clareza do objetivo foi melhor especificada quando feita esta comparação logo de início, assim detalhando melhor o problema a ser resolvido. Porém, nessa mesma questão de diferenças de outras aplicações, é colocado sobre “interface intuitiva” e para dizer que o Mini-Me é intuitivo é necessário atender certos requisitos, ter certas características que comprovem se um site é ou não, baseada em heurística ou \textsl{feedback} dos usuários, pois não havendo isso, os concorrentes também poderiam se considerar intuitivos.

A apresentação da equipe foi clara e objetiva e possibilitou um bom entendimento sobre o projeto na sua totalidade, porém uma questão que acabou não ficando muito clara e pode ser revisada é em relação ao back-end utilizado, considerando que a API é utilizando PHP, não ficou claro onde está sendo feita a utilização do Node JS no projeto. 

Ainda sobre a API, seria interessante pesquisar sobre como documentá-la e disponibilizá-la para consulta posteriormente, pois esse processo independe da publicação da aplicação em alguma plataforma - tomando como base a arquitetura utilizada - e facilita a compreensão dos métodos consumidos pela aplicação. Uma sugestão que poderia ajudar no processo de publicação, é pensar em transformar a forma como puxam os dados do back-end para dentro da arquitetura REST, pois assim poderiam manter os mesmos dados sincronizados em vários dispositivos para o mesmo usuário em um serviço que funciona separadamente a aplicação. Entendemos que o conteúdo sobre APIs REST para PHP é um tanto escasso, porém, uma alternativa seria pesquisar sobre como implementar uma API REST via Node JS utilizando o Express JS (que possui mais conteúdos gratuitos na Internet), sabendo que já estão prevendo o uso da tecnologia em seu planejamento.

De um modo geral, foi notável a grande evolução do projeto desde a última apresentação, principalmente considerando que a tecnologia utilizada (voltada para aplicações móveis) não é lecionada no curso e tendo em vista as dificuldades mencionadas pelas integrantes ao longo do processo. Ainda que não tenhamos conseguido testá-la, ao assistirmos o vídeo apresentado, foi possível compreender bem as partes do sistema, que possui uma boa usabilidade de um \textsl{design} de interface bastante coerente com a proposta objetivada.

\end{document}

