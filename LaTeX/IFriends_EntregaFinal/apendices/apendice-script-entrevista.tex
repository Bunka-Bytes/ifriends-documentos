\chapter{Entrevista de usabilidade}
\label{script_entrevista}

A seguinte seção visa apresentar o modelo de entrevista de usabilidade que será usado pela equipe de \textit{design} no desenvolvimento dos testes de usabilidade:

- Olá, \gls{friend}. Somos a equipe de \textit{design} do projeto \gls{ifriends} e estamos aqui para guiá-lo durante o desenvolvimento do teste que realizaremos hoje. Antes de começar, gostaríamos de lhe passar algumas informações básicas a respeito do que faremos. 

- Suponho que já imagine o motivo da sua presença aqui, mas vou repassá-lo brevemente. Bem, estamos pedindo para que alguns alunos aqui do \acs{ifsp} tentem usar uma plataforma que estamos desenvolvendo para ver se ele está cumprindo com o seu proposto e se está seguindo pelo caminho planejado. 

- Então no desenvolvimento dessa pesquisa gostaríamos de pedir que tente pensar alto, diga se está procurando alguma coisa, o que está tentando fazer, algo que está pensando, alguma dificuldade que encontrou ou até mesmo uma crítica. 

- Antes de mais nada, gostaríamos de deixar claro que não há necessidade ter medo da gente, tenha em mente que a nossa missão hoje é avaliar a plataforma e não a você, nosso usuário, portanto pode ficar tranquilo(a/e) que não tem nada de certo ou de errado, a gente apenas gostaria de saber se o nosso produto está funcionando ou não aos seus olhos, então sinta-se a vontade de falar com sinceridade.

- Mas quando falo com sinceridade, é com sinceridade mesmo. Por favor, não pense que alguma crítica pode estar ferindo nossos sentimentos, pois estamos fazendo isso justamente para melhorar nosso produto e melhorar a experiência de nossos usuários, então é fundamental ouvir respostas e reações honestas. 

- Em relação às dúvidas, você pode fazer perguntas no decorrer do teste, mas pode acontecer de eu não respondê-las de imediato, já que também temos interesse em saber como o usuário se comporta nesses momentos.  Mas se a dúvida continuar mesmo após finalizar o teste, prometo que tentarei respondê-las. E caso precise de uma pausa a qualquer momento, é só me avisar. Você tem alguma dúvida até o momento? 

- Perfeito! Iniciaremos a pesquisa com algumas perguntas: 
\begin{itemize}
    \item Me conta, é a sua primeira vez ou já participou de uma experiência parecida antes?
    \item Qual é o seu curso? Tem alguma coisa que gosta de fazer durante seu dia?
    \item Costuma navegar na internet com frequência, poderia me dar uma estimativa diária?
    \item Dento dessa estimativa, você julga que usa a internet em sua maioria para estudos, para o trabalho ou para o lazer? 
    \item Costuma participar de fóruns de mensagens de comunidades on-line?
    \item Costuma participar dos eventos divulgados pelo \acs{ifsp}?
    \item Quando você tem interesse em encontrar alguma informação a respeito da instituição ou do ensino, como você realiza essa procura, tem algum meio para conseguir essas informações? 
    \item Você sente ou sentiu dificuldades em encontrar informações a respeito do \acs{ifsp}, seja no ensino ou sobre a instituição?
\end{itemize}

- Ótimo, terminamos com as perguntas, agora partiremos à parte prática. - Abrir o site do navegador e apresentar a plataforma ao \gls{friend}. 

- Primeiro pedirei para você visualizar está página por um minuto e me faça uma pequena narrativa sobre o que lhe chama atenção, consegue ter uma noção de que tipo de sistema é, apenas visualizando a interface você consegue ter uma ideia de como usá-lo? Você pode baixar e subir a barra de rolagem, mas não clique em nenhum elemento por enquanto. - Permitir que o \gls{friend} visualize a página por um minuto.  

- Muito bem, agora preciso que você tente realizar algumas tarefas específicas, tudo bem?
\begin{itemize}
    \item A sua primeira tarefa será criar uma pergunta, por favor pode começar. E, mais uma vez, nos ajudará muito se você pensar alto enquanto estiver desenvolvendo o desafio;
    \item A sua seguinte tarefa é tentar responder à pergunta que acabou de criar;
    \item Como próxima tarefa, você deverá criar um evento;
    \item Por último, gostaria que tente pesquisar por algo que desejar; 
\end{itemize}

- Maravilha! Você nos ajudou bastante. Agora estamos partindo para as etapas finais, você tem alguma pergunta até o momento? 

- Bem, agora eu gostaria que você me dê um \textit{feedback} geral. Nele você pode falar coisas do tipo:
\begin{itemize}
    \item Como foi a sua experiência? 
    \item Achou muito ruim? Ou está razoável? 
    \item Foi difícil executar as tarefas?
    \item Teve algo que te incomodou muito?
    \item Você tem uma forte sugestão de melhoria? 
\end{itemize}

- Não precisa me dar muitos detalhes, pois após concluirmos a entrevista ainda teremos um formulário de satisfação que você poderá compartilhar com seus colegas e, por gentileza, convidá-los a utilizar nossa plataforma. Você tem alguma pergunta que queira fazer, agora que estamos finalizando?

- Então dou como concluída a nossa entrevista, gostaríamos de agradecer novamente pela sua participação e pela ajuda e espero que tenha gostado da experiência. 