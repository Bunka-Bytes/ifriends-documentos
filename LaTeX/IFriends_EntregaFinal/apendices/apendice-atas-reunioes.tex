\chapter{Atas das Reuniões}
\label{atasReunioes}
%----------------------------------------------------------------------------------

\section{1\textordmasculine bimestre}
\subsection{Planejamento - 15/03/2022}
\noindent Integrantes: Anaí Rojas, Jamilli Gioielli, José Roberto, Julia Romualdo, Kaiky Matsumoto \\
Local: \gls{discord}. \\
Pauta(s): Primeira reunião realizada pela equipe com o objetivo de planejar os passos iniciais do projeto, nela a equipe estabeleceu um contrato social na plataforma \gls{figma} contendo os combinados essenciais para o convívio social entre os componentes da equipe e também iniciou-se um \textsl{board} de ideias iniciais para o projeto.

\subsection{Planejamento/Alinhamento - 17/03/2022}
\noindent Integrantes:\,Anaí Rojas, Jamilli Gioielli, José Roberto, Julia Romualdo, Kaiky Matsumoto \\
Local: \gls{discord}.\\
Pauta(s): Reunião de planejamento e alinhamento, onde o foco da equipe esteve em discutir as ideias pré-selecionadas na reunião anterior para o projeto. Desde este primeiro momento a atenção da equipe se voltava principalmente para uma comunidade de dúvidas entre os estudantes do IF. \\
Para organização das tarefas e da equipe a plataforma \gls{notion}, onde o quadro de \textsl{kanban} se encontra, foi criado e organizado inicialmente. \\
A equipe decidiu inicialmente que realizará reuniões nos dias e horários das janelas entre as aulas, permitidas pela grade curricular.

\subsection{Planejamento/Alinhamento - 18/03/2022}
\noindent Integrantes: Anaí Rojas, Jamilli Gioielli, José Roberto, Julia Romualdo, Kaiky Matsumoto \\
Local: \gls{discord}.\\
Pauta(s): Reunião de planejamento onde a equipe discutiu as tarefas a serem realizadas e discutiu outras ferramentas de organização. \\
Criamos o canal do \gls{youtube}.\\
Realizamos a primeira postagem para o \textsl{blog} da equipe.\\
Amadurecemos ainda mais a ideia da comunidade, pensando em mais funcionalidades para agregar na aplicação e anotando as dúvidas a respeito do tema e do funcionamento.

\subsection{Alinhamento - 21/03/2022}
\noindent Integrantes: Anaí Rojas, Jamilli Gioielli, José Roberto, Julia Romualdo, Kaiky Matsumoto \\
Local: Saguão \acs{ifsp}.\\
Pauta(s): Reunião realizada antes da aula de \acs{pds} para a equipe discutir algumas dúvidas, perspectivas e ideias sobre o projeto, visando otimizar o tempo em sala de aula.

\subsection{Alinhamento - 25/03/2022}
\noindent Integrantes: Anaí Rojas, Jamilli Gioielli, José Roberto, Julia Romualdo, Kaiky Matsumoto \\
Local: \gls{discord}.\\
Pauta(s): Reunião de alinhamento, onde a equipe pesquisou em trabalhos anteriores a formatação do documento da proposta inicial para preparar as etapas. 
Aproveitamos também para melhorar o \textsl{layout} do \textsl{blog} da equipe.

\subsection{Planejamento/Alinhamento - 28/03/2022}
\noindent Integrantes:\,Anaí Rojas, Jamilli Gioielli, José Roberto, Julia Romualdo, Kaiky Matsumoto\\
Local: Laboratório \acs{ifsp}.\\
Pauta(s): Reunião de alinhamento e planejamento, onde a equipe conversou com os orientadores sobre soluções e funcionalidades para a proposta da comunidade. Neste ponto, resolvemos um problema antigo, validar que o usuário seja de fato aluno da instituição, a solução encontrada foi enviar um \textsl{e-mail} de validação para o \textsl{e-mail} institucional do aluno. Foi citado também outra proposta de projeto para a equipe realizar, uma plataforma de controle e gestão financeira. Porém, a equipe optou por seguir na proposta da comunidade.\\
Partimos para as tarefas, iniciando o desenvolvimento de um questionário direcionado aos alunos do instituto para estudar a viabilidade de criação do projeto.\\
Sobre o \textsl{blog}, finalizamos a melhoria de seu \textsl{layout} e definimos que as publicações serão realizadas aos sábados pela manhã.

\subsection{Planejamento/Retrospectiva - 02/04/2022}
\noindent Integrantes:\,Anaí Rojas, Jamilli Gioielli, José Roberto, Julia Romualdo, Kaiky Matsumoto \\
Local: \gls{discord}.\\
Pauta(s): Reunião de retrospectiva e planejamento das atividades, onde revisamos a publicação da semana no \textsl{blog} e as perguntas para a pesquisa de viabilidade. Realizamos a entrega sobre as tecnologias que serão utilizadas no desenvolvimento do projeto no \gls{moodle} da disciplina e definimos duas tarefas: divulgação da pesquisa de viabilidade a partir de segunda-feira para alunos e ex-alunos da instituição e gerenciamento do \textsl{backlog} para cada parte do projeto.

\subsection{Planejamento - 04/04/2022}
\noindent Integrantes:\,Anaí Rojas, Jamilli Gioielli, José Roberto, Julia Romualdo, Kaiky Matsumoto \\
Local: Laboratório \acs{ifsp}.\\
Pauta(s): Reunião de planejamento, onde a equipe após receber as orientações para a apresentação da proposta inicial, organizou as tarefas a serem realizadas por cada componente, organizou a formatação do documento para a proposta inicial e iniciou a divulgação do formulário para pesquisa de viabilidade da proposta.

\subsection{Alinhamento - 09/04/2022}
\noindent Integrantes: Anaí Rojas, Jamilli Gioielli, José Roberto, Julia Romualdo, Kaiky Matsumoto \\
Local: \gls{discord}.\\
Pauta(s): Reunião de alinhamento, onde a equipe se reuniu para realizar as atividades destinadas a apresentação da proposta inicial.

\subsection{Alinhamento - 10/04/2022}
\noindent Integrantes: Anaí Rojas, Jamilli Gioielli, José Roberto, Julia Romualdo, Kaiky Matsumoto \\
Local: \gls{discord}.\\
Pauta(s): Reunião de alinhamento, onde a equipe se reuniu para realizar as atividades destinadas a apresentação da proposta inicial.

\subsection{Retrospectiva - 12/04/2022}
\noindent Integrantes:\,Anaí Rojas, Jamilli Gioielli, José Roberto, Julia Romualdo, Kaiky Matsumoto \\
Local: Biblioteca \acs{ifsp}.\\
Pauta(s): Reunião retrospectiva, onde a equipe analisou como foi o processo para realizar a entrega da proposta inicial, desta forma foram levantados os pontos positivos: todos estarem reunidos em chamada para realizar as tarefas, conseguimos entregar o que era esperado e apesar das dificuldades enfrentadas a apresentação fluiu bem. Os pontos negativos: realizar muitas tarefas no final de semana ficou puxado para a equipe e não ter realizado um ensaio antes da apresentação, como melhoria, queremos marcar mais reuniões com os professores.

\subsection{Planejamento/Alinhamento - 17/04/2022}
\noindent Integrantes: Anaí Rojas, Jamilli Gioielli, José Roberto, Julia Romualdo, Kaiky Matsumoto \\
Local: \gls{discord}.\\
Pauta(s): Reunião de planejamento e alinhamento para a semana, onde a equipe planejou as próximas \textsl{sprints} da \acs{POC} juntamente com os professores e aproveitou para conversar sobre as avaliações das equipes em relação a apresentação da proposta inicial. 

\subsection{Planejamento - 18/04/2022}
\noindent Integrantes:\,Anaí Rojas, Jamilli Gioielli, José Roberto, Julia Romualdo, Kaiky Matsumoto. \\
Local: Laboratório \acs{ifsp}.\\
Pauta(s): Reunião de planejamento onde a equipe criou o \textsl{backlog} do produto com o uso das histórias de usuário e definiu as tarefas da semana, preparando-se para os dois épicos para a \acs{POC}.

\subsection{Alinhamento - 21/04/2022}
\noindent Integrantes: Anaí Rojas, Jamilli Gioielli, José Roberto, Julia Romualdo, Kaiky Matsumoto \\
Local: \gls{discord}.\\
Pauta(s): Reunião de alinhamento, onde a equipe terminou de elaborar as histórias de usuário, definiu as prioridades para a \acs{POC} e votou por meio do \textsl{Planning Poker} - descrito pela metodologia Scrum - para estimarmos os esforços necessários para a conclusão de cada história. Durante as discussões da equipe para a execução desta tarefa, muitos pontos sutis, mas que poderiam ser perigosos no futuro, foram levantados e anotados para discutirmos com os orientadores. 

\subsection{Alinhamento - 25/04/2022}
\noindent Integrantes: Anaí Rojas, Jamilli Gioielli, José Roberto, Julia Romualdo, Kaiky Matsumoto \\
Local: Laboratório \acs{ifsp}.\\
Pauta(s): Reunião de alinhamento onde a equipe conversou com os orientadores sobre as dúvidas levantadas na reunião anterior, durante a elaboração da tarefa de definição das histórias de usuários, apresentou os diagramas de entidade e relacionamento e o diagrama de casos de uso para serem alinhados corretamente. Iniciamos também as configurações para criação do vídeo do \gls{gource}.

\subsection{Alinhamento - 01/05/2022}
\noindent Integrantes: Anaí Rojas, Jamilli Gioielli, José Roberto, Julia Romualdo, Kaiky Matsumoto \\
Local: \gls{discord}.\\
Pauta(s): Reunião de alinhamento onde a equipe finalizou as histórias de usuário com base na discussão realizada em aula com os orientadores, alinhou o fluxo de usuário no sistema, os requisitos funcionais, não funcionais e as regras de negócio. Iniciamos o planejamento para realização da apresentação da \acs{POC} e como orientação dos professores, decidimos deixar para desenvolver o épico de Gestão de Eventos apenas se sobrar tempo. 

\subsection{Alinhamento - 03/05/2022}
\noindent Integrantes: Anaí Rojas, Jamilli Gioielli, José Roberto, Julia Romualdo, Kaiky Matsumoto \\
Local: \gls{discord}. \\
Pauta(s): Realização de tarefas para a \acs{POC}, configurações de ambiente e alinhamento da documentação.

\section{2\textordmasculine bimestre}

\subsection{Retrospectiva - 16/05/2022}
\noindent Integrantes: Anaí Rojas, Jamilli Gioielli, José Roberto, Julia Romualdo, Kaiky Matsumoto \\
Local: Laboratório \acs{ifsp}. \\
Pauta(s): Reunião retrospectiva referente ao desenvolvimento e entrega da \acs{POC}, onde a equipe chegou as seguintes conclusões: do que foi bom, entrega da \acs{api} completa; do que foi ruim, falta de tempo, pois mesmo sabendo como desenvolver um dos requisitos não sobrou tempo para fazer; e um problema enfrentado por nós e pelas outras equipes da turma foi a incerteza e a definição errada que construímos sobre a \acs{POC}, isso gerou uma dificuldade na definição do que era extremamente essencial para a entrega. Como plano de ação, a equipe buscará definir melhor o que é simples e objetivo para as próximas entregas, evitando perca de tempo e desgaste com tarefas que não são prioridade.

\subsection{Planejamento - 29/05/2022}
\noindent Integrantes: Anaí Rojas, Jamilli Gioielli, José Roberto, Julia Romualdo, Kaiky Matsumoto \\
Local: \gls{discord}. \\
Pauta(s): Reunião de planejamento onde a equipe encerrou a \gls{Sprint} que tinha como objetivo melhorar aquilo que tinha sido desenvolvido na \acs{POC} e descreveu as tarefas que foram realizadas por cada membro da equipe.

\subsection{Planejamento - 06/06/2022}
\noindent Integrantes: Anaí Rojas, Jamilli Gioielli, José Roberto, Julia Romualdo, Kaiky Matsumoto \\
Local: Laboratório \acs{ifsp}. \\
Pauta(s): Reunião de planejamento onde a equipe após ser orientada pelos professores definiu os tópicos que serão abordados na apresentação da primeira versão e a partir disso, definiu as tarefas a serem desenvolvidas por cada membro.

\subsection{Planejamento 13/06/2022}
\noindent Integrantes: Anaí Rojas, Jamilli Gioielli, José Roberto, Julia Romualdo, Kaiky Matsumoto \\
Local: \gls{discord}. \\
Pauta(s): Reunião de planejamento onde a equipe conversou e definiu melhor alguns pontos do cadastro, como: definição dos requisitos para uma senha válida e quais domínios de e-mails estarão disponíveis para que o usuário selecione. Como a entrega da primeira versão está prevista para a próxima semana, o objetivo é que até quinta feira esteja tudo documentado e até domingo as tarefas estejam concluídas. 

\noindent Iniciamos uma discussão sobre abordagens para utilizarmos num futuro plano de testes e usabilidade. Também aproveitamos para tirar do caminho uma tarefa pendente que tínhamos, da gravação do vídeo referente a apresentação da \acs{POC}.

\subsection{Retrospectiva 23/06/2022}
\noindent Integrantes: Anaí Rojas, Jamilli Gioielli, José Roberto, Julia Romualdo,\,Kaiky Matsumoto \\
Local: Sala \acs{ifsp}. \\
Reunião retrospectiva onde a equipe teve um momento dinâmico de conversa sobre a entrega e apresentação da primeira versão do sistema. Definiu-se como pontos positivos: a comunicação assíncrona através do \gls{WhatsApp} que ajudou a resolver os imprevistos de maneira mais rápida, a apresentação como um todo e o \textit{feedback} positivos dos professores sobre ela. De pontos negativos: falta de um escopo definido sobre a apresentação,  insegurança sobre o que era realmente esperado, o tempo de apresentação passou um pouco do que era estipulado para cada equipe, os \textit{commits} não frequentes durante o desenvolvimento e as histórias de usuário muito grandes durante a \gls{Sprint}. Em pontos a serem melhorados: alimentar mais o \textit{blog} e o canal do \gls{youtube}, usar o \gls{github} como repositório secundário, ter comunicação transparente sobre o que está acontecendo no projeto e preparar uma estratégia de apresentação de acordo com o que for esperado. E um elogio levantado de maneira geral pela equipe é o empenho e dedicação de todos os integrantes, que querem fazer este projeto da melhor maneira possível para que o sistema realmente seja útil para outros alunos. 

\subsection{Planejamento 26/06/2022}
\noindent Integrantes: Anaí Rojas, Jamilli Gioielli, José Roberto, Julia Romualdo,\,Kaiky Matsumoto \\
Local: \gls{discord}. \\
Reunião de planejamento onde a equipe criou uma \gls{Sprint} com melhorias pequenas e detalhes que pudessem ser realizados sem muito esforço, visto que o tempo hábil a ser investido no projeto estava curto devido as outras disciplinas. Em um segundo momento um \textit{backlog} foi elaborado com detalhes e planejamento dos próximos passos a serem realizados no sistema e da documentação que vão ser usados para criar as próximas \glspl{Sprint} com objetivo de serem executadas durante as férias.

\subsection{Planejamento 23/07/2022}
\noindent Integrantes: Anaí Rojas, Jamilli Gioielli, José Roberto, Julia Romualdo, Kaiky Matsumoto \\
Local: \gls{discord}. \\
Reunião de planejamento onde a equipe mediu quantas semanas restam até a próxima entrega, que pode ser considerada a final e quantas horas cada integrante consegue se dedicar ao projeto diariamente durante o período de férias e de aulas. Com essa métrica foi elaborado um plano de entregas para que toda semana uma entrega pequena seja feita. Por último, os tópicos a serem adicionados na revisão de literatura foram abordados e organizados de forma que façam sentido no escopo do documento.

\subsection{Planejamento 28/07/2022}
\noindent Integrantes: Anaí Rojas, Jamilli Gioielli, José Roberto, Julia Romualdo,\,Kaiky Matsumoto \\
Local: \gls{discord}. \\
Reunião de planejamento onde a equipe elaborou, quebrou e refinou algumas histórias de usuário. Aproveitou-se para definir quais histórias são essenciais para a próxima entrega, para que elas já possam começar a ser desenvolvidas durante as férias. Em um segundo momento uma maior atenção foi dada ao protótipo, pensou-se que a usabilidade do sistema não fazia muito sentido, muitos elementos poderiam ser retirados ou substituídos e por isso uma maior atenção será dada a reorganização do \textit{layout} do sistema. Além disso, as categorias foram definidas nos tópicos: ensino, esportes, estágio, entretenimento, institucional e outros.

\subsection{Planejamento 04/08/2022}
\noindent Integrantes: Anaí Rojas, Jamilli Gioielli, José Roberto, Julia Romualdo, Kaiky Matsumoto \\
Local: \gls{discord}. \\
Reunião de planejamento onde a equipe pesquisou e definiu as alterações principais para o protótipo e consequentemente, \textit{layout} do sistema. Foi definido que as pesquisas iniciais para a revisão de literatura deviam começar no final de semana. E por último, as histórias de usuário foram medidas por meio do \textit{planing poker}.  

\section{3\textordmasculine bimestre}

\subsection{Planejamento 08/08/2022}
\noindent Integrantes: Anaí Rojas, Jamilli Gioielli, José Roberto, Julia Romualdo,\,Kaiky Matsumoto \\
Local: Laboratório \acs{ifsp}. \\
Reunião de planejamento realizada pela equipe com a participação de um dos professores onde a equipe apresentou as mudanças realizadas no \textit{layout} do sistema e discutiu o funcionamento dos filtros por categoria. Além disso, com o tempo restante em aula, a equipe quebrou mais histórias de usuário para que o desenvolvimento seja facilitado o máximo possível.

\subsection{Planejamento/Alinhamento 15/08/2022}
\noindent Integrantes: Anaí Rojas, Jamilli Gioielli, José Roberto, Julia Romualdo, Kaiky Matsumoto \\
Local: Laboratório \acs{ifsp}. \\
Reunião de planejamento e alinhamento onde juntamente com os professores a equipe foi convencida de que precisava implantar um sistemas de denúncias no site para a próxima entrega, por isso trabalhou para que a questão fosse solucionada de forma rápida. Assim, definiu-se que um botão será adicionado ao sistema onde o usuário pode enviar um \textit{ e-mail} aos administradores contendo sua denúncia. 

\section{4\textordmasculine bimestre}
\subsection{Retrospectiva 17/10/2022}
\noindent Integrantes: Anaí Rojas, Jamilli Gioielli, José Roberto, Julia Romualdo, Kaiky Matsumoto \\
Local: Laboratório \acs{ifsp}. \\
Reunião retrospectiva relativa à entrega e apresentação final do projeto, onde foi definido o que foi bom: reuniões menos frequentes e maior foco na realização das tarefas, como muito tempo havia sido dedicado em reuniões de planejamento para definição do \textit{backlog}, economizou-se esse tempo que foi utilizado para desenvolvimento das tarefas. O que poderia melhorar: comunicação assíncrona pelo \gls{WhatsApp}, os quadros no \gls{notion} eram  atualizados mas a equipe falhou na comunicação interna sobre o andamento do projeto durante a semana. O plano de ação definido foi: enviar mensagem no grupo do \gls{WhatsApp} no sábado ou domingo, falando sobre as tarefas desenvolvidas durante a semana.

