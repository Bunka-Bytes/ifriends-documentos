\chapter{Histórias de usuário}
\label{historias de usuario}
% ----------------------------------------------------------
A seguinte seção apresentará as histórias de usuário consideradas até a última \gls{Sprint} do projeto \gls{ifriends} classificadas por seus respectivos épicos.
%--------------------------------------------------------------------------------
\section{Épico - Gestão de eventos}
\label{gestão_eventos}
%--------------------------------------------------------------------------------
%%%%%%% História - Criar e exibir eventos %%%%%%%
Para a história de usuário do \autoref{História: Criar e exibir eventos} foram definidos os seguintes itens como critérios de aceitação:

\begin{itemize}
\item É necessária uma aba com a listagem de todos os eventos publicados separados em ordem cronológica;
\end{itemize}

\def\arraystretch{2}
\begin{quadro}[htb]
\centering
\ABNTEXfontereduzida
\caption[História: Criar e exibir eventos]{História: Criar e exibir eventos}
\label{História: Criar e exibir eventos}
\resizebox{\linewidth}{!}{
\begin{tabular}{|p{6.5cm}|c|c|c|c|}
\hline
\thead{Descrição} & \thead{Épico} & \thead{Pontuação} & \thead{Tamanho} & \thead{Prioridade}\\
\hline

Como \gls{friend}, eu gostaria de criar eventos para ajudar meus colegas com suas dúvidas sobre um assunto & Gestão de Eventos & 3 & Pequeno & ALTA \\ \hline

\end{tabular}}\legend{Fonte: Os autores}
\end{quadro}
\FloatBarrier 

%%%%%%% História - Editar eventos %%%%%%% 
Para a história de usuário do \autoref{História: Editar eventos} foram definidos os seguintes itens como critérios de aceitação:

\begin{itemize}
\item Abrir o campo de evento para edição;
\item Deixar o botão para salvar;
\item Deve haver uma data e hora de edição;
\item O evento deve ser sinalizado como editado;
\end{itemize}

\def\arraystretch{2}
\begin{quadro}[htb]
\centering
\ABNTEXfontereduzida
\caption[História: Editar eventos]{História: Editar eventos}
\label{História: Editar eventos}
\resizebox{\linewidth}{!}{
\begin{tabular}{|p{6.5cm}|c|c|c|c|}
\hline
\thead{Descrição} & \thead{Épico} & \thead{Pontuação} & \thead{Tamanho} & \thead{Prioridade}\\
\hline

Como \gls{friend}, eu gostaria de editar os eventos caso exista algum dado com necessidade de alteração & Gestão de Eventos & 3 & Pequeno & ALTA \\ \hline

\end{tabular}}\legend{Fonte: Os autores}
\end{quadro}
\FloatBarrier 

%%%%%%% História - Excluir eventos %%%%%%% 
Para a história de usuário do \autoref{História: Excluir eventos} foram definidos os seguintes itens como critérios de aceitação:

\begin{itemize}
\item Avisar ao \gls{friend} sobre a exclusão de um evento favoritado;
\end{itemize}

\def\arraystretch{2}
\begin{quadro}[htb]
\centering
\ABNTEXfontereduzida
\caption[História: Excluir eventos]{História: Excluir eventos}
\label{História: Excluir eventos}
\resizebox{\linewidth}{!}{
\begin{tabular}{|p{6.5cm}|c|c|c|c|}
\hline
\thead{Descrição} & \thead{Épico} & \thead{Pontuação} & \thead{Tamanho} & \thead{Prioridade}\\
\hline

Como \gls{friend}, eu gostaria de excluir um evento para removê-lo da visualiação & Gestão de Eventos & 3 & Pequeno & ALTA \\ \hline

\end{tabular}}\legend{Fonte: Os autores}
\end{quadro}
\FloatBarrier 

%%%%%%% História - Favoritar os eventos %%%%%%% 

Para a história de usuário do \autoref{História: Favoritar evento} foram definidos os seguintes itens como critérios de aceitação:

\begin{itemize}
\item  Deve haver um contador de número de interessados no evento;
\end{itemize}

\def\arraystretch{2}
\begin{quadro}[htb]
\centering
\ABNTEXfontereduzida
\caption[História: Favoritar evento]{História: Favoritar evento}
\label{História: Favoritar evento}
\resizebox{\linewidth}{!}{
\begin{tabular}{|p{6.5cm}|c|c|c|c|}
\hline
\thead{Descrição} & \thead{Épico} & \thead{Pontuação} & \thead{Tamanho} & \thead{Prioridade}\\
\hline

Como \gls{friend}, eu gostaria de favoritar um evento para expressar meu interesse em atendê-lo. & Gestão de Eventos & 2 & Pequeno & ALTA \\ \hline

\end{tabular}}\legend{Fonte: Os autores}
\end{quadro}
\FloatBarrier 

%%%%%%% História - Listagem de categorias %%%%%%% 
Para a história de usuário do \autoref{História: Listagem de categorias} foram definidos os seguintes itens como critérios de aceitação:

\begin{itemize}
\item Lista de categorias pré-definidas: ensino, estágio, esportes, entretenimento, institucional e outros;  
\end{itemize}

\def\arraystretch{2}
\begin{quadro}[htb]
\centering
\ABNTEXfontereduzida
\caption[História: Listagem de categorias]{História: Listagem de categorias}
\label{História: Listagem de categorias}
\resizebox{\linewidth}{!}{
\begin{tabular}{|p{6.5cm}|c|c|c|c|}
\hline
\thead{Descrição} & \thead{Épico} & \thead{Pontuação} & \thead{Tamanho} & \thead{Prioridade}\\
\hline

Como \gls{friend}, eu gostaria de visualizar a lista de categorias mais para facilitar a busca de temas mais relevantes & Gestão de Eventos & 5 & Médio & ALTA \\ \hline

\end{tabular}}\legend{Fonte: Os autores}
\end{quadro}
\FloatBarrier 

%--------------------------------------------------------------------------------
\section{Épico - Moderação}
\label{gestão_moderacao}
%--------------------------------------------------------------------------------
%%%%%%% História - Denúncia de perguntas, respostas e eventos - Backend %%%%%%% 
Para a história de usuário do \autoref{História: Denúncia de perguntas, respostas e eventos - Backend} foram definidos os seguintes itens como critérios de aceitação:

\begin{itemize}
\item Deve ser criado um modal para envio dos detalhes da denúncia;
\item O modal deve conter dois campos de formulário: o campo de assunto da denúncia e o campo de comentário; 
\end{itemize}

\def\arraystretch{2}
\begin{quadro}[htb]
\centering
\ABNTEXfontereduzida
\caption[História: Denúncia de perguntas, respostas e eventos - Backend]{História: Denúncia de perguntas, respostas e eventos - back-end}
\label{História: Denúncia de perguntas, respostas e eventos - Backend}
\resizebox{\linewidth}{!}{
\begin{tabular}{|p{6.5cm}|c|c|c|c|}
\hline
\thead{Descrição} & \thead{Épico} & \thead{Pontuação} & \thead{Tamanho} & \thead{Prioridade}\\
\hline

Como moderador, eu preciso receber um email com os dados da denúncia para avaliá-la. & Moderação & 5 & Médio & ALTA \\ \hline

\end{tabular}}\legend{Fonte: Os autores}
\end{quadro}
\FloatBarrier 

%%%%%%% História - Denúncia de perguntas, respostas e eventos - Frontend %%%%%%% 
Para a história de usuário do \autoref{História: Denúncia de perguntas, respostas e eventos - Frontend} foram definidos os seguintes itens como critérios de aceitação:

\begin{itemize}
\item Deve ser criado um modal para envio dos detalhes da denúncia;
\item O modal deve conter dois campos de formulário: o campo de assunto da denúncia e o campo de comentário; 
\end{itemize}

\def\arraystretch{2}
\begin{quadro}[htb]
\centering
\ABNTEXfontereduzida
\caption[História: Denúncia de perguntas, respostas e eventos - Frontend]{História: Denúncia de perguntas, respostas e eventos - front-end}
\label{História: Denúncia de perguntas, respostas e eventos - Frontend}
\resizebox{\linewidth}{!}{
\begin{tabular}{|p{6.5cm}|c|c|c|c|}
\hline
\thead{Descrição} & \thead{Épico} & \thead{Pontuação} & \thead{Tamanho} & \thead{Prioridade}\\
\hline

Como \gls{friend}, eu preciso denunciar uma pergunta, resposta ou evento para que possa contribuir e cumprir com as normas da comunidade & Moderação & 5 & Médio & ALTA \\ \hline

\end{tabular}}\legend{Fonte: Os autores}
\end{quadro}
\FloatBarrier 

%%%%%%% História - Exibir modal dos termos de uso - Frontend %%%%%%% 
Para a história de usuário do \autoref{História: Exibir modal dos termos de uso-Frontend} foram definidos os seguintes itens como critérios de aceitação:

\begin{itemize}
\item Termos de uso disponíveis para tradução;
\end{itemize}

\def\arraystretch{2}
\begin{quadro}[htb]
\centering
\ABNTEXfontereduzida
\caption[História: Exibir modal dos termos de uso - Frontend]{História: Exibir modal dos termos de uso - fron-tend}
\label{História: Exibir modal dos termos de uso-Frontend}
\resizebox{\linewidth}{!}{
\begin{tabular}{|p{6.5cm}|c|c|c|c|}
\hline
\thead{Descrição} & \thead{Épico} & \thead{Pontuação} & \thead{Tamanho} & \thead{Prioridade}\\
\hline

Como \gls{friend}, eu quero visualizar os termos de uso para saber se estou ciente com as cláusulas de uso da comunidade. & Moderação & 5 & Médio & ALTA \\ \hline

\end{tabular}}\legend{Fonte: Os autores}
\end{quadro}
\FloatBarrier

%%%%%%% História - Exibir painel de denúncias %%%%%%% 
Para a história de usuário do \autoref{História: Exibir painel de denúncias} foram definidos os seguintes itens como critérios de aceitação:

\begin{itemize}
\item Denúncias disponíveis para avaliação;
\end{itemize}

\def\arraystretch{2}
\begin{quadro}[htb]
\centering
\ABNTEXfontereduzida
\caption[História: Exibir painel de denúncias]{História: Exibir painel de denúncias}
\label{História: Exibir painel de denúncias}
\resizebox{\linewidth}{!}{
\begin{tabular}{|p{6.5cm}|c|c|c|c|}
\hline
\thead{Descrição} & \thead{Épico} & \thead{Pontuação} & \thead{Tamanho} & \thead{Prioridade}\\
\hline

Como administrador do sistema, gostaria de poder visualizar as denúncias feitas e tomar ações a partir disso. & Moderação & 8 & Médio & BAIXA \\ \hline

\end{tabular}}\legend{Fonte: Os autores}
\end{quadro}
\FloatBarrier

%--------------------------------------------------------------------------------
\section{Épico - Gestão de perguntas}
\label{gestão_perguntas}

%%%%%%% História - Adicionar tags %%%%%%%  
Para a história de usuário do \autoref{História: Adicionar tags} foram definidos os seguintes itens como critérios de aceitação:

\begin{itemize}
\item Deve existir o campo para adicionar as \textit{tags} na pergunta;
\item As tags são complementos para a categoria da pergunta; 
\end{itemize}

\def\arraystretch{2}
\begin{quadro}[htb]
\centering
\ABNTEXfontereduzida
\caption[História: Adicionar tags]{História: Adicionar tags}
\label{História: Adicionar tags}
\resizebox{\linewidth}{!}{
\begin{tabular}{|p{6.5cm}|c|c|c|c|}
\hline
\thead{Descrição} & \thead{Épico} & \thead{Pontuação} & \thead{Tamanho} & \thead{Prioridade}\\
\hline

Como \gls{friend}, eu gostaria de adicionar \textit{tags} nas minhas perguntas para encontrá-las mais facilmente & Gestão de Perguntas & 3 & Pequeno & ALTA \\ \hline

\end{tabular}}\legend{Fonte: Os autores}
\end{quadro}
\FloatBarrier 

%--------------------------------------------------------------------------------
%%%%%%% História - Buscar Perguntas %%%%%%% 
Para a história de usuário do \autoref{História: Buscar perguntas} foram definidos os seguintes itens como critérios de aceitação:

\begin{itemize}
\item O usuário precisa informar total ou parcialmente o título da pergunta desejada;
\item As perguntas serão exibidas conforme as informações passadas, podendo ser semelhantes parcial ou totalmente;
\end{itemize}

\def\arraystretch{2}
\begin{quadro}[htb]
\centering
\ABNTEXfontereduzida
\caption[História: Buscar perguntas]{História: Buscar perguntas}
\label{História: Buscar perguntas}
\resizebox{\linewidth}{!}{
\begin{tabular}{|p{6.5cm}|c|c|c|c|}
\hline
\thead{Descrição} & \thead{Épico} & \thead{Pontuação} & \thead{Tamanho} & \thead{Prioridade}\\
\hline

Como \gls{friend}, eu gostaria de buscar perguntas feitas para que possa consultar uma pergunta específica & Gestão de Perguntas & 2 & Pequeno & MÉDIA \\ \hline

\end{tabular}}\legend{Fonte: Os autores}
\end{quadro}
\FloatBarrier 

%%%%%%% História - Contagem de visualização de perguntas %%%%%%%
Para a história de usuário do \autoref{História: Contagem de visualização de perguntas} foram definidos os seguintes itens como critérios de aceitação:

\begin{itemize}
\item Visualização disponível no \textit{card} da pergunta;
\end{itemize}

\def\arraystretch{2}
\begin{quadro}[htb]
\centering
\ABNTEXfontereduzida
\caption[História: Contagem de visualização de perguntas]{História: Contagem de visualização de perguntas}
\label{História: Contagem de visualização de perguntas}
\resizebox{\linewidth}{!}{
\begin{tabular}{|p{6.5cm}|c|c|c|c|}
\hline
\thead{Descrição} & \thead{Épico} & \thead{Pontuação} & \thead{Tamanho} & \thead{Prioridade}\\
\hline

Como \gls{friend}, eu gostaria de verificar quantas vezes uma pergunta foi visualizada para identificar as perguntas mais vistas & Gestão de Perguntas & 5 & Médio & ALTA \\ \hline

\end{tabular}}\legend{Fonte: Os autores}
\end{quadro}
\FloatBarrier

%%%%%%%  História - Criar uma pergunta %%%%%%% 
Para a história de usuário do \autoref{Criar uma pergunta} foram definidos os seguintes itens como critérios de aceitação:

\begin{itemize}
\item Mostrar ``como fazer uma boa pergunta'';
\item O usuário deve conseguir somente criar uma pergunta da visualização;
\end{itemize}

\def\arraystretch{2}
\begin{quadro}[htb]
\centering
\ABNTEXfontereduzida
\caption[História: Criar uma pergunta]{História: Criar uma pergunta}
\label{Criar uma pergunta}
\resizebox{\linewidth}{!}{
\begin{tabular}{|p{6.5cm}|c|c|c|c|}
\hline
\thead{Descrição} & \thead{Épico} & \thead{Pontuação} & \thead{Tamanho} & \thead{Prioridade}\\
\hline

Como \gls{friend}, eu gostaria de criar uma pergunta na comunidade para retirar uma dúvida & Gestão de Perguntas & 13 & Grande & ALTA \\ \hline

\end{tabular}}\legend{Fonte: Os autores}
\end{quadro}
\FloatBarrier 

%%%%%%% História - Curtir uma pergunta %%%%%%% 
Para a história de usuário do \autoref{História: Curtir uma pergunta}  foram definidos os seguintes itens como critérios de aceitação:

\begin{itemize}
\item Um usuário só poderá votar uma única vez;
\item Cada voto equivale a um ponto;
\item Soma dos pontos por pergunta deve ser exibida;
\end{itemize}

\def\arraystretch{2}
\begin{quadro}[htb]
\centering
\ABNTEXfontereduzida
\caption[História: Curtir uma pergunta]{História: Curtir uma pergunta}
\label{História: Curtir uma pergunta}
\resizebox{\linewidth}{!}{
\begin{tabular}{|p{6.5cm}|c|c|c|c|}
\hline
\thead{Descrição} & \thead{Épico} & \thead{Pontuação} & \thead{Tamanho} & \thead{Prioridade}\\
\hline

Como \gls{friend}, eu gostaria de votar em uma pergunta para indicar se ela me foi útil ou não. & Gestão de Perguntas & 2 & Pequeno & ALTA \\ \hline

\end{tabular}}\legend{Fonte: Os autores}
\end{quadro}
\FloatBarrier 

%%%%%%% História - Deletar uma pergunta %%%%%%% 
Para a história de usuário do \autoref{História: Deletar uma pergunta} foram definidos os seguintes itens como critérios de aceitação:

\begin{itemize}
\item A pergunta não é deletada do banco de dados, apenas removida logicamente, isto é, não pode ser visualizada;
\end{itemize}

\def\arraystretch{2}
\begin{quadro}[htb]
\centering
\ABNTEXfontereduzida
\caption[História: Deletar uma pergunta]{História: Deletar uma pergunta}
\label{História: Deletar uma pergunta}
\resizebox{\linewidth}{!}{
\begin{tabular}{|p{6.5cm}|c|c|c|c|}
\hline
\thead{Descrição} & \thead{Épico} & \thead{Pontuação} & \thead{Tamanho} & \thead{Prioridade}\\
\hline

Como \gls{friend}, eu gostaria de excluir minha pergunta para removê-la da visualização caso necessário & Gestão de Perguntas & 3 & Pequeno & ALTA \\ \hline

\end{tabular}}\legend{Fonte: Os autores}
\end{quadro}
\FloatBarrier 

%%%%%%% História - Editar uma pergunta %%%%%%% 
Para a história de usuário do \autoref{História: Editar uma pergunta} foram definidos os seguintes itens como critérios de aceitação:

\begin{itemize}
\item Deve haver uma data e hora de edição;
\item A pergunta deve ser sinalizada como editada;
\end{itemize}

\def\arraystretch{2}
\begin{quadro}[htb]
\centering
\ABNTEXfontereduzida
\caption[História: Editar uma pergunta]{História: Editar uma pergunta}
\label{História: Editar uma pergunta}
\resizebox{\linewidth}{!}{
\begin{tabular}{|p{6.5cm}|c|c|c|c|}
\hline
\thead{Descrição} & \thead{Épico} & \thead{Pontuação} & \thead{Tamanho} & \thead{Prioridade}\\
\hline

Como \gls{friend}, eu gostaria de editar os dados da minha pergunta quando julgue necessário & Gestão de Perguntas & 5 & Médio & ALTA \\ \hline

\end{tabular}}\legend{Fonte: Os autores}
\end{quadro}
\FloatBarrier 

%%%%%%% História - Filtrar pergunta %%%%%%% 
Para a história de usuário do \autoref{História: Filtrar perguntas} foram definidos os seguintes itens como critérios de aceitação:

\begin{itemize}
\item Interface de usuário responsiva;
\item Texto disponível em inglês e português;
\item Tipos de filtros de perguntas: por tags, sem resposta (ninguém respondeu) e sem resposta aceita.
\end{itemize}

\def\arraystretch{2}
\begin{quadro}[htb]
\centering
\ABNTEXfontereduzida
\caption[História: Filtrar perguntas]{História: Filtrar perguntas}
\label{História: Filtrar perguntas}
\resizebox{\linewidth}{!}{
\begin{tabular}{|p{6.5cm}|c|c|c|c|}
\hline
\thead{Descrição} & \thead{Épico} & \thead{Pontuação} & \thead{Tamanho} & \thead{Prioridade}\\
\hline

Como \gls{friend}, eu gostaria de filtrar perguntas para que possa encontrar as mais relevantes. & Gestão de Perguntas & 8 & Médio & MÉDIA \\ \hline

\end{tabular}}\legend{Fonte: Os autores}
\end{quadro}
\FloatBarrier 

%%%%%%% História - Listagem de categorias %%%%%%% 
Para a história de usuário do \autoref{História: Listagem de categorias perguntas} foram definidos os seguintes itens como critérios de aceitação:

\begin{itemize}
\item Lista de categorias pré-definidas: ensino, estágio, esportes, entretenimento, institucional e outros;  
\end{itemize}

\def\arraystretch{2}
\begin{quadro}[htb]
\centering
\ABNTEXfontereduzida
\caption[História: Listagem de categorias]{História: Listagem de categorias}
\label{História: Listagem de categorias perguntas}
\resizebox{\linewidth}{!}{
\begin{tabular}{|p{6.5cm}|c|c|c|c|}
\hline
\thead{Descrição} & \thead{Épico} & \thead{Pontuação} & \thead{Tamanho} & \thead{Prioridade}\\
\hline

Como \gls{friend}, eu gostaria de visualizar a lista de categorias para facilitar a busca de temas mais relevantes & Gestão de Perguntas & 5 & Médio & ALTA \\ \hline

\end{tabular}}\legend{Fonte: Os autores}
\end{quadro}
\FloatBarrier 

%%%%%%% História - Marcar pergunta como resolvida %%%%%%% 
Para a história de usuário do \autoref{História: Marcar pergunta como resolvida} foram definidos os seguintes itens como critérios de aceitação:

\begin{itemize}
\item O usuário deve conseguir decidir se a resposta resolveu seu problema ou não;
\item Opção de marcar como respondida disponível no card de pergunta; 
\end{itemize}

\def\arraystretch{2}
\begin{quadro}[htb]
\centering
\ABNTEXfontereduzida
\caption[História: Marcar pergunta como resolvida]{História: Marcar pergunta como resolvida}
\label{História: Marcar pergunta como resolvida}
\resizebox{\linewidth}{!}{
\begin{tabular}{|p{6.5cm}|c|c|c|c|}
\hline
\thead{Descrição} & \thead{Épico} & \thead{Pontuação} & \thead{Tamanho} & \thead{Prioridade}\\
\hline

Como \gls{friend}, eu gostaria de marcar uma pergunta como resolvida para notificar a resolução do meu problema & Gestão de Perguntas & 5 & Médio & MÉDIA \\ \hline

\end{tabular}}\legend{Fonte: Os autores}
\end{quadro}
\FloatBarrier 

%%%%%%% História - Ordenação das perguntas %%%%%%% 
Para a história de usuário do \autoref{História: Ordenação das perguntas} foram definidos os seguintes itens como critérios de aceitação:

\begin{itemize}
\item mais curtidas;
\item mais recentes;
\end{itemize}

\def\arraystretch{2}
\begin{quadro}[htb]
\centering
\ABNTEXfontereduzida
\caption[História: Ordenação das perguntas]{História: Ordenação das perguntas}
\label{História: Ordenação das perguntas}
\resizebox{\linewidth}{!}{
\begin{tabular}{|p{6.5cm}|c|c|c|c|}
\hline
\thead{Descrição} & \thead{Épico} & \thead{Pontuação} & \thead{Tamanho} & \thead{Prioridade}\\
\hline

Como \gls{friend}, eu gostaria de ver as perguntas de acordo com uma ordenação, para encontrar as mais recentes e mais curtidas antes & Gestão de Perguntas & 5 & Médio & ALTA \\ \hline

\end{tabular}}\legend{Fonte: Os autores}
\end{quadro}
\FloatBarrier 

%--------------------------------------------------------------------------------
\section{Épico - Gestão de respostas}
\label{gestão_respostas}
%--------------------------------------------------------------------------------
%%%%%%% História - Aceitar uma resposta %%%%%%%
Para a história de usuário do \autoref{História: Aceitar uma resposta} foram definidos os seguintes itens como critérios de aceitação:

\begin{itemize}
\item Opção de marcar resposta como aceita disponível no \textit{card} de resposta. 
\end{itemize}

\def\arraystretch{2}
\begin{quadro}[htb]
\centering
\ABNTEXfontereduzida
\caption[História: Aceitar uma resposta]{História: Aceitar uma resposta}
\label{História: Aceitar uma resposta}
\resizebox{\linewidth}{!}{
\begin{tabular}{|p{6.5cm}|c|c|c|c|}
\hline
\thead{Descrição} & \thead{Épico} & \thead{Pontuação} & \thead{Tamanho} & \thead{Prioridade}\\
\hline

Como \gls{friend}, desejo marcar uma resposta como aceita para informar que ela resolveu meu problema. & Gestão de Respostas & 5 & Médio & MÉDIA \\ \hline

\end{tabular}}\legend{Fonte: Os autores}
\end{quadro}
\FloatBarrier 

%%%%%%% História - Criar uma resposta %%%%%%%
Para a história de usuário do \autoref{História: Criar uma resposta} foram definidos os seguintes itens como critérios de aceitação:

\begin{itemize}
\item As respostas mais curtidas devem ser exibidas antes das demais;
\item O usuário deve conseguir somente criar e deletar uma resposta;
\item Todas as respostas devem ser exibidas sem exceção;
\end{itemize}

\def\arraystretch{2}
\begin{quadro}[htb]
\centering
\ABNTEXfontereduzida
\caption[História: Criar uma resposta]{História: Criar uma resposta}
\label{História: Criar uma resposta}
\resizebox{\linewidth}{!}{
\begin{tabular}{|p{6.5cm}|c|c|c|c|}
\hline
\thead{Descrição} & \thead{Épico} & \thead{Pontuação} & \thead{Tamanho} & \thead{Prioridade}\\
\hline

Como \gls{friend}, eu gostaria de criar uma resposta para retirar uma dúvida de um colega. & Gestão de Respostas & 5 & Médio & ALTA \\ \hline

\end{tabular}}\legend{Fonte: Os autores}
\end{quadro}
\FloatBarrier 

%%%%%%% História - Curtir uma resposta %%%%%%% 
Para a história de usuário do \autoref{História: Curtir uma resposta} foram definidos os seguintes itens como critérios de aceitação:

\begin{itemize}
\item Um usuário só poderá curtir uma única vez;
\item Cada curtida equivale a um ponto;
\item Soma das curtidas por pergunta deve ser exibida;
\end{itemize}

\def\arraystretch{2}
\begin{quadro}[htb]
\centering
\ABNTEXfontereduzida
\caption[História: Curtir uma resposta]{História: Curtir uma resposta}
\label{História: Curtir uma resposta}
\resizebox{\linewidth}{!}{
\begin{tabular}{|p{6.5cm}|c|c|c|c|}
\hline
\thead{Descrição} & \thead{Épico} & \thead{Pontuação} & \thead{Tamanho} & \thead{Prioridade}\\
\hline

Como \gls{friend}, eu gostaria de curtir uma resposta para indicar se ela me foi útil ou não. & Gestão de Respostas & 1 & Pequeno & ALTA \\ \hline

\end{tabular}}\legend{Fonte: Os autores}
\end{quadro}
\FloatBarrier 

%%%%%%% História - Editar uma resposta %%%%%%% 
Para a história de usuário do \autoref{História: Editar uma resposta} foram definidos os seguintes itens como critérios de aceitação:

\begin{itemize}
\item Abrir o campo de resposta para edição;
\item Deixar o botão para salvar;
\item Deve haver uma data e hora de edição;
\item A resposta deve ser sinalizada como editada;
\end{itemize}

\def\arraystretch{2}
\begin{quadro}[htb]
\centering
\ABNTEXfontereduzida
\caption[História: Editar uma resposta]{História: Editar uma resposta}
\label{História: Editar uma resposta}
\resizebox{\linewidth}{!}{
\begin{tabular}{|p{6.5cm}|c|c|c|c|}
\hline
\thead{Descrição} & \thead{Épico} & \thead{Pontuação} & \thead{Tamanho} & \thead{Prioridade}\\
\hline

Como \gls{friend}, eu gostaria de excluir minha resposta para removê-la da visualização caso necessário & Gestão de Respostas & 5 & Médio & ALTA \\ \hline

\end{tabular}}\legend{Fonte: Os autores}
\end{quadro}
\FloatBarrier

%%%%%%% História - Excluir uma resposta %%%%%%% 
Para a história de usuário do \autoref{História: Excluir uma resposta} foram definidos os seguintes itens como critérios de aceitação:

\begin{itemize}
\item O \gls{friend} deve conseguir deletar (logicamente) uma resposta;
\end{itemize}

\def\arraystretch{2}
\begin{quadro}[htb]
\centering
\ABNTEXfontereduzida
\caption[História: Excluir uma resposta]{História: Excluir uma resposta}
\label{História: Excluir uma resposta}
\resizebox{\linewidth}{!}{
\begin{tabular}{|p{6.5cm}|c|c|c|c|}
\hline
\thead{Descrição} & \thead{Épico} & \thead{Pontuação} & \thead{Tamanho} & \thead{Prioridade}\\
\hline

Como \gls{friend}, eu gostaria de excluir minha resposta para removê-la da visualização caso necessário & Gestão de Respostas & 3 & Pequeno & ALTA \\ \hline

\end{tabular}}\legend{Fonte: Os autores}
\end{quadro}
\FloatBarrier 

%%%%%%% História - Listagem de categorias %%%%%%% 
Para a história de usuário do \autoref{História: Listagem de categorias respostas} foram definidos os seguintes itens como critérios de aceitação:

\begin{itemize}
\item Lista de categorias pré-definidas: ensino, estágio, esportes, entretenimento, institucional e outros;  
\end{itemize}

\def\arraystretch{2}
\begin{quadro}[htb]
\centering
\ABNTEXfontereduzida
\caption[História: Listagem de categorias]{História: Listagem de categorias}
\label{História: Listagem de categorias respostas}
\resizebox{\linewidth}{!}{
\begin{tabular}{|p{6.5cm}|c|c|c|c|}
\hline
\thead{Descrição} & \thead{Épico} & \thead{Pontuação} & \thead{Tamanho} & \thead{Prioridade}\\
\hline

Como \gls{friend}, eu gostaria de visualizar a lista de categorias para facilitar a busca de temas mais relevantes & Gestão de Respostas & 5 & Médio & ALTA \\ \hline

\end{tabular}}\legend{Fonte: Os autores}
\end{quadro}
\FloatBarrier

%%%%%%% História - Ordenação das respostas mais curtidas %%%%%%% 
Para a história de usuário do \autoref{História: Ordenação das respostas mais curtidas} foram definidos os seguintes itens como critérios de aceitação:

\begin{itemize}
\item As respostas mais curtidas devem ser exibidas antes das demais;
\end{itemize}

\def\arraystretch{2}
\begin{quadro}[htb]
\centering
\ABNTEXfontereduzida
\caption[História: Ordenação das respostas mais curtidas]{História: Ordenação das respostas mais curtidas}
\label{História: Ordenação das respostas mais curtidas}
\resizebox{\linewidth}{!}{
\begin{tabular}{|p{6.5cm}|c|c|c|c|}
\hline
\thead{Descrição} & \thead{Épico} & \thead{Pontuação} & \thead{Tamanho} & \thead{Prioridade}\\
\hline

Como \gls{friend}, eu gostaria de ver as respostas mais curtidas antes das demais, para entender quais foram mais relevantes para equela pergunta & Gestão de Respostas & 5 & Médio & ALTA \\ \hline

\end{tabular}}\legend{Fonte: Os autores}
\end{quadro}
\FloatBarrier 

%--------------------------------------------------------------------------------
\section{Épico - Usabilidade}
\label{gestão_usabilidade}
%--------------------------------------------------------------------------------
%%%%%%% História - Ajustar layout conforme novo protótipo%%%%%%% 
Para a história de usuário do \autoref{História: Ajustar layout conforme novo protótipo} foram definidos os seguintes itens como critérios de aceitação:

\begin{itemize}
\item Página de dashboard deve conter o novo banner;
\item Menu lateral excluído;
\item Menu superior reformulado;
\item Componentes fixos criados e ajuste das rotas funcionando como esperado;
\end{itemize}

\def\arraystretch{2}
\begin{quadro}[htb]
\centering
\ABNTEXfontereduzida
\caption[História: Ajustar layout conforme novo protótipo]{História: Ajustar layout conforme novo protótipo}
\label{História: Ajustar layout conforme novo protótipo}
\resizebox{\linewidth}{!}{
\begin{tabular}{|p{6.5cm}|c|c|c|c|}
\hline
\thead{Descrição} & \thead{Épico} & \thead{Pontuação} & \thead{Tamanho} & \thead{Prioridade}\\
\hline

Como mantenedor do IFriends, preciso que o layout da página principal seja atualizado para que possa melhorar a responsividade e usabilidade do sistema. & Usabilidade & 5 & Médio & ALTA \\ \hline

\end{tabular}}\legend{Fonte: Os autores}
\end{quadro}
\FloatBarrier 

%%%%%%% História - Exibir modal dos termos de uso - Frontend %%%%%%% 
Para a história de usuário do \autoref{História: Exibir modal dos termos de uso - Frontend} foram definidos os seguintes itens como critérios de aceitação:

\begin{itemize}
\item  Textos disponíveis em ambos os idiomas a qualquer momento na aplicação;
\end{itemize}

\def\arraystretch{2}
\begin{quadro}[htb]
\centering
\ABNTEXfontereduzida
\caption[História: Exibir modal dos termos de uso - Frontend]{História: Exibir modal dos termos de uso - Frontend}
\label{História: Exibir modal dos termos de uso - Frontend}
\resizebox{\linewidth}{!}{
\begin{tabular}{|p{6.5cm}|c|c|c|c|}
\hline
\thead{Descrição} & \thead{Épico} & \thead{Pontuação} & \thead{Tamanho} & \thead{Prioridade}\\
\hline

Como \gls{friend}, eu gostaria de escolher o idioma da aplicação para visualização em inglês e português. & Usabilidade & 5 & Médio & ALTA \\ \hline

\end{tabular}}\legend{Fonte: Os autores}
\end{quadro}
\FloatBarrier 

%%%%%%% História - Internacionalização%%%%%%% 
Para a história de usuário do \autoref{História: Internacionalização} foram definidos os seguintes itens como critérios de aceitação:

\begin{itemize}
\item  Textos disponíveis em ambos os idiomas a qualquer momento na aplicação;
\end{itemize}

\def\arraystretch{2}
\begin{quadro}[htb]
\centering
\ABNTEXfontereduzida
\caption[História: Internacionalização]{História: Internacionalização}
\label{História: Internacionalização}
\resizebox{\linewidth}{!}{
\begin{tabular}{|p{6.5cm}|c|c|c|c|}
\hline
\thead{Descrição} & \thead{Épico} & \thead{Pontuação} & \thead{Tamanho} & \thead{Prioridade}\\
\hline

Como \gls{friend}, eu gostaria de escolher o idioma da aplicação para visualização em inglês e português. & Usabilidade & 5 & Médio & ALTA \\ \hline

\end{tabular}}\legend{Fonte: Os autores}
\end{quadro}
\FloatBarrier 

%--------------------------------------------------------------------------------
\section{Épico - Gestão de usuários}
\label{gestão_usuario}
%--------------------------------------------------------------------------------
%%%%%%% História - Abas de criações no perfil %%%%%%% 
Para a história de usuário do \autoref{História: Abas de criações no perfil} foram definidos os seguintes itens como critérios de aceitação:

\begin{itemize}
\item Uma lista de eventos publicados;
\item Uma lista de perguntas feitas;
\item Uma lista de respostas dadas;
\end{itemize}

\def\arraystretch{2}
\begin{quadro}[htb]
\centering
\ABNTEXfontereduzida
\caption[História: Abas de criações no perfil]{História: Abas de criações no perfil}
\label{História: Abas de criações no perfil}
\resizebox{\linewidth}{!}{
\begin{tabular}{|p{6.5cm}|c|c|c|c|}
\hline
\thead{Descrição} & \thead{Épico} & \thead{Pontuação} & \thead{Tamanho} & \thead{Prioridade}\\
\hline

Como \gls{friend}, eu gostaria de ter um espaço no meu perfil para as minhas criações qual me permita ver facilmente as minhas contribuições na comunidade. & Gestão de Usuários & 5 & Médio & BAIXA \\ \hline

\end{tabular}}\legend{Fonte: Os autores}
\end{quadro}
\FloatBarrier

%%%%%%% História - Alteração de senha no perfil %%%%%%% 
Para a história de usuário do \autoref{História: Alteração de senha no perfil} foram definidos os seguintes itens como critérios de aceitação:

\begin{itemize}
\item Inserir a senha atual;
\item Informar nova senha;
\item Confirmar nova senha;
\end{itemize}

\def\arraystretch{2}
\begin{quadro}[htb]
\centering
\ABNTEXfontereduzida
\caption[História: Alteração de senha no perfil]{História: Alteração de senha no perfil}
\label{História: Alteração de senha no perfil}
\resizebox{\linewidth}{!}{
\begin{tabular}{|p{6.5cm}|c|c|c|c|}
\hline
\thead{Descrição} & \thead{Épico} & \thead{Pontuação} & \thead{Tamanho} & \thead{Prioridade}\\
\hline

Como \gls{friend}, eu gostaria de alterar a minha senha de acesso quando julgue necessário para me sentir mais seguro. & Gestão de Usuários & 5 & Médio & BAIXA \\ \hline

\end{tabular}}\legend{Fonte: Os autores}
\end{quadro}
\FloatBarrier 

%%%%%%% História - Autenticação do usuário%%%%%%% 
Para a história de usuário do \autoref{História: Autenticação do usuário} foram definidos os seguintes itens como critérios de aceitação:

\begin{itemize}
\item Verificação do \textsl{Token} da \acs{api} no \textsl{login} via \acs{jwt};
\end{itemize}

\def\arraystretch{2}
\begin{quadro}[htb]
\centering
\ABNTEXfontereduzida
\caption[História: Autenticação do usuário]{História: Autenticação do usuário}
\label{História: Autenticação do usuário}
\resizebox{\linewidth}{!}{
\begin{tabular}{|p{6.5cm}|c|c|c|c|}
\hline
\thead{Descrição} & \thead{Épico} & \thead{Pontuação} & \thead{Tamanho} & \thead{Prioridade}\\
\hline

Como \gls{friend}, eu gostaria de me autenticar no sistema para realizar ações personalizadas. & Gestão de Usuários & 5 & Médio & ALTA \\ \hline

\end{tabular}}\legend{Fonte: Os autores}
\end{quadro}
\FloatBarrier 

%%%%%%% História - Edição dos dados no perfil%%%%%%% 
Para a história de usuário do \autoref{História: Edição dos dados no perfil} foram definidos os seguintes itens como critérios de aceitação:

\begin{itemize}
\item Alteração de todos os dados do perfil básico; 
\end{itemize}

\def\arraystretch{2}
\begin{quadro}[htb]
\centering
\ABNTEXfontereduzida
\caption[História: Edição dos dados no perfil]{História: Edição dos dados no perfil}
\label{História: Edição dos dados no perfil}
\resizebox{\linewidth}{!}{
\begin{tabular}{|p{6.5cm}|c|c|c|c|}
\hline
\thead{Descrição} & \thead{Épico} & \thead{Pontuação} & \thead{Tamanho} & \thead{Prioridade}\\
\hline

Como \gls{friend}, eu gostaria de editar meus dados no perfil para que eu possa manter dados atuais na comunidade. & Gestão de Usuários & 5 & Médio & MÉDIA \\ \hline

\end{tabular}}\legend{Fonte: Os autores}
\end{quadro}
\FloatBarrier 

%%%%%%% História - Exclusão do perfil %%%%%%% 
Para a história de usuário do \autoref{História: Exclusão do perfil} foram definidos os seguintes itens como critérios de aceitação:

\begin{itemize}
\item Ao excluir o perfil, o usuário se torna um usuário anônimo, exemplo: user12345;
\end{itemize}

\def\arraystretch{2}
\begin{quadro}[htb]
\centering
\ABNTEXfontereduzida
\caption[História: Exclusão do perfil]{História: Exclusão do perfil}
\label{História: Exclusão do perfil}
\resizebox{\linewidth}{!}{
\begin{tabular}{|p{6.5cm}|c|c|c|c|}
\hline
\thead{Descrição} & \thead{Épico} & \thead{Pontuação} & \thead{Tamanho} & \thead{Prioridade}\\
\hline

Como \gls{friend}, eu gostaria de excluir meu perfil para não ser mais encontrado por outros membros da comunidade & Gestão de Usuários & 8 & Médio & BAIXA \\ \hline

\end{tabular}}\legend{Fonte: Os autores}
\end{quadro}
\FloatBarrier 

%%%%%%% História - Exibir e buscar usuários %%%%%%% 
Para a história de usuário do \autoref{Exibir e buscar usuários} foram definidos os seguintes itens como critérios de aceitação:

\begin{itemize}
\item ;
\end{itemize}

\def\arraystretch{2}
\begin{quadro}[htb]
\centering
\ABNTEXfontereduzida
\caption[História: Exibir e buscar usuários]{História: Exibição e busca de usuários}
\label{Exibir e buscar usuários}
\resizebox{\linewidth}{!}{
\begin{tabular}{|p{6.5cm}|c|c|c|c|}
\hline
\thead{Descrição} & \thead{Épico} & \thead{Pontuação} & \thead{Tamanho} & \thead{Prioridade}\\
\hline

Como \gls{friend}, Como aluno, gostaria de procurar pelos meus colegas que são membros da comunidade. & Gestão de Usuários & 5 & Médio & ALTA \\ \hline

\end{tabular}}\legend{Fonte: Os autores}
\end{quadro}
\FloatBarrier

%%%%%%% História - Exibição do ranking %%%%%%% 
Para a história de usuário do \autoref{História: Exibição do ranking} foram definidos os seguintes itens como critérios de aceitação:

\begin{itemize}
\item Exibição de todos os usuários da comunidade em uma aba, ordenados pelas suas reputações;
\end{itemize}

\def\arraystretch{2}
\begin{quadro}[htb]
\centering
\ABNTEXfontereduzida
\caption[História: Exibição do ranking de usuários]{História: Exibição do ranking de usuários}
\label{História: Exibição do ranking}
\resizebox{\linewidth}{!}{
\begin{tabular}{|p{6.5cm}|c|c|c|c|}
\hline
\thead{Descrição} & \thead{Épico} & \thead{Pontuação} & \thead{Tamanho} & \thead{Prioridade}\\
\hline

Como \gls{friend}, eu gostaria de visualizar outros membros da comunidade para conhecê-los melhor & Gestão de Usuários & 5 & Médio & BAIXA \\ \hline

\end{tabular}}\legend{Fonte: Os autores}
\end{quadro}
\FloatBarrier 

%%%%%%% História - Exibição do troféu e reputação no perfil %%%%%%% 
Para a história de usuário do \autoref{História: Exibição do troféu e reputação} foram definidos os seguintes itens como critérios de aceitação:

\begin{itemize}
\item Tipos de reputação: Bronze (80 pontos), Prata (160 pontos), Ouro (300 pontos); 
\item Meios de reputação: responder (10 pontos), perguntar (10 pontos), criar evento (20 pontos), curtir (2 pontos), descurtir (-2 pontos);
\end{itemize}

\def\arraystretch{2}
\begin{quadro}[htb]
\centering
\ABNTEXfontereduzida
\caption[História: Exibição do troféu e reputação]{História: Exibição do troféu e reputação}
\label{História: Exibição do troféu e reputação}
\resizebox{\linewidth}{!}{
\begin{tabular}{|p{6.5cm}|c|c|c|c|}
\hline
\thead{Descrição} & \thead{Épico} & \thead{Pontuação} & \thead{Tamanho} & \thead{Prioridade}\\
\hline

Como \gls{friend}, eu gostaria de visualizar meu desempenho na plataforma para melhorar a minha experiência  & Gestão de Usuários & 13 & Grande & MÉDIA \\ \hline

\end{tabular}}\legend{Fonte: Os autores}
\end{quadro}
\FloatBarrier 

%%%%%%% História - Exibição dos dados do usuário no perfil %%%%%%%
Para a história de usuário do \autoref{Exibição dos dados do usuário} foram definidos os seguintes itens como critérios de aceitação:

\begin{itemize}
\item O perfil deve exibir: nome, nome de usuário, foto de perfil, biografia, curso e ano;
\end{itemize}

\def\arraystretch{2}
\begin{quadro}[htb]
\centering
\ABNTEXfontereduzida
\caption[História: Exibição dos dados do usuário ]{História: Exibição dos dados do usuário}
\label{Exibição dos dados do usuário}
\resizebox{\linewidth}{!}{
\begin{tabular}{|p{6.5cm}|c|c|c|c|}
\hline
\thead{Descrição} & \thead{Épico} & \thead{Pontuação} & \thead{Tamanho} & \thead{Prioridade}\\
\hline

Como \gls{friend}, eu gostaria de visualizar meu perfil no site para exibir minhas informações & Gestão de Usuários & 5 & Médio & ALTA \\ \hline

\end{tabular}}\legend{Fonte: Os autores}
\end{quadro}
\FloatBarrier 

%%%%%%% História - Histórico de eventos favoritados %%%%%%% 
Para a história de usuário do \autoref{História: Histórico de eventos favoritados} foram definidos os seguintes itens como critérios de aceitação:

\begin{itemize}
\item Mais recentes são exibidos primeiro na lista;
\end{itemize}

\def\arraystretch{2}
\begin{quadro}[htb]
\centering
\ABNTEXfontereduzida
\caption[História: Histórico de eventos favoritados]{História: Histórico de eventos favoritados}
\label{História: Histórico de eventos favoritados}
\resizebox{\linewidth}{!}{
\begin{tabular}{|p{6.5cm}|c|c|c|c|}
\hline
\thead{Descrição} & \thead{Épico} & \thead{Pontuação} & \thead{Tamanho} & \thead{Prioridade}\\
\hline

Como \gls{friend}, eu gostaria de ter um espaço para visualizar os eventos que curti & Gestão de Usuários & 5 & Médio & BAIXA \\ \hline

\end{tabular}}\legend{Fonte: Os autores}
\end{quadro}
\FloatBarrier 

%%%%%%% História - Recuperação de senha %%%%%%% 
Para a história de usuário do \autoref{História: Recuperação de senha} foram definidos os seguintes itens como critérios de aceitação:

\begin{itemize}
\item Enviar link com código de autenticação para verificar o e-mail;
\item Identificar o usuário pelo código enviado e permitir a troca de senha;
\end{itemize}

\def\arraystretch{2}
\begin{quadro}[htb]
\centering
\ABNTEXfontereduzida
\caption[História: Recuperação de senha]{História: Recuperação de senha}
\label{História: Recuperação de senha}
\resizebox{\linewidth}{!}{
\begin{tabular}{|p{6.5cm}|c|c|c|c|}
\hline
\thead{Descrição} & \thead{Épico} & \thead{Pontuação} & \thead{Tamanho} & \thead{Prioridade}\\
\hline

Como \gls{friend}, eu gostaria de recuperar minha senha no site para conseguir me autenticar caso a esqueça & Gestão de Usuários & 5 & Médio & MÉDIA \\ \hline

\end{tabular}}\legend{Fonte: Os autores}
\end{quadro}
\FloatBarrier 

%%%%%%% História - Verificação de email %%%%%%% 
Para a história de usuário do \autoref{História: Verificação de e-mail} foram definidos os seguintes itens como critérios de aceitação:

\begin{itemize}
\item Enviar o link com código de autenticação para verificar o e-mail;
\item Identificar o \gls{friend} pelo código enviado e permitir a autenticação;
\end{itemize}

\def\arraystretch{2}
\begin{quadro}[htb]
\centering
\ABNTEXfontereduzida
\caption[História: Verificação de e-mail]{História: Verificação de e-mail}
\label{História: Verificação de e-mail}
\resizebox{\linewidth}{!}{
\begin{tabular}{|p{6.5cm}|c|c|c|c|}
\hline
\thead{Descrição} & \thead{Épico} & \thead{Pontuação} & \thead{Tamanho} & \thead{Prioridade}\\
\hline

Como \gls{friend}, eu preciso verificar meu e-mail para realizar meu cadastro no sistema & Gestão de Usuários & 8 & Médio & ALTA \\ \hline

\end{tabular}}\legend{Fonte: Os autores}
\end{quadro}
\FloatBarrier 

% ------------------------------------------------------------------------------------
\chapter{Outras histórias de usuário}
\label{outras historias de usuario}
% ------------------------------------------------------------------------------------
Já a seguinte seção apresentará as demais histórias de usuário discutidas pela equipe que não entraram nas \glspl{Sprint} do projeto \gls{ifriends} classificadas por seus respectivos épicos.

%--------------------------------------------------------------------------------
\section{Épico - Gestão de perguntas}
\label{outro_gestão_perguntas}
%--------------------------------------------------------------------------------
%%%%%%% História - Filtragem de tags %%%%%%% 
Para a história de usuário do \autoref{História: Filtragem de tags} foram definidos os seguintes itens como critérios de aceitação:

\begin{itemize}
\item Deve haver um filtro por tag;
\item A tag do filtro pode ser pesquisada em um campo de \textit{input}; 
\end{itemize}

\def\arraystretch{2}
\begin{quadro}[htb]
\centering
\ABNTEXfontereduzida
\caption[História: Filtragem de tags]{História: Filtragem de tags}
\label{História: Filtragem de tags}
\resizebox{\linewidth}{!}{
\begin{tabular}{|p{6.5cm}|c|c|c|c|}
\hline
\thead{Descrição} & \thead{Épico} & \thead{Pontuação} & \thead{Tamanho} & \thead{Prioridade}\\
\hline

Como \gls{friend}, eu quero utilizar uma tag como campo de filtragem para facilitar a busca por perguntas relacionadas & Gestão de Perguntas & 8 & Médio & MÉDIA \\ \hline

\end{tabular}}\legend{Fonte: Os autores}
\end{quadro}
\FloatBarrier 
