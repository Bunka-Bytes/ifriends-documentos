\chapter{Resultados da entrevista de usabilidade}
\label{resultado_entrevista}
%----------------------------------------------------------------
Feita as entrevistas, com alguns alunos do \glsxtrshort{ifsp}, a seguinte seção visa documentar os resultados que foram obtidos:

\section{1\textordfeminine \, Entrevista}

\textbf{Nome:} Ana Luisa Oliveira da Silva

\textbf{Curso:} Informática - Terceiro ano

Por meio das perguntas iniciais, Ana Luisa confirmou ser a primeira vez em que participa de uma entrevista do tipo, mas por fazer parte do curso técnico de informática ela já se deparou com cenários onde houve necessidade de realizar testes em atividades proporcionadas pelos professores do técnico.

Ela gosta de acessar a Internet, principalmente redes sociais, confirmou que conhece e às vezes participa de comunidades como o Stack Overflow e de eventos fornecidos pelo instituto.

Quando sente dificuldades e precisa de informações a respeito do ensino, ela costuma procurar por monitórias, tirar dúvidas com amigos e por plantões fornecidos pelos professores.

As informações a respeito da instituição ela procurar no \textit{site} do \acs{ifsp}, pergunta na secretaria ou para pessoas da área. Mas relata ter dificuldades devido à demora da divulgação dos comunicados no \textit{site} da instituição.

Com a apresentação da solução, dos desafios e por meio do \textit{feedback} geral, chegaram-se as seguintes considerações:

\subsection{Dificuldades e apontamentos}
\begin{itemize}
    \item Sentiu falta do botão cadastrar na página inicial;
    \item No \textit{login}, o campo de senha validava mesmo informando uma senha errada; 
    \item Sentiu um pouco de dificuldade em entender alguns ícones, mas comentou que a explicação ajudou;
    \item Relatou não saber identificar quando um item no cadastro é obrigatório, por exemplo: ``Não sabia se o \textit{link} do evento era obrigatório'', diz; 
\end{itemize}

\subsection{Pontos positivos}
\begin{itemize}
    \item Apenas por meio da visualização conseguiu identificar o propósito do sistema e informou ter uma noção de como utilizá-lo;
    \item Categorizou o sistema como intuitivo;
    \item Usou o tutorial de como fazer uma boa pergunta;
    \item Elogiou a padronização e \textit{design} do sistema assim como o seu propósito;
\end{itemize}

\section{2\textordfeminine \, Entrevista}

\textbf{Nome:} Paulo Ricardo Ribeiro

\textbf{Curso:} Letras - Quarto semestre

Com base nas perguntas iniciais, Paulo confirmou também que era a primeira vez em que participava de um teste de usabilidade: ``Já participei de entrevistas para provar produtos, mas é a primeira vez testando um site'', diz. 

Ele gosta de praticar esportes e do contato com a natureza. Sua relação com a Internet, em sua maioria, também se dá por meio das redes sociais, ele comenta que gosta de dar aquela conferida no Instagram de vez em quando. Ele costuma participar de comunidades on-line e dos eventos divulgados pelo \acs{ifsp}. 

Com a apresentação da solução, dos desafios e por meio do \textit{feedback} geral, chegaram-se as seguintes considerações:

\subsection{Dificuldades e apontamentos}
\begin{itemize}
    \item Sentiu dificuldades em localizar onde criar uma pergunta/evento, comentou que o botão que realiza a ação é muito sutil; 
    \item O ícone usado para identificar a categoria no cadastro de perguntas/eventos acaba confundindo;  
\end{itemize}

\subsection{Pontos positivos}
\begin{itemize}
    \item Por meio da visualização conseguiu identificar o propósito do sistema e teve como primeira impressão que se tratava de uma comunidade de compartilhamento de informações;
    \item Gostou da proposta do sistema e concordou que seria de grande ajuda;
\end{itemize}

\subsection{Sugestões de melhoria}
\begin{itemize}
    \item Ele considerou que seria importante filtrar mensagens aleatórias (ele reparou em algumas mensagens usadas nos testes desenvolvidos pela equipe);
    \item Outra sugestão fornecida foi a respeito da descrição ser obrigatória, ele julga que em alguns casos não há necessidade; 
    \item Por último, ele considera que seria bom deixar alguns ícones mais nítidos ou acrescentar algum texto, por exemplo, no botão de criar; 
\end{itemize}