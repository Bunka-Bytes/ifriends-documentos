\chapter{Termos de uso}
\label{termos_uso}
%% texto de introdução
% ----------------------------------------------------------
 A seguinte seção contém as informações que estão presentes nos termos de uso do sistema \gls{ifriends}, desenvolvido pela equipe com base nas regras de negócio definidos no início do projeto.
\section{Termos e condições de uso do sistema}
 
Seja bem-vindo ao Ifriends. Leia com muita atenção todos os termos abaixo.
 
Este documento, e todo o conteúdo do sistema é oferecido por Bunka Bytes, neste termo representado apenas por ``EMPRESA'', que regulamenta todos os direitos e obrigações com todos que acessam o sistema, denominado neste termo como ``VISITANTE'', resguardado todos os direitos previstos na legislação, trazem as cláusulas abaixo como requisito para acesso e visita do sistema, situado no endereço \href{https://ifriends.vercel.app/}{https://ifriends.vercel.app/}.

A permanência no sistema implica-se automaticamente na leitura e aceitação tácita do presente termos de uso a seguir. Este termo foi atualizado pela última vez em 21 de Agosto de 2022.
 
\subsection{DA FUNÇÃO DO SITE}
 
Este sistema foi criado e desenvolvido com o objetivo de oferecer uma plataforma que quebre barreiras de comunicação e acolha os alunos no seu processo de exploração acadêmica. A ``EMPRESA'' busca através da plataforma, desenvolvida por alunos do curso técnico de informática, trazer o conhecimento ao alcance de todo corpo estudantil da instituição, assim como a adaptação dos alunos ao meio acadêmico.

Nesta plataforma, poderá ser realizadas tanto a postagem de perguntas e respostas, assim como a divulgação de eventos.

A maioria do conteúdo presente neste sistema é destinada para os próprios VISITANTES alimentarem e consumirem, assim como perguntas, respostas, eventos e perfis.

Todo o conteúdo é revisado periodicamente, porém, pode conter alguma informação, texto ou imagem que não reflita a verdade atual, não podendo a EMPRESA ser responsabilizada de nenhuma forma ou meio por qualquer conteúdo que não esteja devidamente atualizado.

É de responsabilidade do VISITANTE usar todas as informações presentes na plataforma com senso crítico, utilizando como fonte e compartilhamento de informação, e sempre buscando conteúdos confiáveis para a solução concreta do seu(s) conflito(s).

O VISITANTE possui a capacidade de denunciar quaisquer tipos de informação ou conteúdo presente na plataforma que infligem as regras deste termo de uso ou seus princípios.

\subsection{DO ACEITE DOS TERMOS}
 
Este documento, chamado ``Termos de Uso'', aplicáveis a todos os visitantes do sistema

Este termo especifica e exige que todo usuário ao acessar o sistema da EMPRESA, leia e compreenda todas as cláusulas dele, visto que ele estabelece entre a EMPRESA e o VISITANTE direitos e obrigações entre ambas as partes, aceitos expressamente pelo VISITANTE a permanecer navegando no site da EMPRESA.

Ao continuar acessando o sistema, o VISITANTE expressa que aceita e entende todas as cláusulas, assim como concorda integralmente com cada uma delas, sendo este aceite imprescindível para a permanência na mesma. Caso o VISITANTE discorde de alguma cláusula ou termo deste contrato, deve imediatamente interromper sua navegação de todas as formas e meios.

Este termo pode e irá ser atualizado periodicamente pela EMPRESA, que se resguarda no direito de alteração, sem qualquer tipo de aviso prévio e comunicação. É importante que o VISITANTE confira sempre se houve movimentação e qual foi a última atualização do termo no início da página.

\subsection{DO GLOSSÁRIO}
 
Este termo pode conter algumas palavras específicas que podem não se de conhecimento geral. Entre elas:

\begin{itemize}
\item VISITANTE: Todo e qualquer usuário do site, de qualquer forma e meio, que acesse através de computador, notebook, tablet, celular ou quaisquer outros meios, o website ou plataforma da empresa.
\item NAVEGAÇÃO: O ato de visitar páginas e conteúdo do website ou plataforma da empresa.
\item LOGIN: Dados de acesso do visitante ao realizar o cadastro junto a EMPRESA, dividido entre usuário e senha, que dá acesso a funções restritas do site.
\item HIPERLINKS: São links clicáveis que podem aparecer pelo site ou no conteúdo, que levam para outra página da EMPRESA ou site externo.
\item OFFLINE: Quando o site ou plataforma se encontra indisponível, não podendo ser acessado externamente por nenhum usuário.
\end{itemize}
 
Em caso de dúvidas sobre qualquer palavra utilizada neste termo, o VISITANTE deverá entrar em contato com a EMPRESA através dos canais de comunicação encontradas no site.
 
 \subsection{DO ACESSO AO SITE}
 
O sistema funciona normalmente 24 (vinte e quatro) horas por dia, porém podem ocorrer pequenas interrupções de forma temporária para ajustes, manutenção, mudança de servidores, falhas técnicas ou por ordem de força maior, que podem deixar a plataforma indisponível por tempo limitado.

Em caso de manutenção que exigirem um tempo maior, a EMPRESA irá informar previamente aos clientes da necessidade e do tempo previsto em que o site ou plataforma ficará offline.

O acesso ao site só é permitido a VISITANTES que tiverem e-mail institucional e que possuírem algum tipo de relação ao meio acadêmico do IFSP. Na ausência desses critérios, é necessário o acesso ao site SUAP com prontuário e senha para solicitação do e-mail institucional.

Caso seja necessário realizar um cadastro junto a plataforma, onde o VISITANTE deverá preencher um formulário com seus dados e informações, para ter acesso a alguma parte restrita, ou realizar alguma ação dentro da plataforma.

Todos os dados estão protegidos conforme a Lei Geral de Proteção de Dados, e ao realizar o cadastro junto ao site, o VISITANTE concorda integralmente com a coleta de dados conforme a Lei e com a Política de Privacidade da EMPRESA.
 
\subsection{DA LICENÇA DE USO E CÓPIA}
 
O visitante poderá acessar todo o conteúdo da plataforma, como perguntas, respostas, imagens e eventos, não significando nenhum tipo de cessão de direito ou permissão de uso, ou de cópia.
 
Todos os direitos são preservados, conforme a legislação brasileira, principalmente na Lei de Direitos Autorais (regulamentada na \href{http://www.planalto.gov.br/ccivil_03/leis/l9610.htm}{Lei nº 9.610/18}, assim como no Código Civil brasileiro (regulamentada na \href{http://www.planalto.gov.br/ccivil_03/leis/l9610.htm}{Lei nº 10.406/02}), ou quaisquer outras legislações aplicáveis.
 
\subsection{DAS OBRIGAÇÕES}
 
O VISITANTE ao utilizar o website da EMPRESA, concorda integralmente em:

\begin{itemize}
\item De nenhuma forma ou meio realizar qualquer tipo de ação que tente invadir, hacker, destruir ou prejudicar a estrutura da plataforma da EMPRESA. Incluindo-se, mas não se limitando, ao envio de vírus de computador, de ataques de DDOS, de acesso indevido por falhas da mesma ou quaisquer outras forma e meio.
\item Não realizar divulgação indevida nos comentários da plataforma de conteúdo de SPAM, empresas concorrentes, vírus, conteúdo que não possua direitos autorais ou quaisquer outros que não seja pertinente a discussão daquele texto ou imagem.
\item Da proibição em reproduzir qualquer conteúdo do sistema ou plataforma sem autorização expressa, podendo responder civil e criminalmente por isso.
\item De nenhuma forma ou meio realizar citação de conteúdo malicioso, discurso de ódio ou qualquer tipo de atitude que exclui, separa e inferioriza pessoas tendo como base ideias preconceituosas, podendo causar o desligamento do VISITANTE do campus ou responder civil e criminalmente pelas ações.
\item Respeitar a Política de Privacidade do sistema, assim como tratamos os dados referentes ao cadastro e visita no sistema, podendo a qualquer momento e forma, requerer a exclusão do cadastro, através das ferramentas dentro da plataforma ou, em caso de dificuldades externas, do e-mail da EMPRESA.
\item Ser impossibilitado de excluir definitivamente os conteúdos criados pelo próprio VISITANTE, apenas havendo a opção de alterar a visibilidade deles para criações anônimas, a qualquer momento.
\end{itemize}

\subsection{DOS TERMOS GERAIS}
Apesar da EMPRESA apenas criar um espaço para o compartilhamento de sites externos através de hiperlinks, caso o usuário acesse um site externo, a EMPRESA não tem nenhuma responsabilidade pelo meio, sendo uma mera indicação de complementação de conteúdo, ficando o ele responsável pelo acesso, assim como sobre quaisquer ações que venham a realizar neste site.

Este Termo de uso é valido a partir de 21 de agosto de 2022.