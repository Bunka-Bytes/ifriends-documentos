% ---
% RESUMOS
% ---

% resumo em português
\setlength{\absparsep}{18pt} % ajusta o espaçamento dos parágrafos do resumo
\begin{resumo}

O presente documento é o resultado do desenvolvimento obtido a partir da elaboração de um projeto que visa a criação de uma comunidade virtual do \acs{ifsp} através do aprendizado adquirido na disciplina técnica de \acs{pds} no quarto ano do Curso Técnico de Informática, realizado no \ac{ifsp}, Campus São Paulo.
O objetivo central deste trabalho, desse modo, é apresentar a projeção e a implementação de um produto mínimo viável em formato de sistema Web para a construção de uma comunidade virtual, em busca da criação de um espaço de acolhimento de alunos para alunos. Propõe-se, assim, utilizar de um método ágil e de ferramentas de desenvolvimento para passar pelos processos de engenharia do sistema, além de estimular o trabalho em equipe. Sob essa perspectiva, o projeto pôde ser apresentado abordando seu tema principal e focando nas suas funcionalidades mais essenciais, como as perguntas e respostas, o perfil do usuário e a gestão de eventos.


\textbf{Palavras-chaves}: Projeto. IFriends. Comunidade virtual.
\end{resumo}

% resumo em inglês
\begin{resumo}[Abstract]
\begin{otherlanguage*}{english}
This document is a result of the work created from the development of a project that looks forward to creating a virtual community for the \acs{ifsp} members with the knowledge practiced during the \acs{pds} technical discipline in the fourth year of studying at the Federal Institute of Education, Science and Technology of Sao Paulo. The main goal of this project, therefore, is to present the design and implementation of a minimum valuable product in terms of a web software, for the construction of the virtual community and in order to create a supportive environment from students to students. Hence, the team intends to use an agile methodology and development tools to go through the software engineering process and encourage the team work. Under this perspective, the project could be presented focusing on its main goal and functionalities related to the minimum user needs, such as the questions and answers, the user profile and the event management.

  \vspace{\onelineskip}
  \noindent 
  \textbf{Keywords}: Project. IFriends. Virtual Community.
 \end{otherlanguage*}
\end{resumo}