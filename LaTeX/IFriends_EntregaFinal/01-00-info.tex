\newcommand{\urlmodelosimples}{https://www.overleaf.com/project/58a3a66af9bb74023ba1bd56}

\newcommand{\urlmodelo}{\url{\urlmodelosimples}}

Esse documento foi feito a partir do modelo canônico do \abnTeX, o acesso ao PDF pode ser feito em 
\urlmodelo. A estrutura utilizada aqui foi um modelo utilizado no curso de Pós Graduação em Gestão de TI do \ac{ifsp}.
\todo[inline]{Remover texto informativo inicial}


Este documento não pode ser considerado como um padrão a ser seguido em sua totalidade, ele tem como maior objetivo demonstrar como utilizar o \LaTeX\ para obter um documento atendendo ao máximo o padrão do \ac{ifsp} e \ac{abnt}.

Faça leitura dos arquivos fonte \LaTeX\ e não somente do PDF gerado.

Esse documento é, em princípio, uma cópia do utilizado na disciplina de PDS e irá sofrer alterações durante o ano letivo. Essas alterações buscam induzir os alunos à prática com o \LaTeX\ . 

Algumas bibliotecas \LaTeX\ disponíveis no overleaf estão desatualizadas, para melhores resultados é recomendável a utilização de outro compilador utilizando as ultimas versões de todas bibliotecas.

Leia com cuidado :
\begin{itemize}
    \item \url{https://dicas.ivanfm.com/aulas/textos/};
    \item \url{https://dicas.ivanfm.com/aulas/textos/revisao-de-textos.html};
    \item exemplos no \autoref{cap-exemplos};
    \item \autoref{elementos-nao-textuais} sobre elementos não textuais que fala sobre o maior problema dos alunos que é de tentar posicionar as ilustrações.
\end{itemize}


\noindent\hrulefill
