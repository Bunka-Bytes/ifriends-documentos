
% ----------------------------------------------------------
% Introdução (exemplo de capítulo sem numeração, mas presente no Sumário)
% ----------------------------------------------------------
\chapter{Introdução}
O presente documento é resultado da proposta de um projeto cujo objetivo é sustentado no planejamento e na execução de um sistema para a Web através do aprendizado obtido nas matérias técnicas do Curso Técnico de Informática, realizado no \acs{ifsp}, e como forma de trabalho de conclusão de curso.

Tendo isto em vista e diante dos desafios presentes na lista de requisitos e orientações propostas para a iniciação deste projeto, os integrantes da equipe Bunka Bytes reuniram-se em busca de encontrar a solução que melhor se encaixasse em seus objetivos e nos da disciplina de \acs{pds}.

Portanto, este documento apresenta o processo de desenvolvimento da comunidade virtual \gls{ifriends}, desde a origem do projeto de sistema, até sua apresentação final, documentando o acompanhamento de todas as atividades executadas pelos membros da equipe para sua elaboração. 

%-------------------------------------------------------------------------------
\section{Problemática e origem do projeto}
% ------------------------------------------------------------------------------
%Foi pensando em soluções viáveis que a equipe voltou seu olhar para sistemas que contribuem para a criação de comunidades colaborativas na área de desenvolvimento de sistemas, que, segundo \citeonline{rosa2008identidade}, foi fortemente difundida pelas comunidades de Código Aberto (do inglês, \textsl{Open Source}), sendo primeiramente criada pela cultura \textsl{hacker}, na qual afirma que a paixão e o interesse dos \textsl{hackers} nas soluções foi uma das principais propulsoras do espírito colaborativo.

Inicialmente, foram trazidas ao debate as possibilidades de aliar as principais dificuldades que os integrantes observaram durante seu período de estudo no \ac{ifsp}, a um sistema que pudesse suprir determinadas necessidades dos alunos, como os questionamentos que começam a surgir com mais frequência conforme o início dos estudos é dado, sendo eles em âmbitos diversos tais como: sobre a instituição de ensino, matérias e assuntos tratados no ensino médio, dúvidas sobre os conteúdos técnicos ou até mesmo a busca por um apoio educacional - como ocorrem nas monitoriais. 


O \ac{ifsp}, especificamente no campus São Paulo, é o cenário no qual os integrantes da equipe e o público-alvo que a comunidade pretende atingir estão inseridos: uma instituição federada atuante nos ensinos médio, técnico profissionalizante e superior, conforme explica \citeonline{ifsmg}: 

\begin{citacao}
Os Institutos Federais são instituições que atuam na oferta da educação profissional e tecnológica, em todos os seus níveis e modalidades, formando e qualificando cidadãos com vistas na atuação nos diversos setores da economia, com ênfase no desenvolvimento socioeconômico local, regional e nacional \cite{ifsmg}.
\end{citacao}

Por ser um ambiente novo e repleto de informações simultâneas, existe a possibilidade de muitos alunos ingressantes ficarem perdidos no momento de lidar com assuntos cotidianos na vida de alunos veteranos, como: funcionamento da instituição, quantidade de disciplinas, volume de atividades, novas pessoas para se conviver, liberdade e responsabilidade concedidas ao aluno, entre outras. Isso torna o \acs{ifsp} um ambiente muito desafiador que pode gerar inúmeras dúvidas entre os novos alunos. 
O \gls{ifriends} surge nesse cenário, no qual a criação de uma comunidade de estudantes que colaborassem entre si, pudesse instigar o interesse dos alunos em ajudarem uns aos outros de maneira acessível e prática, onde uma dúvida estivesse a um palmo de distância.

%-------------------------------------------------------------------------------

\section{Objetivo}
% ------------------------------------------------------------------------------
%Nesta seção serão descritos os respectivos objetivos (principal e específico) pretendidos com a iniciação deste projeto. 
%\subsection{Objetivo Principal}

Dadas as informações citadas anteriormente, o objetivo deste projeto é tentar instigar o interesse dos estudantes que compreendem o \acs{ifsp} para poderem criar espaços colaborativos, por um sistema no qual os usuários interajam entre perguntas e respostas, fornecendo caminhos para o esclarecimento de suas dúvidas sobre a instituição de ensino, as áreas e disciplinas que a ela pertence.  

Dessa forma, o objetivo será aplicado através da construção de uma plataforma de perguntas, respostas e mentorias para a Web, em que a comunidade interna poderá submeter uma pergunta para ser respondida pelos outros membros da comunidade; além de possibilitar que estudantes possam escolher se tornar mentores sobre determinados assuntos, disponibilizando recursos para a criação de anúncios de eventos, principalmente de monitorias (cuja localidade a eles deve competir), dentro de seus perfis de usuário. 

Tendo isso em vista e pensando numa melhor interatividade entre os membros da comunidade, o sistema deve passar por um processo para uso de recursos existentes na \gls{gamificação} em algumas de suas funcionalidades, como as curtidas para respostas e perguntas mais relevantes e os atributos de recompensar aos usuários mais ativos  - assim como outros exemplos que devem ser adicionados durante o planejamento do projeto.

\section{Justificativa}
A reflexão com relação às formas complementares de aprendizagem é importante para a ampliação dos conteúdos interessados tanto aos alunos, quanto aos seus professores, pois permite que enxerguem, juntos, o ensino como um meio que evite a passagem de aprendizados de forma restrita e hierarquizada.

Por isso, \citeonline{fernandes2011redes} traz em sua pesquisa que o desafio da construção de sociedades de aprendizagem parte do pressuposto de que os recursos tecnológicos disponibilizados atualmente permitem aos estudantes aprenderem dentro e fora da escola e das mais variadas formas. Assim, para ele, a melhor forma se dá ``construindo comunidades sustentadas pelo uso de tecnologias Web''.

O autor dá continuidade na exposição desse fenômeno ao atribuir o sucesso da potencialização da aprendizagem complementar e das relações sociais à ``Web 2.0''. Isto, pois, de acordo \citeonline{fernandes2011redes}, permitiu novas formas e possibilidades de criação de conteúdos e possibilitou o enfoque a uma aprendizagem motivada pelos interesses do aluno, em que ele deve assumir um papel exploratório nessa experiência, da qual poderá colher ensinamentos significativos, explica \citeonline{fernandes2011redes}.

Visando atrair atenção para o tema, o projeto tem como principal missão, permitir que os estudantes possam usufruir de uma ferramenta gratuita que proporcione a suavização do seu processo de aprendizagem, quando seus próprios colegas contribuirão com suas experiências passadas, além de deixarem um histórico para possibilitar um caminho menos árduo aos estudantes que virão. Por isso, espera-se que, com este projeto, a instituição de ensino também seja um agente na construção de uma comunidade propícia para estudantes, onde poderão unir-se em razão de dúvidas comuns, e assim incentivarem a disseminação de uma cultura colaborativa dentro de seus espaços.

\chapter{Definição da proposta}
Neste capítulo, com base nas especificações detalhadas durante a introdução, a equipe demonstra alguns dos artifícios utilizados para a construção da proposta do \gls{ifriends} e como eles influenciaram na toma de decisões para o desenvolvimento.

\section{Pesquisa de aceitação} 
\label{pesquisa}
% ------------------------------------------------------------------------------
Ainda como parte da construção de uma justificativa para o projeto, foi realizada uma pesquisa de aceitação para o sistema, visando verificar se o público-alvo realmente poderia ser atingido conforme os objetivos pretendidos para o projeto, ou seja, se os alunos da instituição possuíssem interesse na aplicação e pretendessem utilizá-la como ferramenta cotidiana para auxiliá-los durante os estudos.

Para isto, houve a elaboração de um formulário por meio da ferramenta \gls{googleforms}, no qual solicitamos que os alunos da instituição respondessem dez questões (\autoref{questões}) sobre a proposta. Sua divulgação ocorreu por meio do aplicativo de conversas \gls{WhatsApp}, meio pelo qual os integrantes da equipe ficaram responsáveis por enviar o endereço de compartilhamento do formulário nos grupos de alunos conhecidos na instituição.

A realização desta pesquisa foi de grande importância para avaliação da proposta apresentada e para analisar sua viabilidade por meio dos resultados.

%-------------------------------------------------------------------------------
\subsection{Resultados}
%-------------------------------------------------------------------------------
A partir deste formulário, foram recebidas \textbf{quarenta e cinco} respostas acumuladas dentro de um período de \textbf{cinco} dias, e ao final, foi possível constatar as seguintes características como destaques na maioria das respostas:


\begin{itemize}
    \item 86,7\% dos estudantes são do ensino médio integrado ao técnico;
    \item 66,7\% dos que sentiram dificuldade ao ingressar no \glsxtrshort{ifsp}, compartilhou em forma de resposta curta suas experiências (como, adaptação com as atividades, dificuldades em matérias específicas, falta de transparência nas informações institucionais);
    \item 60\% nunca frequentaram ou vão raramente às monitorias;
    \item 85,5\% usariam o sistema e acreditam que este o ajudaria academicamente;
\end{itemize}

Ainda no \autoref{graficos_pesqAceitacao} é possível visualizar os gráficos referentes a todas as respostas obtidas com mais detalhes. Dessa forma e com base nesses resultados, a equipe entendeu que a proposta tem potencial para cumprir seu propósito de atingir os alunos do \glsxtrshort{ifsp} com o desenvolvimento de uma comunidade de apoio aos assuntos enfrentados durante sua formação.

% ------------------------------------------------------------------------------
\section{Análise de concorrência}
Paralelamente, para a construção da proposta, levou-se em consideração uma análise de concorrência com plataformas que apresentem características semelhantes ou até mesmo parecidas com as que gostaríamos de oferecer. Dessa forma, pretendemos demostrar melhor como o nosso projeto se destaca em relação à concorrência e no que ele pode agregar.

Como primeiro concorrente pensamos no \gls{moodle} já que ele faz parte da vida de muitos alunos do \acs{ifsp}, pois ele é usado como ferramenta complementar de estudos. Ele fornece, nas disciplinas criadas pelos professores, a possibilidade de criar um fórum e dispõe uma versão \textsl{mobile}.

Como segundo concorrente pensou-se no \gls{scoold} devido a ser uma inspiração para o nosso projeto, já que a plataforma possui quase todas as funcionalidades pensadas para o \gls{ifriends}, mas seu foco é fornecer um espaço de perguntas e respostas. 

Dessa forma, o \autoref{AnalisedeConcorrencia} apresenta a comparação dessas duas plataformas com o projeto de sistemas \gls{ifriends}:

\def\arraystretch{2}
\begin{quadro}[htb]
\centering
\ABNTEXfontereduzida
\caption{Análise de Concorrência}
\label{AnalisedeConcorrencia}
\resizebox{\linewidth}{!}{
\begin{tabular}{|p{8cm}|c|c|c|}
\hline
\thead{Funcionalidades} & \thead{IFriends} & \thead{Moodle} & \thead{Scoold} \\
\hline

Comunicação pública (fórum) entre as partes envolvidas & \circlemark &  & \circlemark \\ \hline
Publicação de eventos & \circlemark &  &  \\ \hline
Publicação de perguntas & \circlemark & \circlemark & \circlemark \\ \hline
Filtragem para pesquisa no fórum de perguntas & \circlemark &  & \circlemark \\ \hline
Filtragem para pesquisa de eventos & \circlemark &  &  \\ \hline
Gamificação em determinados recursos & \circlemark &  & \circlemark \\ \hline
Acesso à plataforma por aplicativo móvel &  & \circlemark &  \\ \hline

\end{tabular}}\legend{Fonte: Os autores}
\end{quadro}
\FloatBarrier

Devido ao curto tempo de desenvolvimento, uma questão que ficou em aberto, em relação às funcionalidades propostas inicialmente no \autoref{AnalisedeConcorrencia}, foi essa questão da versão \textit{mobile}. Entretanto, isso não foi considerado uma questão prejudicial para o desenvolvimento do projeto, visto que, o que levou a tal escolha foi a priorização em desenvolver melhor as outras funcionalidades de acordo com o tempo e a experiência que os integrantes tivessem disponíveis, já que a missão do projeto visa dar o primeiro passo para a  criação de uma comunidade que facilite e quebre essa barreira de interação entre as partes envolvidas.

Além disso, considerando a organização e a estruturação do projeto em termos de desenvolvimento, não haverá impedimentos para que, num projeto futuro, seja avaliada a necessidade de inclusão da versão \textsl{mobile}, tendo em vista que seria possível reutilizar grande parte dos recursos já existentes, ver o detalhamento da arquitetura do sistema, na \autoref{Arquitetura_Rest_API}.
