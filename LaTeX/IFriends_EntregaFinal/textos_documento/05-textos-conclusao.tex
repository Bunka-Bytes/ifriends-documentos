% ---
% Conclusão (outro exemplo de capítulo sem numeração e presente no sumário)
% Dependendo do trabalho desenvolvido ele pode ter uma Conclusão ou Considerações finais
% Para trabalhos de disciplina utilizar Considerações Finais

% ---------------------------------------------------------
\chapter{Considerações finais}
% ---------------------------------------------------------
Tendo em vista a jornada passada tanto através deste documento como pelo período em que este projeto foi desenvolvido, a equipe utiliza do espaço deste capítulo para
refletir sobre os aspectos mais relevantes notados pelos integrantes por todo o período de
construção do \gls{ifriends}.

Em primeiro lugar, é importante salientar que o desenvolvimento do projeto se deu em um dos anos mais desafiadores para os membros da equipe, considerando que precisamos adaptar nossa rotina de trabalho e estudos para nos adequarmos aos requisitos e regras da disciplina e do projeto, num momento em que diversas preocupações do final do ensino médio e o início da vida adulta surgem de modo mais evidente. Por isso, é necessário, incialmente, reconhecer o esforço e comprometimento de todos os integrantes para que a entrega do projeto fosse realizada, tendo em vista a parceria e companheirismo que tivemos para trabalharmos em conjunto em detrimento de soluções para o projeto.

Certamente não se pode negar que tal esforço isentou a equipe de falhas, entretanto, cabe lembrar que o processo de melhoria contínua é um dos pilares do trabalho com agilidade, e que é mais importante atentar-se a prática da essência dos princípios e valores por ela explicitados, do que ao uso de ferramentais específicos para monitorar o processo - não que estes não tenham seu valor. Portanto, a equipe acredita que o \textsl{feedback} das partes envolvidas a respeito das falhas e oportunidades de melhoria no projeto, tanto tecnicamente como processualmente, é de suma relevância para que a continuidade e o crescimento da equipe e do sistema sejam propícios.

A respeito das tecnologias e ferramentas utilizadas, a equipe entende que as escolhas tenham sido satisfatórias para o desenvolvimento do projeto, porém foi percebido que, devido à inexperiência em determinados assuntos, acabou-se descobrindo o funcionamento de alguns recursos muito depois de já ter-se iniciado a implementação do projeto, principalmente na parte de \gls{front-end}. Isso pode ser um dos motivadores, por exemplo, para o atraso da inclusão deles em determinadas partes críticas, já que foi necessário certo retrabalho para encaixá-los (dentre os que conseguiu-se incluir) dentro da estrutura do sistema.

Além disso, sentiu-se a necessidade de entender mais sobre usabilidade e interações do usuário com o sistema, para que fossem melhores planejados os recursos a serem utilizados no \gls{front-end}, todavia, houve certa dificuldade para aplicá-los de forma satisfatória num primeiro momento, e por isso, acabamos também perdendo tempo posteriormente para ajustá-los no sistema.

Nesse sentido, é possível afirmar que, de fato, ao seguir os princípios e valores da agilidade, é importante que não fiquemos presos a um escopo bem definido durante toda a execução do projeto; porém, também não devemos deixar de planejar bem as tecnologias e recursos a serem utilizados e de focarmos em testar apenas o mais crítico num primeiro momento, para que o processo de melhoria contínua não seja prejudicado com grandes impedimentos dessa natureza. 

De todo modo, a equipe acredita no potencial do projeto e gostaria de continuar a desenvolvê-lo futuramente, refinando e melhorando as histórias presentes no \autoref{outras historias de usuario}, tendo em vista que construção de uma comunidade só se torna realidade quando mantida e visitada por seus integrantes. Portanto, percebe-se que o desenvolvimento do \gls{ifriends}, além de ter sido um espaço para aprimoramento profissional dos membros da equipe, também nos mostrou a relevância por trás da construção de uma cultura colaborativa, já que, além de procurarmos demonstrar isto pelo sistema, fizemos esforços para praticar nossos princípios junto aos colegas da própria disciplina de \acs{pds} - algo que esperançamos que se fortaleça ao longo das próximas gerações de alunos.
