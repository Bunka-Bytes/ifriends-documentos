% -----------------------------------------------------------------------
\chapter{Revisão da Literatura}
% -----------------------------------------------------------------------
Os assuntos tratados nesta seção pretendem apresentar o que já é sabido na comunidade científica, a respeito do tema proposto para ser discutido durante a execução do presente projeto. 
% -----------------------------------------------------------------------
\section{Construção de comunidades e redes aprendizado virtuais}

A partir da discussão iniciada no estudo de \citeonline{sartori2004comunidades} sobre as redes virtuais de aprendizado, toma-se como base a definição da construção de comunidades em torno de tais características como manifestações dos agrupamentos humanos dentro do ciberespaço. \citeonline{sartori2004comunidades} dão continuidade na introdução ao assunto, quando explicam o funcionamento da construção inicial de comunidades virtuais:
\begin{citacao}

Seu funcionamento está diretamente ligado, num primeiro momento, às redes de conexões proporcionadas pelas tecnologias de informação e comunicação e, num segundo momento, à possibilidade de, neste espaço, pessoas com objetivos comuns, se encontrarem, estabelecerem relações, e desenvolverem novas subjetividades \cite{sartori2004comunidades}.

\end{citacao}

Neste contexto, percebe-se que as características determinantes em uma comunidade virtual giram em torno ``de um sentimento de pertencimento e de um projeto comum'' oriundo da comunicação entre os indivíduos inseridos dentro deste espaço. Por conseguinte, \citeonline{sartori2004comunidades} afirmam que os grupos formados dentro de tais comunidades possuem vínculos sociais que se originam dos fluxos informacionais e comunicacionais, além das atividades e discussões fomentadas nas interações sociais, dentro das quais as pessoas formam unidades a partir de seus interesses comuns. 

Sabendo disso, dentro de sua argumentação, as autoras compreendem o conceito de socialidade presente nas dimensões de comunidades virtuais como um fenômeno independente de regulações que deve ser viabilizado ``através de comunicação multi-direcional que permite que os indivíduos possam estar ligados coletivamente'', além de completarem dizendo que o sentimento de pertencimento nestes espaços só é possível devido a ações executadas a distância, visto que a participação das pessoas tece “uma rede de cooperação oportunizada pelo processo de comunicação bidirecional'', explicam \citeonline{sartori2004comunidades}.

Por fim, \citeonline{sartori2004comunidades} finalizam a introdução ao assunto concluindo que para potencializar novas práticas educativas, comunicacionais, culturais e formar socialidades que possibilitem o ``exercício da cidadania, do desenvolvimento da cultura e de novos saberes'' em comunidades virtuais de aprendizagem, é necessário compreender, de antemão, que:
\begin{citacao}

Para que as comunidades virtuais de aprendizagem oportunizem a sensação de pertencimento, do “estar junto”, torna-se necessário investigar quais as características que devem estar presentes para possibilitar novas experiências culturais, sociais e educativas. Aqui se encontra, sem dúvida, um programa de estudos, pesquisas e reflexões que aprofundem o papel de formação de socialidades destas comunidades, ultrapassando a compreensão de suas possibilidades técnicas e de suas práticas protocolares \cite{sartori2004comunidades}.

\end{citacao}

Dentro desde cenário, é possível tentar compreender algumas das formas de estímulo para que a socialidade característica das comunidades virtuais se manifeste. Desse modo, o autor \citeonline{kratochwill2006possibilidades} descreve como exemplo a utilização de fóruns de discussão como parte do processo.
Segundo \citeonline{kratochwill2006possibilidades}, entende-se por fórum de discussão com finalidades educacionais um determinado espaço de comunicação formado por mensagens, que podem ou não seguir um padrão de classificação. Neste espaço, os usuários podem realizar contribuições, assim como esclarecer e contrariar os demais envolvidos de forma assíncrona.

O desenvolvimento de fóruns \textit{online} surge como uma nova perspectiva de escrita colaborativa capaz de fornecer a interação, debates e favorecer na aprendizagem colaborativa, diz \citeonline{kratochwill2006possibilidades}. O autor ainda afirma que alguns possíveis motivos para abrir um fórum \textit{online} de discussão seriam: o estímulo do espírito de participação, o compartilhamento de experiências, dúvidas e conhecimentos.


Assim, ficam notórias as contribuições que os ambientes virtuais podem trazer, isto, pois, \citeonline{kratochwill2006possibilidades} afirma ser interessante o desenvolvimento da dinâmica quando comenta:

\begin{citacao}

Torna-se interessante a dinâmica desenvolvida no fórum on-line pela sua perspectiva dialógica. Todos os participantes têm a oportunidade de se expressar, interferir e receber interferências, se constituir a partir da constituição do outro e da percepção do outro sobre a expressão do primeiro. Dentro desse processo dialógico, a autonomia e a autoria se constituem em respeito à alteridade, à individualidade e, ao mesmo tempo, em que coletivamente \cite{kratochwill2006possibilidades}.

\end{citacao}

Por outro lado, seguindo a mesma linha de pensamento, outro instrumento para estimular tais características na construção de comunidades de aprendizagem, pode ser a troca de conhecimentos através de monitorias, conforme conta \citeonline{friedlander1984alunos} ao afirmar que as monitorias ``proporcionam ampla troca dos mais diversos conhecimentos entre as partes envolvidas: alunos, monitores e professores''. 
Conforme \citeonline{matoso2014importancia} conta em seu relato de experiência como monitor, a monitoria trouxera-lhe ganhos importantes inclusive em aspectos interpessoais, isto porque, devido à grande procura dos alunos à monitoria apenas no período de provas, muitos chegavam angustiados e tristes, exigindo-lhe uma postura mais séria e confiante para lidar com estes estudantes.

Portanto, a importância de ser um monitor, segundo \citeonline{lins2009importancia}, está principalmente relacionada ao fato de que a melhor forma de se aprender um conteúdo é transmitindo o conhecimento, e neste contexto, o fácil diálogo do monitor com os professores e alunos, o torna o principal receptor nesta rede de troca de conhecimentos. Com isso, é possível aproximar ainda mais os alunos de seu processo educacional, tendo em vista que a troca de conhecimentos e experiências pode ser feita entre os próprios estudantes - praticando assim a socialidade, conforme demonstrado nos argumentos de \citeonline{sartori2004comunidades}.

% -----------------------------------------------------------------------
\section{Interatividade e propostas de gamificação}
Dando continuidade na reflexão acerca de métodos de aprimoramento da experiência de aprendizagem, a gamificação se coloca como um instrumento poderoso para a criação de espaços que sejam mediados por expressões de desafios e entretenimento, conforme contam \citeonline{valentim2016interatividade} em seu artigo, onde consideram a gamificação como aliada “na direção de novas formas de ensinar e aprender, e no cenário educativo online, alia-se à perspectiva da interatividade, novas possibilidades para a prática educativa” \cite{valentim2016interatividade}.

Ainda que este campo de estudo seja relativamente novo na literatura, as autoras argumentam que, dentro do contexto da Sociedade Digital, a interatividade é potencializada pelo uso de estratégias de gamificação, que envolve “uma diversidade de tecnologias no ambiente virtual, com a multiplicidade de interfaces existente”, tendo em vista que com ela é possível criar atividades diferenciadas que motivem a participação do estudante no espaço de aprendizagem. Com isso, \citeonline{valentim2016interatividade} completam que:

\begin{citacao}

[…] A interatividade se consolida como através da possibilidade de o indivíduo intervir na mensagem (Participação-Intervenção), de colaboração e coautoria, onde não existe fronteiras entre emissão e recepção (Bidirecionalidade-Hibridação) e por fim concretiza novas possibilidades de combinar informações produzir narrativas, múltiplas redes de conexões (Potencialidade-Permutabilidade). Interatividade é mais interação, participação, trocas, autoria, coautoria, cooperação, permite o compartilhamento, enfim, aprender através do diálogo e pelas múltiplas interfaces, construir aprendizagem colaborativa […] \cite{valentim2016interatividade}.

\end{citacao}
Seguindo uma linha de raciocínio similar, \citeonline{costa2018revisao} conta que a gamificação tem como objetivo motivar os usuários no processo de aprendizagem e participação na ferramenta na qual é utilizada essa metodologia, ela possui diversos elementos e também possíveis subcomponentes que, juntos, auxiliam para o seu propósito, sendo eles: exploração, cooperação, \textit{feedback}, antecipação, inovação, conquistas, sistema de recompensas, cultura de motivação e social. 

Os autores ressaltam ainda que esses elementos não compõem obrigatoriamente a gamificação, porém, quanto maior a presença deles, mais eficaz torna-se sua ferramenta. Além disso, não existem elementos mais importantes que outros, afirmam \citeonline{costa2018revisao}. Porém, dentre os elementos descritos por \citeonline{costa2018revisao}, o  \textit{feedback}  se destaca em sua característica motivacional, visto que pode estimular o usuário a realizar determinada ação quando aliado a subcomponentes como:

\begin{citacao}

Ao acertar, errar ou conquistar algo dentro de um sistema gamificado, o usuário deve obter uma resposta rápida e precisa, ou \textit{feedback} instantâneo. Ao utilizar algo simbólico como forma de responder ao usuário, o sistema acaba com isso, aumentando ou diminuindo a motivação dele para determinada ação, pois ao ver que o sistema responde algo quando se acerta ou erra, o usuário se sente encorajado a continuar ou não a atividade proposta. São elementos que incorporam o aspecto de \textit{feedback} da gamificação \cite{costa2018revisao}.

\end{citacao}

Além disso, o artigo conta que associados a estes elementos estão os emblemas, crachás ou distintivos (também conhecidos como \textsl{badges}, que trazem uma característica mais próxima a jogos quando aliados ao uso de pontuação (acúmulo de pontos que pode ser numérico ou simbólico), sendo estes elementos possíveis de serem utilizados juntos como um sistema de níveis a serem acrescidos para o usuário, descrevem \citeonline{costa2018revisao} e completa  \citeonline{zichermann2011gamification}.

Dessa forma, de acordo com \citeonline{orrico2012mercado}, um dos motivos para a utilização da gamificação como proposta de usabilidade, se dá pela grande base de pessoas que jogam ao redor do mundo, muitos se sentem mais motivados e dão retornos positivos quando existem elementos de jogos, os quais já são algo do cotidiano de grande parcela da população. Porém, é possível concluir que, ainda que a gamificação traga diversos benefícios, é preciso partir de determinados pressupostos para que seja aplicada de maneira adequada a somar com a interatividade no processo educativo. 

Isto, pois, conforme contam \citeonline{valentim2016interatividade}, enquanto a interatividade ``é mais participação, mais trocas, autoria, colaboração, coautoria, é um mais comunicacional'', a gamificação urge como aparelho de inovação no processo de educação digital para concretizar atividades colaborativas diferenciadas, conforme completa ao descrever:

\begin{citacao}
Cruzadinhas,  produção de  podcast,  videocast,  desafios  surpresas  aliados  a  interfaces  de colaboração do  AVA  como  chat,  fórum  de  discussão,  portfólio e  elementos  dos  games  como ranking,  desafios,  colaboração,  competição, pode  possibilitar  com  que  haja  motivação  pelos discentes, engajamento no processo educativo, e assim compreendemos a gamificação como elemento que pode favorecer a concretização da interatividade \cite{valentim2016interatividade}.
\end{citacao}


% -----------------------------------------------------------------------