%--------------------------------------------------------------
\chapter{Histórico de Desenvolvimento}
%--------------------------------------------------------------
Assim como mencionado na \autoref{metodologia agil}, a equipe optou pelo uso da metodologia ágil Scrum, dessa forma o processo de desenvolvimento foi organizado por meio de \glspl{Sprint}, estas foram planejadas a partir das reuniões organizadas entre os membros da equipe. Logo, a seguinte seção visa descrevê-las, assim como apresentar a sua duração.

\begin{itemize}
%---
\item {\textbf{Sprint I: 17/04 a 01/05}}
    
A primeira \gls{Sprint} do projeto em suma visou a inicialização do épico de Gestão de Perguntas, visto que após a realização de estimativas com o \textsl{Scrum Poker} das histórias de usuário, a equipe concordou que este épico era o indicado para dar início ao desenvolvimento do projeto. Dessa forma, foi planejada ainda para a mesma \gls{Sprint} a inicialização da prototipagem das telas, assim como o uma modelagem de dados inicial.
%---
\item {\textbf{Sprint II: 02/05 a 09/05}}
    
A segunda \gls{Sprint} do projeto visou a finalização de dois épicos, isto é, a Gestão de Respostas e a Gestão de Perguntas. Além disso, para esta \gls{Sprint} estava planejada a apresentação da \gls{POC}, logo, também era esperada uma apresentação bem estruturada, assim como o documento de visão/relatório da \gls{POC}. 

Porém, essa \gls{Sprint} não foi concluída completamente, visto que os épicos não foram finalizados com todos os critérios de aceitação prontos, porém foi possível entregar todos os que eram relacionados a construção da \acs{api}. Logo, passamos as tarefas faltantes para a próxima Sprint.
%---
\item {\textbf{Sprint III: 16/05 a 29/05}}
    
Após a apresentação da \gls{POC} do projeto, para a nossa terceira \gls{Sprint} planejou-se realizar os ajustes que ficaram pendentes, sendo na documentação, no \textit{deploy} e na segurança da aplicação. 
%---
\item {\textbf{Sprint IV: 30/05 a 12/06}}
    
Após concluir os ajustes pendentes, para a quarta \gls{Sprint} se planejou estabelecer de vez uma boa conexão com a API no \textsl{\gls{front-end}} para que todas as requisições do épico de Gestão de Perguntas e o épico de Gestão de Respostas tenham um bom resultado, utilizando a autenticação do usuário. Na parte da documentação se planejou fazer a organização do documento, assim como atualizar as modificações realizadas no desenvolvimento até o momento. 
%---
\item {\textbf{Sprint V: 12/06 a 20/06}}

Devido aos contratempos encontrados na disciplina de \acs{pds} foi possível criar uma \gls{Sprint} extra antes da primeira apresentação parcial, logo, a quinta \gls{Sprint} visou à finalização e melhoria das tarefas pendentes da \gls{Sprint} anterior assim como a preparação da apresentação do projeto. 
%---
\item \textbf{{Sprint VI: 31/07 a 21/08}}

A sexta \gls{Sprint} do projeto teve foco em efetuar a revisão do \textit{backlog} e realizar os ajustes necessários na documentação visando a entrega final. Além disso, a aplicação \textsl{\gls{front-end}} foi hospedada no \gls{vercel} e finalizamos as partes faltantes como a prototipação e a revisão de literatura. Vale ressaltar também que devido à revisão do \textit{backlog} foi possível desenvolver um plano de entregas (\autoref{plano_entrega}).
%---
\item \textbf{{Sprint VII: 21/08 a 04/09}}
A nossa última \gls{Sprint} prevê a entrega da documentação, anteriormente desenvolvida, o desenvolvimento do projeto, conforme as histórias de usuários definidas, assim como realizar a preparação da apresentação final. 
\end{itemize}

% -------------------------------------------------------------
%--------------------------------------------------------------
\section{Descartes}
%--------------------------------------------------------------
A seguinte seção foi criada visando fornecer informações a respeitos dos descartes realizados a partir do espoco inicial proposto para o projeto. 

O primeiro descarte a ser feito foi em relação ao uso do \gls{scoold} no desenvolvimento, inicialmente a equipe pretendia fazer uso dessa plataforma visto que as funcionalidades que ela trazia se assemelhavam a boa parte do que se pretendia desenvolver no \gls{ifriends}. 

Mas, após fazer os estudos iniciais a respeito de como fazer essa implementação e de como fazer a integração com os outros recursos que ainda precisariam ser desenvolvidos, a equipe denotou que a tarefa se tornaria mais complexa do que o esperado, já que era necessário um estudo mais aprofundado, e como não dispomos um tempo de desenvolvimento significativo decidiu-se usar o \gls{scoold} apenas como uma inspiração. Vale lembrar que o descarte da plataforma já tinha sido previsto no escopo inicial do projeto, então o tempo aplicado no estudo não foi perdido e se tomou como aprendizado.

Outro descarte/problema que enfrentamos foi a inexistência da seção de testes unitários no \gls{front-end}, tendo em vista que pecamos com planejamento para a sua inclusão desde o início do desenvolvimento, e, por este motivo, não conseguimos incluir a porcentagem de cobertura de testes, com a inclusão da biblioteca JEST no ReactJS, a tempo da entrega final deste documento, porém é possível executar os testes através do comando especificado na documentação do projeto (arquivo .md) descrita dentro do repositório da equipe no sistema de versionamento de código escolhido, disponível na pasta \textsl{ifriends-web}.

Vale ressaltar que, esse último descarte, poderia motivar, por exemplo, a escolha de um modelo de trabalho no \gls{front-end} que pudesse facilitar a construção dos testes sem muitas dificuldades, seja com a adesão de algum \gls{framework2} específico que trouxesse isso por padrão, ou com a utilização de metodologias como o desenvolvimento orientado a testes.

 
%--------------------------------------------------------------
 \section{Mudanças}
%--------------------------------------------------------------
Visando também pelas mudanças, a seguinte seção tem o intuito de apresentá-las seguida dos critérios considerados para a sua aplicação no projeto. 
 
A primeira mudança mais evidente foi realizada sobre o \gls{heroku} em relação ao \gls{front-end} do sistema: após várias investigações realizadas pela equipe, foi encontrado o \gls{vercel} como melhor opção de servidor de hospedagem para substituir o \gls{heroku}, visto que ele atendia todos os \href{https://vercel.com/blog/automatic-ssl-with-vercel-lets-encrypt}{requisitos de segurança faltantes}, além de que foi constatado pela equipe posteriormente que o \textsl{Buildpack} utilizado para subir a aplicação no \gls{heroku} seria \href{https://github.com/mars/create-react-app-buildpack}{descontinuado}, o que também contribuiu para a mudança. Para o \gls{back-end}, por outro lado, tentou-se encontrar um serviço de hospedagem que cumprisse todos os requisitos existentes no \gls{heroku}, mas não se obteve sucesso na busca. Como os requisitos de segurança foram cumpridas de forma satisfatória, a equipe considerou que não seria considerado urgente mudar a hospedagem da \gls{api} neste momento.

Por último, precisamos mudar também a quantidade total de estudantes a serem entrevistados no nosso plano de testes, tendo em vista que o número estipulado inicialmente não comportou a agenda que os integrantes da equipe tinham disponível, pois as entrevistas feitas duraram mais tempo do que imaginamos e, portanto, a quantidade original se tornou inviável. De todo modo, disponibilizamos a pesquisa de \acs{nps} para que possamos centralizar os \textit{feedbacks} dados e ainda manter o processo de melhoria contínua com base nela, sem descartar a possibilidade de realizar eventuais entrevistas no futuro.
% -------------------------------------------------------------