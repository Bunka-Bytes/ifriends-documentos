%--------------------------------------------------------------
\chapter{Histórico de Desenvolvimento}
%--------------------------------------------------------------
Assim como mencionado na \autoref{metodologia agil}, a equipe optou pelo uso da metodologia ágil Scrum, dessa forma o processo de desenvolvimento foi organizado por meio de \glspl{Sprint}, estas foram planejadas a partir das reuniões organizadas entre os membros da equipe. Logo, a seguinte seção visa descrevê-las, assim como apresentar a sua duração.

\begin{itemize}
%---
\item {\textbf{Sprint I: 17/04 a 01/05}}
    
A primeira \gls{Sprint} do projeto em suma visou a inicialização do épico de Gestão de Perguntas, visto que após a realização de estimativas com o \textsl{Scrum Poker} das histórias de usuário, a equipe concordou que este épico era o indicado para dar início ao desenvolvimento do projeto. Dessa forma, foi planejada ainda para a mesma \gls{Sprint} a inicialização da prototipagem das telas, assim como o uma modelagem de dados inicial.
%---
\item {\textbf{Sprint II: 02/05 a 09/05}}
    
A segunda \gls{Sprint} do projeto visou a finalização de dois épicos, isto é, a Gestão de Respostas e a Gestão de Perguntas. Além disso, para esta \gls{Sprint} estava planejada a apresentação da \gls{POC}, logo, também era esperada uma apresentação bem estruturada, assim como o documento de visão/relatório da \gls{POC}. 

Porém, essa \gls{Sprint} não foi concluída completamente, visto que os épicos não foram finalizados com todos os critérios de aceitação prontos, porém foi possível entregar todos os que eram relacionados a construção da \acs{api}. Logo, passamos as tarefas faltantes para a próxima Sprint.
%---
\item {\textbf{Sprint III: 16/05 a 29/05}}
    
Após a apresentação da \gls{POC} do projeto, para a nossa terceira \gls{Sprint} planejou-se realizar os ajustes que ficaram pendentes, sendo na documentação, no \textit{deploy} e na segurança da aplicação. 
%---
\item {\textbf{Sprint IV: 30/05 a 12/06}}
    
Após concluir os ajustes pendentes, para a quarta \gls{Sprint} se planejou estabelecer de vez uma boa conexão com a API no \textsl{\gls{front-end}} para que todas as requisições do épico de Gestão de Perguntas e o épico de Gestão de Respostas tenham um bom resultado, utilizando a autenticação do usuário. Na parte da documentação se planejou fazer a organização do documento, assim como atualizar as modificações realizadas no desenvolvimento até o momento. 
%---
\item {\textbf{Sprint V: 12/06 a 20/06}}

Devido aos contratempos encontrados na disciplina de \acs{pds} foi possível criar uma \gls{Sprint} extra antes da primeira apresentação parcial, logo, a quinta \gls{Sprint} visou à finalização e melhoria das tarefas pendentes da \gls{Sprint} anterior assim como a preparação da apresentação do projeto. 
%---
\item \textbf{{Sprint VI: 31/07 a 21/08}}

A sexta \gls{Sprint} do projeto teve foco em efetuar a revisão do \textit{backlog} e realizar os ajustes necessários na documentação visando a entrega final. Além disso, a aplicação \textsl{\gls{front-end}} foi hospedada no \gls{vercel} e finalizamos as partes faltantes como a prototipação e a revisão de literatura. Vale ressaltar também que devido à revisão do \textit{backlog} foi possível desenvolver um plano de entregas (\autoref{plano_entrega}).
%---
\item \textbf{{Sprint VII: 21/08 a 04/09}}
A nossa última \gls{Sprint} prevê a entrega da documentação, anteriormente desenvolvida, o desenvolvimento do projeto, conforme as histórias de usuários definidas, assim como realizar a preparação da apresentação final. 
\end{itemize}

% -------------------------------------------------------------
%--------------------------------------------------------------
\section{Descartes}
%--------------------------------------------------------------
A seguinte seção foi criada visando fornecer informações a respeitos dos descartes realizados a partir do espoco inicial proposto para o projeto. 

O primeiro descarte a ser feito foi em relação ao uso do \gls{scoold} no desenvolvimento, inicialmente a equipe pretendia fazer uso dessa plataforma visto que as funcionalidades que ela trazia se assemelhavam a boa parte do que se pretendia desenvolver no \gls{ifriends}. 

Mas, apos fazer os estudos iniciais a respeito de como fazer essa implementação e de como fazer a integração com os outros recursos que ainda precisariam ser desenvolvidos, a equipe denotou que a tarefa se tornaria mais complexa do que o esperado, já que era necessário um estudo mais aprofundado, e como não dispomos um tempo de desenvolvimento significativo decidiu-se usar o \gls{scoold} apenas como uma inspiração. Vale lembrar que o descarte da plataforma já tinha sido previsto no escopo inicial do projeto, então o tempo aplicado no estudo não foi perdido e se tomou como aprendizado.

%--------------------------------------------------------------
% \section{Mudanças}
%--------------------------------------------------------------
% Visando também pelas mudanças, a seguinte seção tem o intuito de apresentá-las seguida dos critérios considerados para a sua aplicação no projeto. 
% -------------------------------------------------------------