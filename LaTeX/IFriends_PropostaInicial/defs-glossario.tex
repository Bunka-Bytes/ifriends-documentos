%%%%%%%%%%%%%%%%%%%%%%%%%%%%%%%%%%%%%%%%%%%%%%%%%%%%%%%%%%%%
% Normalmente somente as palavras referenciadas são impressas no glossário, portanto é necessário referenciar utilizando \gls{identificação}
%%%%%%%%%%%%%%%%%%%%%%%%%%%%%%%%%%%%%%%%%%%%%%%%%%%%%%%%%%%%
\newglossaryentry{canva} {
    name=Canva,
    plural= {Canva},
    description={Plataforma de design gráfico que permite a criação de gráficos de mídia social, apresentações, infográficos, pôsteres e outros conteúdos visuais}
}
%%%%%%%%%%%%%%%%%%%%%%%%%%%%%%%%%%%%%%%%%%%%%%%%%%%%%%%%%%%%
\newglossaryentry{ifriends} {
    name= IFriends,
    plural= {IFriends},
    description={Nome dado ao projeto de sistemas desenvolvido, cujo significado se dá num trocadilho na junção das palavras friends (amigos, em inglês) e IF (Instituito Federal).}
}
%%%%%%%%%%%%%%%%%%%%%%%%%%%%%%%%%%%%%%%%%%%%%%%%%%%%%%%%%%%%
\newglossaryentry{googleforms} {
    name= Google Forms,
    plural= {Google Forms},
    description={Ferramenta da Google para gerenciamento de pesquisas e formulários, utilizada para coletar e registrar informações de outras pessoas. }
}
%%%%%%%%%%%%%%%%%%%%%%%%%%%%%%%%%%%%%%%%%%%%%%%%%%%%%%%%%%%%
\newglossaryentry{WhatsApp} {
    name= WhatsApp,
    plural= {WhatsApp},
    description={Aplicativo de mensagens instântaneas e chamadas de voz para smartphones. }
}
%%%%%%%%%%%%%%%%%%%%%%%%%%%%%%%%%%%%%%%%%%%%%%%%%%%%%%%%%%%%
\newglossaryentry{gamificação} {
    name= gamificação,
    plural= {gamificações},
    description={Aplicação das estratégias dos jogos nas atividades do dia a dia, com o objetivo de aumentar o engajamento dos participantes. Se baseia no game thinking, que integra a gamificação com outros saberes do meio corporativo e do design. }
}
%%%%%%%%%%%%%%%%%%%%%%%%%%%%%%%%%%%%%%%%%%%%%%%%%%%%%%%%%%%%