
% Definições para glossario

% ATENCAO o SHARELATEX GERA O GLOSSARIO/LISTAS DE SIGLAS SOMENTE UMA VEZ
% CASO SEJA FEITA ALGUMA ALTERAÇÃO NA LISTA DE SIGLAS OU GLOSSARIO É NECESSARIO UTILIZAR A OPÇÃO :
% "Clear Cached Files" DISPONIVEL NA VISUALIZAÇÃO DOS LOGS 
% ---
% https://www.sharelatex.com/learn/Glossaries

\newglossaryentry{Sprint} {
    name=Sprint,
    plural= {Sprints},
    description={Tempo estimado para promover um desenvolvimento mais focalizado no projeto.}
}

\newglossaryentry{SpringBoot} {
    name=SpringBoot,
    plural= {SpringBoot},
    description={É uma ferramenta que visa facilitar o processo de configuração e publicação de aplicações que utilizem o ecossistema Spring.}
}

\newglossaryentry{Eclipse} {
    name=Eclipse,
    plural= {Eclipse},
    description={É uma IDE para desenvolvimento Java, com suporte a várias outras linguagens a partir de plugins. Ele foi feito em Java e segue o modelo open source de desenvolvimento de software.}
}

\newglossaryentry{JUnit} {
    name=JUnit,
    plural= {JUnit},
    description={É um Framework de código aberto que facilita a criação de testes automatizados.}
}

\newglossaryentry{Yamllint}{
    name= Yamllint,
    plural= {Yamllint},
    description= {Validador de arquivos .yaml}
}

\newglossaryentry{linkedin} {
    name=LinkedIn,
    plural= {LinkedIn},
    description={Rede social voltada para relacionamentos profissionais}
}

\newglossaryentry{canva} {
    name=Canva,
    plural= {Canva},
    description={Plataforma de design gráfico que permite a criação de gráficos de mídia social, apresentações, infográficos, pôsteres e outros conteúdos visuais}
}

\newglossaryentry{rugbi} {
    name= R{\'u}gbi,
    plural= {R{\'u}gbi},
    description={Esporte em que duas equipes de 15 jogadores se enfrentam, usando as mãos e os pés, na tentativa de levar a bola oval at{\'e} a linha de fundo adversária ou faz{\^e}-la passar por entre as traves da meta, sobre aquela linha }
}
\newglossaryentry{framework}{
    name = Framework,
    plural = Frameworks,
    description={Um framework ou arcabouço conceitual, é um conjunto de conceitos usado para resolver um problema de um domínio específico.}
}
\newglossaryentry{framework2}{
    name = Framework,
    plural = Frameworks,
    description={Uma abstração que une códigos comuns entre vários projetos de software provendo uma funcionalidade genérica.}
}
\newglossaryentry{front-end} {
    name= Front-end,
    plural= {front-end},
    description={Refere-se a parte visual e gráfica da interface de um sistema, elaborado por meio de outras linguagens e tecnologias.}
}
\newglossaryentry{back-end} {
    name= Back-end,
    plural= {back-end},
    description={Refere-se a parte que está por trás da aplicação, responsável pela manipulação de dados voltada para o funcionamento interno de um sistema}
}

\newglossaryentry{ifriends} {
    name= IFriends,
    plural= {IFriends},
    description={Nome dado ao projeto de sistemas desenvolvido, cujo significado se dá num trocadilho na junção das palavras friends (amigos, em inglês) e IF (Instituito Federal).}
}

\newglossaryentry{friend} {
    name= Friend,
    plural= {Friends},
    description={Nome dado aos usuários do sistema IFriends para exemplificar por meio deles as funcionalidades da aplicação.}
}

\newglossaryentry{pomodoro} {
    name= Pomodoro,
    plural= {Pomodoros},
    description={É um método de gerenciamento de tempo. A técnica consiste na utilização de um cronômetro para dividir o trabalho em períodos de 25 minutos, separados por breves intervalos.}
}

\newglossaryentry{c-sharp} {
    name= C-Sharp,
    plural= {C-Sharp},
    description={Linguagem de programação, multiparadigma, de tipagem forte, desenvolvida pela Microsoft como parte da plataforma .NET. }
}

\newglossaryentry{googleforms} {
    name= Google Forms,
    plural= {Google Forms},
    description={Ferramenta da Google para gerenciamento de pesquisas e formulários, utilizada para coletar e registrar informações de outras pessoas. }
}

\newglossaryentry{WhatsApp} {
    name= WhatsApp,
    plural= {WhatsApp},
    description={Aplicativo de mensagens instântaneas e chamadas de voz para smartphones. }
}

\newglossaryentry{gamificação} {
    name= gamificação,
    plural= {gamificações},
    description={Aplicação das estratégias dos jogos nas atividades do dia a dia, com o objetivo de aumentar o engajamento dos participantes. Se baseia no game thinking, que integra a gamificação com outros saberes do meio corporativo e do design. }
}

\newglossaryentry{heroku} {
    name= Heroku,
    plural= {Heroku},
    description={Plataforma de nuvem como serviço que suporta várias linguagens de programação.}
}

\newglossaryentry{statsvn} {
    name= StatSVN,
    plural= {StatSVN},
    description={Ferramenta que funciona a partir de arquivos de log extraídos do
    repositório do SVN, fornecendo gráficos e dados estatísticos a
    partir do cruzamento dessas informações.}
}

\newglossaryentry{laravel}{
    name= Laravel,
    plural= {Laravel},
    description= {Framework PHP livre e open-source criado por Taylor B. Otwell para o desenvolvimento de sistemas web que utilizam o padrão MVC.}
}   

\newglossaryentry{react-bootstrap}{
    name= React-Bootstrap,
    plural= {React-Bootstrap},
    description= {biblioteca que oferece os componentes clássicos do Bootstrap construídos em React.}
}   

\newglossaryentry{ant}{
    name= Ant,
    plural= {Ant},
    description= {biblioteca React UI antd que auxilia na criação de interfaces interativas.}
} 

\newglossaryentry{discord}{
    name= Discord,
    plural= {Discord},
    description= {Aplicativo de comunicação instantânea, muito utilizado pela comunidade \textsl{gamer}, por sua simplicidade em possibilitar troca de mensagens, áudio, texto e vídeo.}
}   

\newglossaryentry{figma}{
    name= Figma,
    plural= {Figma},
    description= {Plataforma de criação de gráficos e prototipagem de projetos, focados principalmente em aplicações web.}
}

\newglossaryentry{youtube}{
    name= YouTube,
    plural= {YouTube},
    description= {Plataforma destinada a compartilhamento de vídeos.}
}

\newglossaryentry{subversion}{
    name= Subversion,
    plural= {Subversion},
    description= {Sistema de controle de versão.}
}

\newglossaryentry{notion}{
    name= Notion,
    plural= {Notion},
    description= {Aplicação que contém ferramentas úteis para o gerenciamento do projeto e da equipe, uma delas sendo o quadro de kanban.}
}

\newglossaryentry{moodle}{
    name= Moodle,
    plural= {Moodle},
    description= {Ambiente virtual de ensino com o objetivo de auxiliar à aprendizagem dos alunos.}
}

\newglossaryentry{scoold}{
    name= Scoold,
    plural= {Scoold},
    description= {Código aberto que funciona como uma plataforma de perguntas e respostas, fórum, base de conhecimento ou suporte ao cliente.}
}

\newglossaryentry{svn}{
    name= SVN,
    plural= {SubVersion},
    description= {Sistema de controle de versão de uso obrigatório na disciplina.}
}

\newglossaryentry{gource}{
    name= Gource,
    plural= {Gource},
    description= {Ferramenta utilizada para visualização em forma de diagramas e vídeos o desenvolvimento de um software.}
}

\newglossaryentry{postegreSQL}{
    name= PostgreSQL,
    plural= {PostgreSQL},
    description= {Sistema gerenciador de banco de dados objeto relacional, desenvolvido como projeto de código aberto. }
}

\newglossaryentry{Python}{
    name= Python,
    plural= {Python},
    description= {Linguagem de programação de alto nível, interpretada por \textit{Script}, de tipagem dinâmica e forte.}
}

\newglossaryentry{REST API}{
    name= REST API,
    plural= {REST APIs},
    description= {Arquitetura de sistema para serviços web. }
}

\newglossaryentry{Node.js}{
    name= Node.js,
    plural= {Node.js},
    description= {Software de código aberto que possibilita a execução de códigos JavaScript fora dos navegadores Web.}
}

\newglossaryentry{Overleaf}{
    name= Overleaf,
    plural= {Overleaf},
    description= {Editor para LaTeX em nuvem que permite aos colaboradores escrever, editar e publicar documentos devidamente formatados.}
}

\newglossaryentry{Maven}{
    name= Maven,
    plural= {Maven},
    description= {ferramenta de compilação e gerenciamento de projetos Java.}
}

\newglossaryentry{endpoint}{
    name= endpoint,
    plural= {endepoints},
    description= {URL que permite o acesso ao serviço por meio de uma aplicação cliente.}
}


\newglossaryentry{openAPI}{
    name= OpenAPI,
    plural= {OpenAPI},
    description= {Conhecida anteriormente com Swagger, é uma especificação para arquivos de interface entendível por máquina.}
}

\newglossaryentry{imgbb}{
    name= ImgBB,
    plural= {ImgBB},
    description= {Site de hospedagem de imagens.}
}

\newglossaryentry{github}{
    name= GitHub,
    plural= {GitHub},
    description= {Ferramenta de hospedagem e versionamento de código.}
}

% Normalmente somente as palavras referenciadas são impressas no glossário, portanto é necessário referenciar utilizando \gls{identificação}                
