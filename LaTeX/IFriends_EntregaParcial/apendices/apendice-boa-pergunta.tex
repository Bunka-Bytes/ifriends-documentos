\chapter{Boas práticas para uma pergunta}
\label{boa-pergunta}
% ----------------------------------------------------------
 
Para auxiliar na elaboração de perguntas, foi disponibilizado aos \glspl{friend} um pequeno tutorial de como pode ser feita uma pergunta. 

\begin{enumerate}
%---
\item \textbf{Resuma o seu problema:} 
%---
\begin{itemize}
\item Antes de fazer uma pergunta tenha em mente "qual é o problema". Para isso é recomendado que primeiramente se atente a reunir detalhes e informações que poderão ser uteis.
\item Considere que uma pergunta não se caracteriza como boa devido ao seu tamanho, mas sim devido às informações fornecidas.
\item Opte por um título sucinto e detalhado, e recorra a termos chaves, as tags podem ajudar nesse sentido.
\end{itemize}
%---
\item \textbf{Descreva o seu problema:}
%---
\begin{itemize}
\item Apresente seu problema com o máximo de detalhes, o que você já tentou e conte-nos o que você conseguiu até então.
\item Lembre-se que conseguirá melhores respostas quando você fornecer e detalhar melhor os seus dados.
\item Quando for apropriado, recorra a imagens para exemplificar melhor o problema.
\end{itemize}
%---
\item \textbf{Objetivo final:}
%---
\begin{itemize}
\item O que é preciso para chegar a um resultado viável, tente ser o mais claro possível em expressar qual é o seu objetivo.
\end{itemize}

\end{enumerate}