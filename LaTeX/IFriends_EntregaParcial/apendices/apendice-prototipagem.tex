\chapter{Prototipação das telas}
\label{prototipação}
% ----------------------------------------------------------
A seguinte seção apresentará as figuras referentes a prototipagem de alta fidelidade do projeto \gls{ifriends}. Para a prototipação levou-se em consideração, dois estados, sendo eles o do usuário com \textit{login} e o do usuário sem \textit{login}. 

%--------------------------------------------------------------
\section{\gls{friend} sem \textit{login}}
%--------------------------------------------------------------
Quando um usuário externo ou até mesmo um \gls{friend} que ainda não tenha cadastro na aplicação decidirem visualizar ou conhecer a comunidade eles podem, mas os mesmos terão algumas funcionalidades bloqueadas, podendo apenas visualizar o sistema.

Dessa forma, a \autoref{Home Page s/ login} apresenta a primeira tela exibida para o \gls{friend} ou usuário externo quando ele faz o primeiro contato com o sistema. Nela o \gls{friend} pode navegar através de dois menus disponíveis na página: o lateral e o superior, usar a barra de pesquisa e \textit{logar} no seu perfil.

\begin{figure}[htb]
\centering
\caption{\label{Home Page s/ login} Página inicial sem \textit{login}}
\includesvg[inkscapelatex=false,width=1\textwidth]{anexos/Imagens_Prototipo/2_home_page.svg}
\fonte{os autores}
\end{figure}
\FloatBarrier

A \autoref{pergunta e respostas s/ login} representa a página de perguntas e suas respostas.

\begin{figure}[htb]
\centering
\caption{\label{pergunta e respostas s/ login} Página da pergunta sem \textit{login}}
\includesvg[inkscapelatex=false,width=1\textwidth]{anexos/Imagens_Prototipo/2_pergunta_e_respostas.svg}
\fonte{os autores}
\end{figure}
\FloatBarrier

A \autoref{cadastro perguntas s/ login} corresponde a página de cadastro de perguntas, perceba que quando o \gls{friend} não está \textit{logado} no sistema o mesmo bloqueia o processo pedindo primeiro a autenticação do mesmo.

\begin{figure}[htb]
\centering
\caption{\label{cadastro perguntas s/ login} Página de cadastro de perguntas sem \textit{login}}
\includesvg[inkscapelatex=false,width=0.7\textwidth]{anexos/Imagens_Prototipo/erro_cadastro_perguntas.svg}
\fonte{os autores}
\end{figure}
\FloatBarrier

Caso o \gls{friend} escolha \textit{logar} no sistema será exibida a tela de \textit{login} (\autoref{login}). Nela, para fazer a autenticação é solicitada o e-mail e a senha cadastrados, e caso o \gls{friend} esqueça a sua senha ele consegue recurar por meio do \textit{link} ``Esqueceu a sua senha?'', e se desejar continuar navegando sem fazer \textit{login} ele também pode basta clicar no \textit{link} ``Continuar sem login''.

\begin{figure}[htb]
\centering
\caption{\label{login} Página de \textit{login}}
\includesvg[inkscapelatex=false,width=0.7\textwidth]{anexos/Imagens_Prototipo/login.svg}
\fonte{os autores}
\end{figure}
\FloatBarrier

Já, caso ele opte por realizar o cadastro será exibida a \autoref{cadastro}, que corresponde a página de cadastro, nela serão necessárias algumas informações como o nome, 

\begin{figure}[htb]
\centering
\caption{\label{cadastro} Página de cadastro}
\includesvg[inkscapelatex=false,width=1\textwidth]{anexos/Imagens_Prototipo/cadastro.svg}
\fonte{os autores}
\end{figure}
\FloatBarrier

%--------------------------------------------------------------
\section{\gls{friend} com \textit{login}}
%--------------------------------------------------------------
Já para o \gls{friend} \textit{logado} no sistema, não há restrições de funcionalidades, então desde que as use com responsabilidade e da forma certa a aplicação cumprirá o que foi proposto.

Logo, a \autoref{Home Page} apresenta a página inicial do projeto, onde o \gls{friend} entra em contato pela primeira vez com o sistema. Nela o \gls{friend} pode navegar através de dois menus disponíveis na página: o lateral e o superior, usar a barra de pesquisa, \textit{logar} no seu perfil, acessar as suas configurações, entre outras ações disponibilizadas. Na página encontram as questões mais relevantes da comunidade, assim como os espaços destinados para a realização de uma pergunta ou de uma monitoria.

\begin{figure}[htb]
\centering
\caption{\label{Home Page} Página inicial}
\includesvg[inkscapelatex=false,width=1\textwidth]{anexos/Imagens_Prototipo/1_home_page.svg}
\fonte{os autores}
\end{figure}
\FloatBarrier

Quando o \gls{friend} clica em uma pergunta ou em ``Responder'' ele é direcionado à página dessa pergunta como mostra a \autoref{Pergunta e respostas}, nela ele pode encontrar as respostas já fornecidas por outros membros da comunidade, assim como também pode deixar a sua contribuição.

\begin{figure}[htb]
\centering
\caption{\label{Pergunta e respostas} Pergunta e respostas}
\includesvg[inkscapelatex=false,width=1\textwidth]{anexos/Imagens_Prototipo/1_pergunta_e_respostas.svg}
\fonte{os autores}
\end{figure}
\FloatBarrier

Já a \autoref{Cadastro de perguntas} corresponde a página de cadastro de perguntas, nessa tela são apresentados todos os elementos julgados necessários para a sua realização, nesta página ainda de encontra o manual de uma boa pergunta (\autoref{boa-pergunta}), tal foi elaborado com o intuito de ajudar e auxiliar o \gls{friend} na preparação de sua problemática. 

\begin{figure}[htb]
\centering
\caption{\label{Cadastro de perguntas} Cadastro de perguntas}
\includesvg[inkscapelatex=false,width=1\textwidth]{anexos/Imagens_Prototipo/1_cadastro_perguntas.svg}
\fonte{os autores}
\end{figure}
\FloatBarrier
