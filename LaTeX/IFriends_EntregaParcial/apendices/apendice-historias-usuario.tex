\chapter{Histórias de usuário}
\label{historias de usuario}
% ----------------------------------------------------------
A seguinte seção apresentará as histórias de usuário do projeto \gls{ifriends} classificadas por seus respectivos épicos.

%--------------------------------------------------------------------------------
\section{Épico - Gestão de perguntas}
\label{gestão_perguntas}
%--------------------------------------------------------------------------------
%%%%%%%  História - Manter uma pergunta %%%%%%% 
\def\arraystretch{2}
\begin{quadro}[htb]
\centering
\ABNTEXfontereduzida
\caption[História: Manter uma pergunta]{História: Manter uma pergunta}
\resizebox{\linewidth}{!}{
\begin{tabular}{|p{6.5cm}|c|c|c|c|}
\hline
\thead{Descrição} & \thead{Épico} & \thead{Pontuação} & \thead{Tamanho} & \thead{Prioridade}\\
\hline

Como \gls{friend}, eu gostaria de manter uma pergunta na comunidade para retirar uma dúvida & Gestão de Perguntas & 13 & Grande & ALTA \\ \hline

\end{tabular}}\legend{Fonte: Os autores}
\end{quadro}
\FloatBarrier 

Para esta história de usuário foram definidos os seguintes itens como critérios de aceitação:

\begin{itemize}
\item Mostrar ``como fazer uma boa pergunta'';
\item O usuário deve conseguir somente criar e remover uma pergunta da visualização;
\item O usuário deve conseguir fechar o espaço de resposta para a pergunta;
\end{itemize}

%%%%%%% História - Filtrar pergunta %%%%%%% 
\def\arraystretch{2}
\begin{quadro}[htb]
\centering
\ABNTEXfontereduzida
\caption[História: Filtrar perguntas]{História: Filtrar perguntas}
\resizebox{\linewidth}{!}{
\begin{tabular}{|p{6.5cm}|c|c|c|c|}
\hline
\thead{Descrição} & \thead{Épico} & \thead{Pontuação} & \thead{Tamanho} & \thead{Prioridade}\\
\hline

Como \gls{friend}, eu gostaria de filtrar perguntas para que possa encontrar as mais relevantes. & Gestão de Perguntas & 8 & Médio & MÉDIA \\ \hline

\end{tabular}}\legend{Fonte: Os autores}
\end{quadro}
\FloatBarrier 

Para esta história de usuário foram definidos os seguintes itens como critérios de aceitação:

\begin{itemize}
\item Tipos de filtro de perguntas: mais populares, recentes, por assuntos, sem resposta;
\end{itemize}

%%%%%%% História - Adicionar assuntos %%%%%%% 
\def\arraystretch{2}
\begin{quadro}[htb]
\centering
\ABNTEXfontereduzida
\caption[História: Adicionar assuntos]{História: Adicionar assuntos}
\resizebox{\linewidth}{!}{
\begin{tabular}{|p{6.5cm}|c|c|c|c|}
\hline
\thead{Descrição} & \thead{Épico} & \thead{Pontuação} & \thead{Tamanho} & \thead{Prioridade}\\
\hline

Como \gls{friend}, eu gostaria de adicionar assuntos nas minhas perguntas para encontrá-las mais facilmente & Gestão de Perguntas & 3 & Pequeno & ALTA \\ \hline

\end{tabular}}\legend{Fonte: Os autores}
\end{quadro}
\FloatBarrier 

Para esta história de usuário foram definidos os seguintes itens como critérios de aceitação:

\begin{itemize}
\item Devem existir tipos de tags (exemplo: o tipo é informática, que possui várias tags);
\item As tags serão divididas em uma lista pré-definida de acordo com seu tipo e um campo “outras” para tags que não existem.
\end{itemize}

%%%%%%% História - Curtir uma pergunta %%%%%%% 
\def\arraystretch{2}
\begin{quadro}[htb]
\centering
\ABNTEXfontereduzida
\caption[História: Curtir uma pergunta]{História: Curtir uma pergunta}
\resizebox{\linewidth}{!}{
\begin{tabular}{|p{6.5cm}|c|c|c|c|}
\hline
\thead{Descrição} & \thead{Épico} & \thead{Pontuação} & \thead{Tamanho} & \thead{Prioridade}\\
\hline

Como \gls{friend}, eu gostaria de votar em uma pergunta para indicar se ela me foi útil ou não. & Gestão de Perguntas & 2 & Pequeno & ALTA \\ \hline

\end{tabular}}\legend{Fonte: Os autores}
\end{quadro}
\FloatBarrier 

Para esta história de usuário foram definidos os seguintes itens como critérios de aceitação:

\begin{itemize}
\item Um usuário só poderá votar uma única vez;
\item Cada voto equivale a um ponto;
\item Soma dos pontos por pergunta deve ser exibida;
\end{itemize}

%%%%%%% História - Buscar Perguntas %%%%%%% 
\def\arraystretch{2}
\begin{quadro}[htb]
\centering
\ABNTEXfontereduzida
\caption[História: Buscar perguntas]{História: Buscar perguntas}
\resizebox{\linewidth}{!}{
\begin{tabular}{|p{6.5cm}|c|c|c|c|}
\hline
\thead{Descrição} & \thead{Épico} & \thead{Pontuação} & \thead{Tamanho} & \thead{Prioridade}\\
\hline

Como \gls{friend}, eu gostaria de buscar perguntas feitas para que possa consultar uma pergunta específica & Gestão de Perguntas & 2 & Pequeno & MÉDIA \\ \hline

\end{tabular}}\legend{Fonte: Os autores}
\end{quadro}
\FloatBarrier 

Para esta história de usuário foram definidos os seguintes itens como critérios de aceitação:

\begin{itemize}
\item O usuário precisa informar total ou parcialmente o título da pergunta desejada;
\item As perguntas serão exibidas conforme as informações passadas, podendo ser semelhantes parcial ou totalmente;
\end{itemize}

%--------------------------------------------------------------------------------
\section{Épico - Gestão de respostas}
\label{gestão_respostas}
%--------------------------------------------------------------------------------
%%%%%%% História - Manter uma resposta %%%%%%% 
\def\arraystretch{2}
\begin{quadro}[htb]
\centering
\ABNTEXfontereduzida
\caption[História: Manter uma resposta]{História: Manter uma resposta}
\resizebox{\linewidth}{!}{
\begin{tabular}{|p{6.5cm}|c|c|c|c|}
\hline
\thead{Descrição} & \thead{Épico} & \thead{Pontuação} & \thead{Tamanho} & \thead{Prioridade}\\
\hline

Como \gls{friend}, eu gostaria de manter uma resposta para retirar uma dúvida de um colega. & Gestão de Respostas & 5 & Médio & ALTA \\ \hline

\end{tabular}}\legend{Fonte: Os autores}
\end{quadro}
\FloatBarrier 

Para esta história de usuário foram definidos os seguintes itens como critérios de aceitação:

\begin{itemize}
\item As respostas mais curtidas devem ser exibidas antes das demais;
\item O usuário deve conseguir somente criar e deletar uma resposta;
\item Todas as respostas devem ser exibidas sem exceção;
\end{itemize}

%%%%%%% História - Curtir uma resposta %%%%%%% 
\def\arraystretch{2}
\begin{quadro}[htb]
\centering
\ABNTEXfontereduzida
\caption[História: Curtir uma resposta]{História: Curtir uma resposta}
\resizebox{\linewidth}{!}{
\begin{tabular}{|p{6.5cm}|c|c|c|c|}
\hline
\thead{Descrição} & \thead{Épico} & \thead{Pontuação} & \thead{Tamanho} & \thead{Prioridade}\\
\hline

Como \gls{friend}, eu gostaria de curtir uma resposta para indicar se ela me foi útil ou não. & Gestão de Respostas & 1 & Pequeno & ALTA \\ \hline

\end{tabular}}\legend{Fonte: Os autores}
\end{quadro}
\FloatBarrier 

Para esta história de usuário foram definidos os seguintes itens como critérios de aceitação:

\begin{itemize}
\item Um usuário só poderá curtir uma única vez;
\item Cada curtida equivale a um ponto;
\item Soma das curtidas por pergunta deve ser exibida;
\end{itemize}

%--------------------------------------------------------------------------------
\section{Épico - Gestão de eventos}
\label{gestão_eventos}
%--------------------------------------------------------------------------------
%%%%%%% História - Manter evento %%%%%%% 
\def\arraystretch{2}
\begin{quadro}[htb]
\centering
\ABNTEXfontereduzida
\caption[História: Manter evento]{História: Manter evento}
\resizebox{\linewidth}{!}{
\begin{tabular}{|p{6.5cm}|c|c|c|c|}
\hline
\thead{Descrição} & \thead{Épico} & \thead{Pontuação} & \thead{Tamanho} & \thead{Prioridade}\\
\hline

Como \gls{friend}, eu gostaria de manter eventos para ser ajudado e ajudar meus colegas com suas dúvidas sobre um assunto. & Gestão de Eventos & 3 & Pequeno & BAIXA \\ \hline

\end{tabular}}\legend{Fonte: Os autores}
\end{quadro}
\FloatBarrier 

Para esta história de usuário foram definidos os seguintes itens como critérios de aceitação:

\begin{itemize}
\item É necessário uma aba com a listagem de todos os eventos publicados separados em ordem cronológica;
\end{itemize}

%%%%%%% História - Favoritar os eventos %%%%%%% 
\def\arraystretch{2}
\begin{quadro}[htb]
\centering
\ABNTEXfontereduzida
\caption[História: Favoritar evento]{História: Favoritar evento}
\resizebox{\linewidth}{!}{
\begin{tabular}{|p{6.5cm}|c|c|c|c|}
\hline
\thead{Descrição} & \thead{Épico} & \thead{Pontuação} & \thead{Tamanho} & \thead{Prioridade}\\
\hline

Como \gls{friend}, eu gostaria de favoritar um evento para expressar meu interesse em atendê-lo. & Gestão de Eventos & 2 & Pequeno & BAIXA \\ \hline

\end{tabular}}\legend{Fonte: Os autores}
\end{quadro}
\FloatBarrier 

Para esta história de usuário foram definidos os seguintes itens como critérios de aceitação:

\begin{itemize}
\item  Deve ser um contador de número de inscritos no evento;
\end{itemize}

%--------------------------------------------------------------------------------
\section{Épico - Gestão de usuários}
\label{gestão_usuario}
%--------------------------------------------------------------------------------
%%%%%%% História - Autenticação do usuário%%%%%%% 
\def\arraystretch{2}
\begin{quadro}[htb]
\centering
\ABNTEXfontereduzida
\caption[História: Autenticação do usuário]{História: Autenticação do usuário}
\resizebox{\linewidth}{!}{
\begin{tabular}{|p{6.5cm}|c|c|c|c|}
\hline
\thead{Descrição} & \thead{Épico} & \thead{Pontuação} & \thead{Tamanho} & \thead{Prioridade}\\
\hline

Como \gls{friend}, eu gostaria de me autenticar no sistema para realizar ações personalizadas. & Gestão de Usuários & 5 & Médio & ALTA \\ \hline

\end{tabular}}\legend{Fonte: Os autores}
\end{quadro}
\FloatBarrier 

Para esta história de usuário foram definidos os seguintes itens como critérios de aceitação:

\begin{itemize}
\item \textsl{Layout} das telas de \textsl{login} e cadastro finalizado;
\item Verificação do \textsl{Token} da \acs{api} no \textsl{login} via \acs{jwt}.
\end{itemize}

%--------------------------------------------------------------------------------
\section{Épico - Usabilidade}
\label{gestão_usabilidade}
%--------------------------------------------------------------------------------
%%%%%%% História - Internacionalização%%%%%%% 
\def\arraystretch{2}
\begin{quadro}[htb]
\centering
\ABNTEXfontereduzida
\caption[História: Internacionalização]{História: Internacionalização}
\resizebox{\linewidth}{!}{
\begin{tabular}{|p{6.5cm}|c|c|c|c|}
\hline
\thead{Descrição} & \thead{Épico} & \thead{Pontuação} & \thead{Tamanho} & \thead{Prioridade}\\
\hline

Como \gls{friend}, eu gostaria de escolher o idioma da aplicação para visualização em inglês e português. & Usabilidade & 5 & Médio & ALTA \\ \hline

\end{tabular}}\legend{Fonte: Os autores}
\end{quadro}
\FloatBarrier 

Para esta história de usuário foram definidos os seguintes itens como critérios de aceitação:

\begin{itemize}
\item  Textos disponíveis em ambos os idiomas a qualquer momento na aplicação.
\end{itemize}
