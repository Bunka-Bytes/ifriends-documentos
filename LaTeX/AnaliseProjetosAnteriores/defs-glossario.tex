%%%%%%%%%%%%%%%%%%%%%%%%%%%%%%%%%%%%%%%%%%%%%%%%%%%%%%%%
% Normalmente somente as palavras referenciadas são impressas no glossário, portanto é necessário referenciar utilizando \gls{identificação}  
%%%%%%%%%%%%%%%%%%%%%%%%%%%%%%%%%%%%%%%%%%%%%%%%%%%%%%%%
\newglossaryentry{front-end} {
    name= Front-end,
    plural= {front-end},
    description={Refere-se a parte visual e gráfica da interface de um sistema, elaborado por meio de outras linguagens e tecnologias.}
}
%%%%%%%%%%%%%%%%%%%%%%%%%%%%%%%%%%%%%%%%%%%%%%%%%%%%%%%%
\newglossaryentry{back-end} {
    name= Back-end,
    plural= {back-end},
    description={Refere-se a parte que está por trás da aplicação, responsável pela manipulação de dados voltada para o funcionamento interno de um sistema}
}
%%%%%%%%%%%%%%%%%%%%%%%%%%%%%%%%%%%%%%%%%%%%%%%%%%%%%%%%
\newglossaryentry{ifriends} {
    name= IFriends,
    plural= {IFriends},
    description={Nome dado ao projeto de sistemas desenvolvido, cujo significado se dá num trocadilho na junção das palavras friends (amigos, em inglês) e IF (Instituito Federal).}
}
%%%%%%%%%%%%%%%%%%%%%%%%%%%%%%%%%%%%%%%%%%%%%%%%%%%%%%%%
\newglossaryentry{gamificação} {
    name= gamificação,
    plural= {gamificações},
    description={Aplicação das estratégias dos jogos nas atividades do dia a dia, com o objetivo de aumentar o engajamento dos participantes. Se baseia no game thinking, que integra a gamificação com outros saberes do meio corporativo e do design. }
}
%%%%%%%%%%%%%%%%%%%%%%%%%%%%%%%%%%%%%%%%%%%%%%%%%%%%%%%%
\newglossaryentry{statsvn} {
    name= StatSVN,
    plural= {StatSVN},
    description={Ferramenta que funciona a partir de arquivos de log extraídos do
    repositório do SVN, fornecendo gráficos e dados estatísticos a
    partir do cruzamento dessas informações.}
}
%%%%%%%%%%%%%%%%%%%%%%%%%%%%%%%%%%%%%%%%%%%%%%%%%%%%%%%%
\newglossaryentry{laravel}{
    name= Laravel,
    plural= {Laravel},
    description= {Framework PHP livre e open-source criado por Taylor B. Otwell para o desenvolvimento de sistemas web que utilizam o padrão MVC.}
}   
%%%%%%%%%%%%%%%%%%%%%%%%%%%%%%%%%%%%%%%%%%%%%%%%%%%%%%%%