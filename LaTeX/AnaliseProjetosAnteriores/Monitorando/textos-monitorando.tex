\chapter[Monitorando]{Monitorando - Equipe Loading}
\label{monitorando}
O motivo pelo qual a equipe Bunka Bytes realizou análise do projeto Monitorando, se deu pelo seu objetivo em resolver a falta de informações sobre as monitorias e plantões de dúvida. O projeto desenvolvido pela equipe Loading do curso Técnico Integrado em Informática como \acs{tcc}, pode ser encontrado na pasta A2021-PDS413.

Após os estudos de campo realizados pela equipe, constatou-se, falta de organização das salas e laboratórios, falta de comunicação, e computadores insuficientes para uso dos alunos. De tal forma, o sistema agrega atuando no gerenciamento, divulgação e organização das mesmas, assim tendo controle sobre quantos alunos frequentariam a monitoria, facilitando o acesso a essas informações.

Para desenvolver esta aplicação Web, foi utilizado \acs{html}, \acs{css} e JavaScript para o \textsl{\gls{front-end}}, e PHP junto com o \gls{laravel} para o \textsl{\gls{back-end}}. Houve ainda a aplicação da arquitetura no padrão \acs{mvc} e um banco de dados \textsl{MySQL}, utilizando a metodologia ágil \textsl{Scrum}.

\section[Aprendizados]{Aprendizados}
A equipe fez uso do \textsl{Scrum}, com algumas adaptações de acordo com as necessidades discutidas, de tal forma, vimos a importância da comunicação e organização entre os membros da equipe. Para podermos tornar viável e eficiente o desenvolvimento do sistema, é necessário uma comunicação constante sobre o andamento das etapas e possíveis dificuldades, problemas encontrado pelos integrantes, assim, temos um bom planejamento inicial e passível de mudanças.

O projeto Monitorando possui ideia similar a equipe Bunka Bytes, com o gerenciamento de monitorias e um fórum de dúvidas, assim, podemos entender a organização usada, reaproveitar e aprimorar ideias aplicadas no sistema.

\section{Precauções e sugestões de melhoria}
Por conta da similaridade presente entre o projeto Monitorando e o projeto que esta em desenvolvimento pela equipe Bunka Bytes, entende-se que alguns cuidados devem ser tomados, como, garantir que os usuários (monitores e alunos) pertençam ao \acs{ifsp}, exibir ao aluno as informações das monitorias e fornecer usabilidade intuitiva ao usuário.

Por isso, a equipe Loading poderia melhorar a aplicação Monitorando em características de usabilidade, pois algumas informações estão colocadas na página de maneira dispersa e desorganizada. Além disso, também seria interessante melhorar a responsividade que quebra em determinados momentos, apresentar ao usuário o nome do monitor responsável pela monitoria e não apenas o prontuário, fornecer ao usuário um botão “cancelar participação” que cancele a presença do mesmo e também remover da visualização as monitorias que ocorreram anteriormente e que por este motivo não estão mais disponíveis.