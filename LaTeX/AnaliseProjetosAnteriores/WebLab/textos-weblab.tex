\chapter[Web Lab]{Web Lab - Equipe The Coders}
\label{weblab}

O projeto Web Lab foi realizado pela equipe The Coders, composta por alunos do curso Técnico em Informática Integrado ao Ensino Médio, na disciplina de Prática de Desenvolvimento de Sistemas. O acesso pode ser feito através da pasta A2020-PDS413 do repositório \textsl{Subversion}.

Segundo a proposta do Web Lab, é possível notar uma semelhança, após à análise, em relação ao projeto \gls{ifriends}. Visto que ambas tem como missão fornecer um espaço lúdico para os alunos, principalmente aqueles que frequentam o \acs{ifsp}, adquirirem conhecimento sobre aspectos escolares. Portanto, isso significa que seu conteúdo é benéfico, de certa forma, para a construção do sistema que a equipe Bunka Bytes idealizou.

Além disso, outro fator que instigou a analise deste projeto, foi seu desempenho em relação à produtividade, de modo geral. Visto que a equipe conseguiu realizar uma ótima organização entre os membros, destinando tarefas de acordo com a facilidade de cada um, por exemplo. O que resultou em ganho de tempo na maior parte do  desenvolvimento do Web Lab.

\section[Aprendizados]{Aprendizados}

Um dos aprendizados possíveis, a princípio, seria o conhecimento sobre a existência de ferramentas para análise de produtividade e performance do projeto. Algumas delas, das quais a equipe The Coders se submeteu a usar, serão pesquisadas e, conforme o andamento do projeto \gls{ifriends}, aplicadas para visualização, de maneira ampla, do desenvolvimento já realizado. Por exemplo, a ferramenta que o The Coders utilizou para analisar a estatística de linhas de código do Web Lab, foi o \gls{statsvn}. Ademais, é possível aprender a forma como trabalharam em equipe, para ser reproduzida e adaptada de acordo com a rotina dos integrantes que fazem parte do Bunka Bytes.  

\section[Precauções]{Precauções e sugestões de melhoria}

A \gls{gamificação} do sistema que está presente em ambos projetos, e pode, eventualmente, ocasionar riscos e criar uma margem de erro grande para o desenvolvimento dos códigos fontes. No caso do Web Lab, é possível observar, pelo documento de visão, problemas em relação a algumas ideias a mais que a equipe planejou implementar no sistema, e que acabaram não funcionando muito bem por falta de tempo e conhecimento, resultando na desistência delas. 

Ao fim da análise do projeto, é possível concluir que não foram encontradas tantas correções e melhorias, apenas algumas partes do documento sem virgula; figuras com baixa qualidade, dificultando a visualização; a estética das telas do sistema poderiam ser mais trabalhadas. Entretanto, de modo geral, está bem escrito e planejado.