% ------------------------------------------------------
% Introdução (exemplo de capítulo sem numeração, mas presente no Sumário)
% ------------------------------------------------------
\chapter{Introdução}
Este documento tem como finalidade a análise de dois projetos realizados nos anos anteriores da disciplina de Prática de Desenvolvimento de Sistemas (PDS).

Pensando na proposta inicial e no objetivo do projeto \gls{ifriends}, pensado pelos integrantes da equipe Bunka Bytes, escolheu-se investigar os projetos Monitorando (\acs{pds} 2021) e Web Lab (\acs{pds} 2020), pois seus objetivos educativos são semelhantes aos que o \gls{ifriends} se propõe, principalmente o Monitorando, visto que algumas de suas funcionalidades são bastante próximas ao projeto da equipe Bunka Bytes. Logo, tais motivos serão explicitados ao longo dos textos representativos para cada projeto escolhido.

Espera-se que, com essas análises, a equipe possa aprender mais sobre o funcionamento da disciplina e ter uma melhor visão sobre como os projetos são desenvolvidos na prática ao longo do ano.

%----
\chapter[Monitorando]{Monitorando - Equipe Loading}
\label{monitorando}
O motivo pelo qual a equipe Bunka Bytes realizou análise do projeto Monitorando, se deu pelo seu objetivo em resolver a falta de informações sobre as monitorias e plantões de dúvida. O projeto desenvolvido pela equipe Loading do curso Técnico Integrado em Informática como \acs{tcc}, pode ser encontrado na pasta A2021-PDS413.

Após os estudos de campo realizados pela equipe, constatou-se, falta de organização das salas e laboratórios, falta de comunicação, e computadores insuficientes para uso dos alunos. De tal forma, o sistema agrega atuando no gerenciamento, divulgação e organização das mesmas, assim tendo controle sobre quantos alunos frequentariam a monitoria, facilitando o acesso a essas informações.

Para desenvolver esta aplicação Web, foi utilizado \acs{html}, \acs{css} e JavaScript para o \textsl{\gls{front-end}}, e PHP junto com o \gls{laravel} para o \textsl{\gls{back-end}}. Houve ainda a aplicação da arquitetura no padrão \acs{mvc} e um banco de dados \textsl{MySQL}, utilizando a metodologia ágil \textsl{Scrum}.

\section[Aprendizados]{Aprendizados}
A equipe fez uso do \textsl{Scrum}, com algumas adaptações de acordo com as necessidades discutidas, de tal forma, vimos a importância da comunicação e organização entre os membros da equipe. Para podermos tornar viável e eficiente o desenvolvimento do sistema, é necessário uma comunicação constante sobre o andamento das etapas e possíveis dificuldades, problemas encontrado pelos integrantes, assim, temos um bom planejamento inicial e passível de mudanças.

O projeto Monitorando possui ideia similar a equipe Bunka Bytes, com o gerenciamento de monitorias e um fórum de dúvidas, assim, podemos entender a organização usada, reaproveitar e aprimorar ideias aplicadas no sistema.

\section{Precauções e sugestões de melhoria}
Por conta da similaridade presente entre o projeto Monitorando e o projeto que esta em desenvolvimento pela equipe Bunka Bytes, entende-se que alguns cuidados devem ser tomados, como, garantir que os usuários (monitores e alunos) pertençam ao \acs{ifsp}, exibir ao aluno as informações das monitorias e fornecer usabilidade intuitiva ao usuário.

Por isso, a equipe Loading poderia melhorar a aplicação Monitorando em características de usabilidade, pois algumas informações estão colocadas na página de maneira dispersa e desorganizada. Além disso, também seria interessante melhorar a responsividade que quebra em determinados momentos, apresentar ao usuário o nome do monitor responsável pela monitoria e não apenas o prontuário, fornecer ao usuário um botão “cancelar participação” que cancele a presença do mesmo e também remover da visualização as monitorias que ocorreram anteriormente e que por este motivo não estão mais disponíveis.
\chapter[Web Lab]{Web Lab - Equipe The Coders}
\label{weblab}

O projeto Web Lab foi realizado pela equipe The Coders, composta por alunos do curso Técnico em Informática Integrado ao Ensino Médio, na disciplina de Prática de Desenvolvimento de Sistemas. O acesso pode ser feito através da pasta A2020-PDS413 do repositório \textsl{Subversion}.

Segundo a proposta do Web Lab, é possível notar uma semelhança, após à análise, em relação ao projeto \gls{ifriends}. Visto que ambas tem como missão fornecer um espaço lúdico para os alunos, principalmente aqueles que frequentam o \acs{ifsp}, adquirirem conhecimento sobre aspectos escolares. Portanto, isso significa que seu conteúdo é benéfico, de certa forma, para a construção do sistema que a equipe Bunka Bytes idealizou.

Além disso, outro fator que instigou a analise deste projeto, foi seu desempenho em relação à produtividade, de modo geral. Visto que a equipe conseguiu realizar uma ótima organização entre os membros, destinando tarefas de acordo com a facilidade de cada um, por exemplo. O que resultou em ganho de tempo na maior parte do  desenvolvimento do Web Lab.

\section[Aprendizados]{Aprendizados}

Um dos aprendizados possíveis, a princípio, seria o conhecimento sobre a existência de ferramentas para análise de produtividade e performance do projeto. Algumas delas, das quais a equipe The Coders se submeteu a usar, serão pesquisadas e, conforme o andamento do projeto \gls{ifriends}, aplicadas para visualização, de maneira ampla, do desenvolvimento já realizado. Por exemplo, a ferramenta que o The Coders utilizou para analisar a estatística de linhas de código do Web Lab, foi o \gls{statsvn}. Ademais, é possível aprender a forma como trabalharam em equipe, para ser reproduzida e adaptada de acordo com a rotina dos integrantes que fazem parte do Bunka Bytes.  

\section[Precauções]{Precauções e sugestões de melhoria}

A \gls{gamificação} do sistema que está presente em ambos projetos, e pode, eventualmente, ocasionar riscos e criar uma margem de erro grande para o desenvolvimento dos códigos fontes. No caso do Web Lab, é possível observar, pelo documento de visão, problemas em relação a algumas ideias a mais que a equipe planejou implementar no sistema, e que acabaram não funcionando muito bem por falta de tempo e conhecimento, resultando na desistência delas. 

Ao fim da análise do projeto, é possível concluir que não foram encontradas tantas correções e melhorias, apenas algumas partes do documento sem virgula; figuras com baixa qualidade, dificultando a visualização; a estética das telas do sistema poderiam ser mais trabalhadas. Entretanto, de modo geral, está bem escrito e planejado.
%----

\chapter{Considerações finais}
Após a leitura e análise dos dois projetos, a equipe acredita que poderá aplicar seus aprendizados de maneira a precaver possíveis problemas  durante o desenvolvimento do \gls{ifriends}, além de ter conseguido abstrair algumas ideias interessantes que poderão compreender o projeto futuramente: como as ideias de emblemas presentes na parte de gamificação do projeto Web Lab, por exemplo. Por outro lado, também será possível tomar como base algumas das estratégias de organização das equipes, assim como está sendo feito ao estudar documentos de outros projetos semelhantes, como o Plannic, da equipe Winx (Projeto Integrado 2020), que também possuía fins estudantis, visto que pretendia ajudar alunos na organização de seus estudos. 

Por fim, acredita-se que a Equipe Loading tenha cumprido grande partes dos seus objetivos iniciais, visto que sua aplicacao disponível em  \url{https://loading-monitorando.herokuapp.com}, fornece o serviço de monitoriais esperado - ainda que a aba "Monitorias" esteja gerando um erro no servidor, foi possível inscrever-se nelas através da busca. Para a Equipe The Coders, poderia ser feita uma melhor análise caso a aplicação ainda estivesse disponível, mas infelizmente, não foi possível testá-la desta maneira para chegar a uma conclusão semelhante. 