\chapter{Publicações no blog da equipe}
\label{postsBlog}
\section{1\textordfeminine \, Semana - 14/03 à 20/03}

Primeiramente, bem vindos à primeira postagem da equipe! O principal objetivo deste blog é dar aos leitores a possibilidade de acompanhar nosso "diário semanal" de desenvolvimento dentro da disciplina de Prática de Desenvolvimento de Sistemas (ou \acs{pds} para os mais íntimos).

Mas antes de dar continuidade ao texto, precisamos apresentá-los a equipe Bunka Bytes, cujo nome é parte inspirado no tema do trabalho anterior (em \acs{tds}), já que este possuía um foco cultural e "Bunka" em japonês significa "cultura"; e outra parte se deve a uma referência/trocadilho com a área de informática, pois falando, é quase como se estivéssemos dizendo "Boom Kabytes". 

Integrantes da equipe:

\begin{itemize}
    \item Anai Rojas
    \item Jamilli Gioielli
    \item José Roberto
    \item Julia Romualdo
    \item Kaiky Matsumoto
\end{itemize}

Dos integrantes, escolhemos a Jamilli como nossa representação de gerente, devido a seus conhecimentos em organização, metodologias e ferramentas de gerenciamento e por sua facilidade comunicativa. 

Tendo isso em vista, nesta primeira semana da disciplina, realizamos três reuniões pela plataforma \gls{discord}, nas quais conhecemos melhor nossos colegas de equipe, fizemos um Contrato Social e criamos um \textsl{Brainstorming} utilizando conceitos de \textsl{Design Thinking} na plataforma \gls{figma}, que ajudou numa melhor visualização das ideias centrais para o projeto.  Também tivemos acesso aos trabalhos anteriores e consultamos dois trabalhos que eram mais semelhantes ao nosso tema principal. 

A partir do \textsl{Brainstorm}, a ideia que mais se consolidou foi a proposta de criação de uma comunidade de \acs{Q/A} e mentorias como forma de apoio aos estudantes do \acs{ifsp}. Pensamos em muitas outras, mas esta acabou sendo nossa favorita, com base nos requisitos da disciplina de \acs{pds}. Portanto, nosso objetivo para a segunda semana é conversar com os professores sobre a ideia e a desenvolver melhor com base nas nossas discussões e dúvidas a serem sanadas.

Por último, ao final dessa semana, criamos um \textsl{e-mail} conjunto para a criação deste \textsl{blog} e o canal no \gls{youtube}, também conseguimos acessar o Subversion, criar um canal no \gls{discord}, uma página de gerenciamento no \gls{notion} e uma logo para a equipe.


 \textbf{Por: Jamilli Gioielli e Julia Romualdo}

\section{2\textordfeminine \, Semana - 21/03 à 27/03}

Estamos de volta, leitor!

No início dessa semana, voltamos às atividades presenciais e tivemos um primeiro contato com a disciplina neste formato. A partir daí, separamos as 5 ideias que mais se destacaram do nosso \textsl{Brainstorm} da semana anterior e as compartilhamos com os professores. Recebemos algumas sugestões sobre a ideia de comunidade e nos foi recomendado documentar as ideias para enviar no ambiente do \gls{moodle} da disciplina. 

Tendo isso em vista, precisamos nos reunir durante os próximos dias para passar a construção dos nossos pensamentos em formato textual e explicativo. Entretanto, enfrentamos algumas dificuldades nesse sentido, estando a maioria delas em torno do curto tempo que temos entre trabalho e escola para fazer as reuniões. Tentamos conversar na hora anterior às aulas, mas percebemos que faltava apenas documentar melhor o que havíamos conversado. Não foi possível realizar isto na escola devido aos problemas de conexão de rede, então optamos por fazermos aos poucos durante a semana e nos reunirmos depois da aula, via \gls{discord}, para revisar o conteúdo. De todo modo, conseguimos fazer a entrega das duas tarefas da semana no \gls{moodle}.

Nossos objetivos para próxima semana são: definir a proposta para o projeto com base no \textsl{feedback} dos professores e planejar melhor nossos dias e horários para reuniões.

\textbf{Por:  Jamilli Gioielli} 

\section{3\textordfeminine \, Semana - 28/03 à 03/04}
Estamos de volta, Bunkers!

Nesta semana, trabalhamos na viabilidade da nossa proposta. Em aula, discutimos com os professores as funcionalidades que a comunidade pode ter, encontramos uma solução para validar os alunos cadastrados nela, enviando um \textsl{e-mail} de confirmação apenas ao \textsl{e-mail} institucional, que é de posse dos alunos, garantindo assim que os usuários sejam apenas alunos do instituto e por fim, falamos também sobre uma segunda proposta da equipe, que os professores gostaram bastante e deram forte impulso em desenvolve-la por atingir um grupo maior de usuários e ser algo que todos precisam cotidianamente, que é um sistema baseado em controlar gastos e auxiliar na organização financeira.

Entretanto, optamos por continuar com a comunidade, a qual batizamos com o nome: \gls{ifriends}, principalmente porque é mais próximo da nossa realidade como alunos do instituto e por não termos experiência em desenvolvimento de aplicações \textsl{mobile}, o modelo que enxergamos ser o ideal para a criação da segunda proposta. E como análise prática da viabilidade do \gls{ifriends}, elaboramos um formulário, via \gls{googleforms}, para no início da próxima semana, enviarmos aos nossos colegas e alunos do instituto com o intuito de investigar se eles fariam uso da comunidade.

Falando agora sobre a nossa organização como equipe, ainda estamos nos adaptando com nossos horários entre estudos, trabalho e locomoção, então nosso foco será realizar reuniões rápidas e objetivas principalmente nos horários disponíveis antes das aulas começarem.

Resumo das atividades de cada membro da equipe:

\begin{itemize}
    \item Anai - Elaborou formulário da pesquisa de viabilidade
    \item Jamilli - Organizou as atividades a serem realizadas pela equipe 
    \item José, Julia e Kaiky - Trabalharam em melhorias de layout e postagens do blog
\end{itemize}

Para a próxima semana a equipe tem como objetivo estudar as tecnologias e linguagens a serem utilizadas no desenvolvimento do projeto, estruturar um \textsl{backlog} para gerenciamento eficiente das atividades, trabalhar na identidade visual e design da marca \gls{ifriends}.    

\textbf{Por: Julia Romualdo}

\section{4\textordfeminine \, Semana - 04/04 à 10/04}
Estamos de volta, Bunkers!

No inicio desta semana, realizamos a pesquisa de viabilidade da nossa proposta inicial da comunidade, para isto, elaboramos um formulário via \gls{googleforms}, que ficou disponível para receber respostas de segunda-feira (04/04) à sexta-feira (08/04) e todos os integrantes da equipe ficaram responsáveis por enviar o endereço de compartilhamento - \textsl{link} - nos grupos de \gls{WhatsApp} para os alunos da instituição - público-alvo da nossa proposta - respondessem a dez perguntas e compartilharem algumas experiências como alunos do \acs{ifsp} que ajudassem a equipe a compreender se a proposta era ou não viável.

Ainda nesta semana, recebemos dos professores as orientações para apresentação da proposta inicial, juntamente com a entrega da documentação e estudo de dois projetos anteriores. Para realização desta tarefa, durante a semana a equipe se organizou da seguinte forma:
\begin{itemize}
    \item Anai -  Criou \textsl{branchmarketing} para a aplicação e estudou o projeto WebLab.
    \item Jamilli - Organizou as atividades a serem realizadas pela equipe e estruturou as documentações.
    \item José - Estudou as tecnologias a serem utilizadas no projeto e também do projeto Monitorando.
    \item Julia - Compilou dados da pesquisa de viabilidade de proposta e estudou o  projeto Monitorando.
    \item Kaiky -  Estudou as tecnologias a serem utilizadas no projeto e também do projeto WebLab.
\end{itemize}
Para a próxima semana a equipe tem como objetivo apresentar a proposta e estudar os \textsl{feedbacks} fornecidos pelos professores e colegas durante a apresentação.

\textbf{Por: Julia Romualdo} 

\section{5\textordfeminine \, Semana - 11/04 à 17/04}
 Estamos de volta, Bunkers!

Iniciamos esta semana com a apresentação da proposta inicial da comunidade \gls{ifriends} aos nossos professores e colegas de classe, dos quais recebemos orientações e \textsl{feedbacks}. Neste mesmo cenário, também participamos da apresentação das outras equipes e compartilhamos nossos \textsl{feedbacks} aos mesmos.

Após a apresentação, a equipe se reuniu na biblioteca do \acs{ifsp} para realizar uma retrospectiva, com o objetivo de avaliar o funcionamento da equipe durante a intensa semana de trabalhos que tivemos para realizar a entrega da proposta inicial. Aproveitamos este momento, para decidir os pontos a serem melhorados na apresentação para realizarmos a gravação e entrega do vídeo da proposta e também ajustamos alguns pontos que faltavam ser encaixados sobre as tecnologias que serão utilizadas no projeto.

Além disso, aproveitamos o feriado para revisar alguns conceitos que tínhamos dúvidas e procurar possíveis melhorias para nossa apresentação e documentação. Uma das coisas que percebemos era que ainda precisávamos validar o arquivo equipe.yaml no yamllint, já que ele não estava sendo refletido na página de Blogs de Trabalhos, por isso aproveitamos para ajustá-lo de acordo com os apontamentos que foram dados pelo validador.

De todo modo, a equipe tem como meta para a próxima semana a atualização das fontes do projeto de acordo com o \textsl{feedback} que será dado pelos demais colegas e pelos professores após a apresentação da proposta. Além disso, pretendemos postar o vídeo da proposta - já com as melhorias - e também entregar nossas avaliações sobre as demais equipes.

Por isso, contamos com os feriados para adiantar o máximo de atividades possíveis e já pensar em alguns itens de \textsl{backlog} para que possamos iniciar a primeira \textsl{sprint} com o cronograma do projeto já bem definido. Pensando nisso também, a equipe não se dividiu como na semana anterior para a execução das tarefas, já que neste primeiro feriado focaremos juntos em tratar das melhorias, estudar as tecnologias propostas e trabalhar no planejamento do projeto.

\textbf{Por: Julia Romualdo e Jamilli Gioielli}

\section{6\textordfeminine \, Semana - 18/04 à 24/04}
Estamos de volta, Bunkers!

Nesta semana, iniciamos a aula de segunda-feira de forma bem produtiva, pois conversamos bastante com os orientadores a respeito das próximas \textsl{sprints} a serem planejadas e executadas pela equipe. Assim partimos para a criação do \textsl{backlog} do produto com o uso das histórias de usuário, depois definimos as tarefas da semana já nos planejando para os três épicos a serem elaborados para a entrega da Prova de Conceito, sendo eles: Gestão de Perguntas, Gestão de respostas e Gestão de Eventos.

Além disso, aproveitamos o feriado de Tiradentes para nos reunirmos através da plataforma \gls{discord}, com o objetivo de terminar a definição das histórias de usuário, neste momento muitos pontos sutis sobre a aplicação, que poderiam se tornar inimigos da equipe futuramente, foram levantados e anotados para discutir com os professores. Também votamos por meio do \textsl{Planning Poker} - descrito pela metodologia \textsl{Scrum} - para entendermos sobre uma estimativa de esforço para que cada história seja concluída. Junto a isso, também podemos elencar alguns requisitos não funcionais e regras de negócio.

Desta forma, as tarefas realizadas pela equipe durante esta semana foram organizadas da seguinte forma:

\begin{itemize}
    \item Anai e Jamilli - Iniciar a prototipagem das telas.
    \item José e Kaiky - Iniciar a modelagem de dados.
    \item Julia - Iniciar os ajustes na documentação.
\end{itemize}

Para a próxima semana, além de dar início as tarefas da \textsl{sprint}, a equipe tem o objetivo de terminar as tarefas de planejamento e apresenta-las aos orientadores para que possamos esclarecer dúvidas e saber os pontos a serem melhorados.

\textbf{Por: Julia Romualdo}

\section{7\textordfeminine \, Semana - 25/04 à 01/05}
Estamos de volta, Bunkers!

Iniciamos esta semana validando com os professores as atividades que a equipe esteve realizando desde a semana passada, fomos orientados quanto a melhor organização dos tópicos do documento, a remover algumas histórias dos épicos para a \acs{POC}, - pois alguns pontos estavam fugindo da ideia da \acs{POC}, que é provar que o conceito principal, ou seja, o fluxo principal da aplicação está funcionando -, a realizar a entrega do épico de Gestão de Eventos, na \acs{POC}, apenas se sobrar tempo, ajustar o diagrama de entidade relacionamento e o diagrama de casos de uso. Aproveitamos este momento ainda em aula para iniciar as configurações do \gls{gource}, pensando em futuras entregas da disciplina.

Devido ao final de bimestre, a equipe esteve ocupada com as atividades de outras disciplinas e não conversou muito durante esta semana, mas os componentes continuaram no desenvolvimento - visando o termino - das atividades propostas na semana passada, organizadas da seguinte forma:
\begin{itemize}
    \item Anai - Terminar a prototipagem das telas.
    \item José e Kaiky - Ajustar a modelagem de dados.
    \item Jamilli - Terminar a prototipagem das telas e ajustar os tópicos de gerenciamento e metodologias da documentação. 
    \item Julia - Adicionar os apêndices na documentação e pesquisar sobre o \gls{gource}.
\end{itemize}
\noindent Para a próxima semana a equipe tem como objetivo desenvolver a Prova de Conceito e a documentação relacionada a mesma.

\textbf{Por: Julia Romualdo}

\section{8\textordfeminine \, Semana - 02/05 à 08/05}
Estamos de volta, Bunkers!

Esta semana devido a problemas na infraestrutura do encanamento do campus \acs{ifsp} não tivemos aulas de maneira presencial e poucos professores ministraram no formato \acs{ead}, aproveitamos então o plantão de segunda-feira com os orientadores para alinharmos o fluxo de usuário e o protótipo para a \acs{POC}. No decorrer da semana utilizamos os horários que seriam destinados às aulas para realizarmos as tarefas de desenvolvimento da \acs{POC} (desenvolvimento, documentação e apresentação), para isto a equipe ficou organizada da seguinte maneira: 
\begin{itemize}
    \item Anai - Realizar ajustes finais no prototipo e ajustar a documentação.
    \item José e Jamilli - Configurar o ambiente e desenvolver o \gls{front-end}
    \item Julia - Realizar ajustes na documentação e gerar vídeo do \gls{gource}.
    \item Kaiky - Configurar o ambiente e desenvolver \gls{back-end} e Banco de Dados.
\end{itemize}
\noindent Para a próxima semana a equipe tem como objetivo apresentar a Prova de Conceito e realizar reunião retrospectiva sobre o desenvolvimento da \gls{POC}.

\textbf{Por: Julia Romualdo}

\section{9\textordfeminine \, Semana - 09/05 à 15/05}
Estamos de volta, Bunkers!

Iniciamos esta semana com a apresentação da Prova de Conceito para a turma e orientadores, e devemos dizer que nosso maior inimigo para esta entrega com certeza foi o tempo, pois deixamos de cumprir alguns requisitos da bíblia do Ivan porque nos faltou tempo hábil para concluir alguns tópicos ali estipulados. Ainda assim, acreditamos que conseguimos demonstrar que o fluxo principal da nossa aplicação estava funcionando. 

Por outro lado, pudemos cumprir grande parte dos requisitos necessários para as entregas do primeiro bimestre, faltando apenas nossa planilha de avaliação da equipe, que foi postada no \gls{svn} nessa semana, um pouco tarde devido a um imprevisto ocorrido com o sistema (que ficou fora do ar durante algumas horas). 

A equipe tinha como um dos objetivos para essa semana a realização da reunião retrospectiva referente ao desenvolvimento da \gls{POC}, porém devido a alta demanda que as outras disciplinas exigiram durante a semana - provas, apresentações e outras atividades -, não encontramos um bom momento para nos reunirmos, ficando isso como uma meta para a próxima semana, juntamente com o planejamento para as próximas etapas do desenvolvimento do \gls{ifriends}. Além disso, precisamos atualizar nosso canal no \gls{youtube} com os vídeos para o segundo bimestre, visto que não conseguimos gravá-los e editá-los até o momento. 

\textbf{Por: Julia Romualdo e Jamilli Gioielli}

\section{10\textordfeminine \, Semana - 16/05 à 22/05}
Estamos de volta, Bunkers!

Podendo respirar um pouco mais após a intensa semana de entrega de atividades que tivemos durante a semana passada, iniciamos os trabalhos novamente com a reunião de retrospectiva referente a entrega da \gls{POC}, com isso chegamos as seguintes conclusões: do que foi bom, consideramos a entrega da \acs{api} completa;  para o que foi ruim, por outro lado, consideramos a falta de tempo, pois por mais que soubéssemos como desenvolver um requisito, como a internacionalização ou o vídeo de demonstração, não sobrou tempo para fazer; e por último, outro problema, enfrentado não apenas por nossa equipe mas de maneira geral na turma, foi a incerteza e a definição errada que construímos sobre a \gls{POC} a partir de experiências anteriores, por isso percebemos que não conseguimos definir o que seria extremamente essencial para essa entrega e pode ser que tenhamos focado mais em coisas adicionais do que no principal. Assim, como plano de ação escolhemos que precisamos definir melhor o que é simples e objetivo para nossas entregas, para evitar que tenhamos confusões desnecessárias que acabem dificultando a entregarmos o que era necessário.

Os professores também realizaram alguns apontamentos a partir das nossas entregas relativas do primeiro bimestres, para que possamos realizar os alinhamentos necessários, tais como: representar o \gls{heroku} na arquitetura, postar os vídeos da apresentação da \gls{POC} no \gls{youtube}, realizar \textit{commits} com mais frequência e enviar os arquivos do \LaTeX que estamos usando no Overleaf.

Além disso, os professores separaram uma parte da aula para assistirmos a alguns projetos dos alunos da turma 231 em \acs{pji}, e permitiu que fizemos apontamentos e sugestões para os colegas sobre suas ideias.

Tendo realizado esses dois momentos de alinhamento a equipe documentou tudo, organizou as tarefas e ajustes que deveriam ser realizados durante a semana e partimos para os ajustes. Conseguimos enviar e ajustar o que faltava para o primeiro bimestre, como a questão do \gls{gource} e ainda incluímos alguns requisitos que faltavam na aplicação e na \gls{api} (como a inclusão do \textit{Swagger UI} e criptografia da autenticação). Como a próxima semana será de conselhos de classe, a equipe espera poder trabalhar mais no projeto e dar continuidade no desenvolvimento do mesmo, além de reavaliarmos o que for necessário de acordo com os demais \textit{feedbacks} dos professores (que ainda serão feitos sobre a \acs{POC}).

\textbf{Por: Julia Romualdo e Jamilli Gioielli}