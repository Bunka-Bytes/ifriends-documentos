% Para facilitar a manutenção é sempre melhore criar um arquivo por capitulo, para exemplo isso não é necessário 

%---------------------------------------------------------------------------------------

%---------------------------------------------------------------------------------------


\input{06-textos-materias-metodos}

%---------------------------------------------------------------------------------------



%--------------------------------------------------------------------------------
\chapter{Desenvolvimento}
  Baseando-se na conceituação de engenharia de \textsl{software} dada por \citeonline{SOMMERVILLE:2019}, neste capítulo é descrito o desenvolvimento do sistema, cujas etapas estão apoiadas em técnicas que vão desde sua especificação até sua evolução. Neste sentido, \citeonline{SOMMERVILLE:2019} destaca quatro etapas como fundamentais para os processos de \textsl{software}, sendo elas sequenciadas em: especificação, desenvolvimento, validação e evolução. As etapas correspondentes a este capítulo consistem na definição, junto as suas restrições, e na projeção do sistema para ser programado - isto é, a especificação e o desenvolvimento.
  
  Desse modo, as especificações descritas no capítulo estão separadas em: análise de requisitos, regras do negócio, modelagem do sistema e prototipação.

%------------------------------------------------------


%------------------------------------------------------
\section{Análise de Requisitos}
Segundo \citeonline{machado2018analise}, ao definir as características de um requisito, é preciso salientar que não são dependentes da tecnologia empregada, visto que suas especificações estão contidas no campo do cumprimento das necessidades do usuário. Dessa forma,  \citeonline{machado2018analise} define os requisitos como ``objetivos ou restrições estabelecidas por clientes e usuários do sistema que definem suas diversas propriedades''.

Assim, tanto \citeonline{machado2018analise} como \citeonline{SOMMERVILLE:2019} concordam que a fase de definição de requisitos, a chamada engenharia de requisitos, é essencial para a tomada de decisões sobre os passos para adquirir ou desenvolver o sistema. Por outro lado, Sommerville ainda acrescenta sobre a necessidade de mudança durante o desenvolvimento:

\begin{citacao}
Naturalmente, são feitas mudanças subsequentes nos requisitos de usuário, que podem ser ampliados para requisitos de sistema mais detalhados. Às vezes, pode-se utilizar uma abordagem ágil para elicitar simultaneamente os requisitos à medida que o sistema é desenvolvido, a fim de acrescentar detalhes e refinar os requisitos de usuário \cite{SOMMERVILLE:2019}.
\end{citacao}

Tendo tais definições em vista, as próximas seções visam apresentar os requisitos funcionais e não funcionais, e as regras do negócio característicos do projeto tratado neste documento. Para representar os requisitos funcionais e os requisitos não funcionais se usou as histórias de usuário. Foi com base na utilização da abordagem ágil no projeto (mais especificada na seção 3), que se definiu três prioridades principais: alta, média e baixa. 

A prioridade alta é aquela correspondente aos requisitos obrigatórios para o funcionamento do sistema, isto é, aqueles que caso faltem, o sistema em si não existe; já na prioridade média, por exemplo, são característicos aqueles desejáveis, ou seja, são importantes para o sistema, mas não interferem diretamente numa mudança brusca de seu comportamento. Desse modo, os requisitos de prioridade baixa, são tidos como opcionais, aqueles que podem entrar no sistema eventualmente, mas que, numa fila de produção, não serão feitos antes dos demais. Vale ressaltar, por conseguinte, que tais decisões dependem da negociação entre os envolvidos no projeto e do método de produção utilizado, como descrevem \citeonline{vazquezsimoes:2016}. Os autores ainda completam, dizendo:


\begin{citacao}
A priorização tem como função assegurar que os recursos do projeto sejam focados nos itens mais relevantes. Daí a importância de, na especificação, diferenciar cada requisito em termos de importância, dentre dezenas ou centenas de outros requisitos.

A responsabilidade por definir a prioridade do requisito deveria ser da parte interessada, facilitada pelo gerente de projetos \cite{vazquezsimoes:2016}.
\end{citacao}

\subsection{Histórias de Usuário}
Segundo \citeonline{cruz:2018} as histórias de usuário se caracterizam como uma descrição resumida, clara e objetiva de uma funcionalidade fornecida pelo produto a ser entregue, visando o ponto de vista final do usuário. Ainda segundo o autor para uma história ser tida como completa, ela deve possuir uma descrição objetiva e critérios de aceitação, esses critérios representam o que ela precisa fazer para ser considerada válida.

No projeto, a equipe aproveitou as histórias de usuário para representar os requisitos funcionais e não funcionais, dessa forma os requisitos funcionais podem ser identificados a partir do nome das histórias, e o não funcionais por meio dos critérios de aceitação definidos para tais. Além disso, considerando a entrega da \acs{POC}, as histórias foram separadas conforme a elaboração dos dois épicos preparados para esta entrega, sendo elas a gestão de perguntas e a gestão de respostas.

Todas as histórias apresentam sete componentes: o seu nome, a sua descrição, seus critérios de aceitação, o épico a qual pertence, a pontuação que ela recebeu no \textsl{planning poker}, a estimativa de tamanho conforme a sua pontuação e a sua prioridade conforme o seu tamanho e pontuação.

%%%%%%%  História - Manter uma pergunta %%%%%%% 
\def\arraystretch{2}
\begin{quadro}[htb]
\centering
\ABNTEXfontereduzida
\caption[História: Manter uma pergunta]{História: Manter uma pergunta}
\resizebox{\linewidth}{!}{
\begin{tabular}{|p{6.5cm}|c|c|c|c|}
\hline
\thead{Descrição} & \thead{Épico} & \thead{Pontuação} & \thead{Tamanho} & \thead{Prioridade}\\
\hline

Como aluno, eu gostaria de manter uma pergunta na comunidade para retirar uma dúvida & Gestão de Perguntas & 13 & Grande & ALTA \\ \hline

\end{tabular}}\legend{Fonte: Os autores}
\end{quadro}
\FloatBarrier 

Para esta história de usuário foram definidos os seguintes itens como critérios de aceitação:

\begin{itemize}
\item Mostrar ``como fazer uma boa pergunta'';
\item O usuário deve conseguir somente criar e remover uma pergunta da visualização;
\item O usuário deve conseguir fechar o espaço de resposta para a pergunta;
\end{itemize}

%%%%%%% História - Buscar Perguntas %%%%%%% 
\def\arraystretch{2}
\begin{quadro}[htb]
\centering
\ABNTEXfontereduzida
\caption[História: Buscar perguntas]{História: Buscar perguntas}
\resizebox{\linewidth}{!}{
\begin{tabular}{|p{6.5cm}|c|c|c|c|}
\hline
\thead{Descrição} & \thead{Épico} & \thead{Pontuação} & \thead{Tamanho} & \thead{Prioridade}\\
\hline

Como aluno, eu gostaria de buscar perguntas feitas para que possa consultar uma pergunta específica & Gestão de Perguntas & 2 & Pequena & Média \\ \hline

\end{tabular}}\legend{Fonte: Os autores}
\end{quadro}
\FloatBarrier 

Para esta história de usuário foram definidos os seguintes itens como critérios de aceitação:

\begin{itemize}
\item O usuário precisa informar total ou parcialmente o título da pergunta desejada;
\item As perguntas serão exibidas conforme as informações passadas, podendo ser semelhantes parcial ou totalmente;
\end{itemize}

%%%%%%% História - Curtir uma pergunta %%%%%%% 
\def\arraystretch{2}
\begin{quadro}[htb]
\centering
\ABNTEXfontereduzida
\caption[História: Curtir uma pergunta]{História: Curtir uma pergunta}
\resizebox{\linewidth}{!}{
\begin{tabular}{|p{6.5cm}|c|c|c|c|}
\hline
\thead{Descrição} & \thead{Épico} & \thead{Pontuação} & \thead{Tamanho} & \thead{Prioridade}\\
\hline

Como aluno, eu gostaria de votar em uma pergunta para indicar se ela me foi útil ou não. & Gestão de Perguntas & 2 & Pequena & Alta \\ \hline

\end{tabular}}\legend{Fonte: Os autores}
\end{quadro}
\FloatBarrier 

Para esta história de usuário foram definidos os seguintes itens como critérios de aceitação:

\begin{itemize}
\item Um usuário só poderá votar uma única vez;
\item Cada voto equivale a um ponto;
\item Soma dos pontos por pergunta deve ser exibida;
\end{itemize}

%%%%%%% História - Manter uma resposta %%%%%%% 
\def\arraystretch{2}
\begin{quadro}[htb]
\centering
\ABNTEXfontereduzida
\caption[História: Manter uma resposta]{História: Manter uma resposta}
\resizebox{\linewidth}{!}{
\begin{tabular}{|p{6.5cm}|c|c|c|c|}
\hline
\thead{Descrição} & \thead{Épico} & \thead{Pontuação} & \thead{Tamanho} & \thead{Prioridade}\\
\hline

Como aluno, eu gostaria de manter uma resposta para retirar uma dúvida de um colega. & Gestão de Respostas & 5 & Média & Alta \\ \hline

\end{tabular}}\legend{Fonte: Os autores}
\end{quadro}
\FloatBarrier 

Para esta história de usuário foram definidos os seguintes itens como critérios de aceitação:

\begin{itemize}
\item As respostas mais curtidas devem ser exibidas antes das demais;
\item O usuário deve conseguir somente criar e deletar uma resposta;
\item Todas as respostas devem ser exibidas sem exceção;
\end{itemize}

%%%%%%% História - Curtir uma resposta %%%%%%% 
\def\arraystretch{2}
\begin{quadro}[htb]
\centering
\ABNTEXfontereduzida
\caption[História: Curtir uma resposta]{História: Curtir uma resposta}
\resizebox{\linewidth}{!}{
\begin{tabular}{|p{6.5cm}|c|c|c|c|}
\hline
\thead{Descrição} & \thead{Épico} & \thead{Pontuação} & \thead{Tamanho} & \thead{Prioridade}\\
\hline

Como aluno, eu gostaria de curtir uma resposta para indicar se ela me foi útil ou não. & Gestão de Respostas & 1 & Pequena & Alta \\ \hline

\end{tabular}}\legend{Fonte: Os autores}
\end{quadro}
\FloatBarrier 

Para esta história de usuário foram definidos os seguintes itens como critérios de aceitação:

\begin{itemize}
\item Um usuário só poderá curtir uma única vez;
\item Cada curtida equivale a um ponto;
\item Soma das curtidas por pergunta deve ser exibida;
\end{itemize}


\subsection{Regras de Negócio}
Regras de negócio são requisitos de domínio de aplicação tratado no desenvolvimento de um sistema, isso significa, as declarações sobre como determinada empresa faz negócio. É a partir dessas regras que se define como o negócio funciona e quais são suas especificações, além da sua importância para o desenvolvimento de um sistema, pois, auxiliam no controle dos processos, ajudam na tomada de decisões além de afetarem diretamente os requisitos funcionais, como descrevem \citeonline{dallavalle2000regras}. Dessa forma, o \autoref{regras negocio} lista as regras de negócio levantadas para o projeto \gls{ifriends}.

\def\arraystretch{2}
\begin{quadro}[thb]
\centering
\ABNTEXfontereduzida
\caption{Regras de Negócio}
\label{regras negocio}
\resizebox{\linewidth}{!}{
\begin{tabular}{|c|p{13cm}|}
\hline
\textbf{ID} & \textbf{Descrição} \\
\hline
\acs{RN}01 & Não serão permitidos usuários com os mesmos dados, ou seja, o sistema só permitirá a criação de uma conta por usuário \\
\hline
\acs{RN}02 & A fim de garantir a segurança de nossos usuários, o sistema deverá fazer uso de algumas diretrizes da Lei Marco Civil da Internet, lei n\textdegree 12.965/2014, que tem como objetivo estabelecer princípios, garantias, direitos e deveres para o uso da internet no Brasil. \\
\hline
\acs{RN}03 & Dentro do direito a exclusão, ao excluir seu perfil, o usuário deve ter ciência de que suas perguntas e respostas serão mantidas na comunidade, porém como parte de um usuário anônimo (exemplo: user12345678). \\
\hline
\acs{RN}04 & O usuário deve resumir seu problema dentro do título da pergunta. \\
\hline
\acs{RN}05 & O usuário deve descrever seu problema, informar de forma clara o que ele precisa. \\
\hline
\acs{RN}06 &  O usuário deve descrever qual cenário o colocou nessa situação, o que já tentou e qual seu objetivo final. \\
\hline
\acs{RN}07 &  Se necessário, se deve fazer uso de imagens a fim de explicando melhor o problema. \\
\hline
\acs{RN}08 & O usuário deve lembrar que podem haver respostas diferentes, portanto deve manter a mente aberta a novas sugestões. \\
\hline

\end{tabular}}\legend {Fonte: os autores}
\end{quadro}
\FloatBarrier 


\section{Modelagem}
Segundo \citeonline{SOMMERVILLE:2019}, a modelagem de sistemas é definida como ``um processo de desenvolvimento de modelos abstratos de um sistema, cada um apresentando uma visão diferente do mesmo''. Para isso, descreve também que são geralmente usadas notações gráficas baseadas nos tipos de diagrama em \acs{uml} durante o processo de engenharia de requisitos ``para ajudar a derivar os requisitos detalhados de um sistema; durante o processo [...]; e depois da implementação, para documentar a estrutura e operação do sistema'' \cite{SOMMERVILLE:2019}. 

\subsection{Diagrama de Casos de Uso}
De acordo com \citeonline{umlGuedes}, o diagrama de casos de uso \acs{uml} é descrito como:

\begin{citacao}
O diagrama de casos de uso [...] tem por objetivo apresentar uma visão externa geral das funcionalidades que o sistema deverá oferecer aos usuários, sem se preocupar com a questão de como tais funcionalidades serão implementadas. [...] É de grande auxílio para a identificação e compreensão dos requisitos do sistema, ajudando a especificar, visualizar e documentar as características, funções e serviços do sistema desejados pelo usuário \cite{umlGuedes}.
\end{citacao}

Logo, a \autoref{diagrama_CasosUso} representa os casos de uso do projeto de sistema \gls{ifriends}.

\begin{figure}[htb]
\centering
\caption{Diagrama de Casos de Uso}
\label{diagrama_CasosUso}
\includegraphics[width=1.0\textwidth]{anexos/Imagens_Diagramas/CasosDeUso_IFriends.png}
\fonte{Os autores.}
\end{figure}
\FloatBarrier


%\subsection{Diagrama de Classes}
% É necessário?

\subsection{Diagrama de Entidade e Relacionamento}

De acordo com \citeonline{leal2015linguagem}, a abordagem entidade-relacionamento é a técnica de modelagem de dados mais difundida e utilizada e representa a modelo conceitual do banco de dados. Nela, a estrutura do banco de dados é descrita como coleção de entidades, relacionamentos e representada graficamente por meio do Diagrama Entidade Relacionamento.

Através dele, é possível descrever um subconjunto do mundo real que será retratado no banco de dados com um alto nível de abstração. Além disso, o modelo Entidade Relacionamento é um modelo formal e caracteriza-se por ter uma grande capacidade semântica, o que garante que todos possam ter o mesmo entendimento \cite{leal2015linguagem}.

A \autoref{diagrama_EntidadeRelacionamento} representa o \ac{der} do projeto de sistema \gls{ifriends}.

\begin{figure}[htb]
\centering
\caption{Diagrama de Entidade e Relacionamento}
\label{diagrama_EntidadeRelacionamento}
\includegraphics[width=1.0\textwidth]{anexos/Imagens_Diagramas/DER_IFriends.png}
\fonte{Os autores.}
\end{figure}
\FloatBarrier


\subsection{Diagrama de Tabelas Relacionais}

O Diagrama de Tabelas Relacionais \acs{dtr} representa o modelo lógico do Banco de Dados. Segundo \citeonline{utilidadepublica:201?}, através do modelo lógico é representado de maneira mais clara as entidades e os relacionamentos, pois considera algumas limitações e implementa recursos como adequação de padrão e nomenclatura, define as chaves primárias e estrangeiras, normalização, integridade referencial, entre outras.

Deste modo, a \autoref{diagrama_TabelasRelacionais} representa o \ac{dtr} do projeto de sistema \gls{ifriends}.

\begin{figure}[htb]
\centering
\caption{Diagrama de Tabelas Relacionais}
\label{diagrama_TabelasRelacionais}
\includegraphics[width=0.9\textwidth]{anexos/Imagens_Diagramas/DTR_IFriends.png}
\fonte{Os autores.}
\end{figure}
\FloatBarrier

\subsection{Dicionário de Dados}

O Dicionário de dados é responsável por armazenar as informações de configuração do banco de dados e as estruturas que compõem suas respectivas tabelas. As estruturas definem os campos e suas propriedades \cite{alvesbanco}.

Conforme \citeonline{date2004introdução}, o Dicionário de dados é o lugar em que – dentre outras coisas – todos os diversos esquemas (externo, conceitual, interno) e todos os mapeamentos correspondentes são mantidos.

O Dicionário de dados contém os metadados, dados que explicam dados, com relação aos diversos objetos que são de interesse do próprio sistema. Exemplos desses objetos incluem índices, usuários, restrições de integridade, restrições de segurança, e assim por diante, informações que essenciais para que o sistema faça seu trabalho apropriadamente \cite{date2004introdução}.

De tal modo, abaixo encontram-se os quadros que representam o \ac{dd} das tabelas de banco de dados do projeto \gls{ifriends}.

%%%%%%%%% Tabela usuario
\def\arraystretch{1.5}

\begin{quadro}[htb]
\centering
\ABNTEXfontereduzida
\caption[Usuário]{Usuário.}
\begin{tabular}{|>{\Centering}m{3cm}|>{\Centering}m{1.75cm}|>{\Centering}m{1.6cm}|>{\Centering}m{1.15cm}|>{\Centering}m{1.25cm}|m{4.5cm}|}
\hline
\thead{Atributo} & \thead{Tipo} & \thead{Tamanho} & \thead{Nulo} & \thead{Chave} & \thead{Descrição}\\
\hline

id\_usuario & INT & 11 & N & PK & Chave primária do usuário \\ \hline
email & VARCHAR & 50 & N &  & E-mail institucional do usuário \\ \hline
senha & VARCHAR & 50 & N &  & Senha de acesso ao sistema \\ \hline
link\_img & TEXT &  & S &  & link da imagem de perfil \\ \hline
curso & VARCHAR & 50 & S &  & Curso atual \\ \hline
nome\_usuario & VARCHAR & 120 & N &  & Nome do usuário \\ \hline
ano & INT & 1 & S &  & Ano que o usuário cursa, ex.: 1\textdegree ano \\ \hline
reputacao\_total & INT & 11 & N  &  & Pontuação da reputação do usuário \\ \hline

\end{tabular}\legend{Fonte: Os autores}
\end{quadro}
\FloatBarrier 

%%%%%%%% Tabela usuario - pergunta
\def\arraystretch{1.5}

\begin{quadro}[htb]
\centering
\ABNTEXfontereduzida
\caption[Usuário\_Pergunta]{Usuário\_Pergunta.}
\label{quadro-dicionario-dados}
\begin{tabular}{|>{\Centering}m{3cm}|>{\Centering}m{1.75cm}|>{\Centering}m{1.6cm}|>{\Centering}m{1.15cm}|>{\Centering}m{1.25cm}|m{4.5cm}|}
\hline
\thead{Atributo} & \thead{Tipo} & \thead{Tamanho} & \thead{Nulo} & \thead{Chave} & \thead{Descrição}\\
\hline

id\_usuario & INT & 11 & N & FK & Chave estrangeira no usuário \\ \hline
id\_pergunta & INT & 11 & N & FK & Chave estrangeira na pergunta \\ \hline

\end{tabular}\legend{Fonte: Os autores}
\end{quadro}
\FloatBarrier 

%\clearpage

%usuario - resposta
\def\arraystretch{1.5}

\begin{quadro}[htb]
\centering
\ABNTEXfontereduzida
\caption[Usuário\_Resposta]{Usuário\_Resposta.}
\label{quadro-dicionario-dados}
\begin{tabular}{|>{\Centering}m{3cm}|>{\Centering}m{1.75cm}|>{\Centering}m{1.6cm}|>{\Centering}m{1.15cm}|>{\Centering}m{1.25cm}|m{4.5cm}|}
\hline
\thead{Atributo} & \thead{Tipo} & \thead{Tamanho} & \thead{Nulo} & \thead{Chave} & \thead{Descrição}\\
\hline

id\_usuario & INT & 11 & N & FK & Chave estrangeira no usuário \\ \hline
id\_resposta & INT & 11 & N & FK & Chave estrangeira na resposta \\ \hline

\end{tabular}\legend{Fonte: Os autores}
\end{quadro}
\FloatBarrier 

%usuario - Titulo
\def\arraystretch{1.5}

\begin{quadro}[htb]
\centering
\ABNTEXfontereduzida
\caption[Usuário\_Título]{Usuário\_Título.}
\label{quadro-dicionario-dados}
\begin{tabular}{|>{\Centering}m{3cm}|>{\Centering}m{1.75cm}|>{\Centering}m{1.6cm}|>{\Centering}m{1.15cm}|>{\Centering}m{1.25cm}|m{4.5cm}|}
\hline
\thead{Atributo} & \thead{Tipo} & \thead{Tamanho} & \thead{Nulo} & \thead{Chave} & \thead{Descrição}\\
\hline

id\_usuario & INT & 11 & N & FK & Chave estrangeira no usuário \\ \hline
id\_titulo & INT & 11 & N & FK & Chave estrangeira no titulo \\ \hline

\end{tabular}\legend{Fonte: Os autores}
\end{quadro}
\FloatBarrier 


%Titulo
\def\arraystretch{1.5}

\begin{quadro}[htb]
\centering
\ABNTEXfontereduzida
\caption[Título]{Título.}
\label{quadro-dicionario-dados}
\begin{tabular}{|>{\Centering}m{3cm}|>{\Centering}m{1.75cm}|>{\Centering}m{1.6cm}|>{\Centering}m{1.15cm}|>{\Centering}m{1.25cm}|m{4.5cm}|}
\hline
\thead{Atributo} & \thead{Tipo} & \thead{Tamanho} & \thead{Nulo} & \thead{Chave} & \thead{Descrição}\\
\hline

id\_titulo & INT & 11 & N & PK & Chave primária do título \\ \hline
nome\_titulo & VARCHAR & 50 & N &  & Nome do título \\ \hline
reputacao & INT & 11 & N & & Acumulo da pontuação \\\hline

\end{tabular}\legend{Fonte: Os autores}
\end{quadro}
\FloatBarrier 
%\clearpage

%Resposta
\def\arraystretch{1.5}

\begin{quadro}[htb]
\centering
\ABNTEXfontereduzida
\caption[Resposta]{Resposta.}
\label{quadro-dicionario-dados}
\begin{tabular}{|>{\Centering}m{3cm}|>{\Centering}m{1.75cm}|>{\Centering}m{1.6cm}|>{\Centering}m{1.15cm}|>{\Centering}m{1.25cm}|m{4.5cm}|}
\hline
\thead{Atributo} & \thead{Tipo} & \thead{Tamanho} & \thead{Nulo} & \thead{Chave} & \thead{Descrição}\\ \hline

id\_resposta & INT & 11 & N & PK & Chave primária da resposta \\ \hline
id\_usuario & INT & 11 & N & FK  & Chave estrangeira de usuário \\ \hline
id\_pergunta & INT & 11 & N & FK & Chave estrangeira de pergunta \\ \hline
texto\_resp & TEXT &  & N &  & Conteúdo da resposta \\ \hline
ativo & BOOLEAN & & N & & Resposta ativa ou não \\ \hline
img\_resp & TEXT & & S & & Imagem da resposta \\ \hline
aceita & BOOLEAN & & N & & Se a resposta foi aceita como solução válida para o autor da pergunta \\ \hline
dt\_resposta & DATE & 50 & N & & Data em que a resposta foi publicada \\ \hline

\end{tabular}\legend{Fonte: Os autores}
\end{quadro}
\FloatBarrier 

%Tag
\def\arraystretch{1.5}

\begin{quadro}[htb]
\centering
\ABNTEXfontereduzida
\caption[Tag]{Tag.}
\label{quadro-dicionario-dados}
\begin{tabular}{|>{\Centering}m{3cm}|>{\Centering}m{1.75cm}|>{\Centering}m{1.6cm}|>{\Centering}m{1.15cm}|>{\Centering}m{1.25cm}|m{4.5cm}|}
\hline
\thead{Atributo} & \thead{Tipo} & \thead{Tamanho} & \thead{Nulo} & \thead{Chave} & \thead{Descrição}\\
\hline

id\_tag & INT & 3 & N & PK & Chave primária da tag \\ \hline
nome\_tag & VARCHAR & 50 & N &  & Nome da tag \\ \hline

\end{tabular}\legend{Fonte: Os autores}
\end{quadro}
\FloatBarrier 
%\clearpage

%Pergunta
\def\arraystretch{1.5}

\begin{quadro}[htb]
\centering
\ABNTEXfontereduzida
\caption[Pergunta]{Pergunta.}
\label{quadro-dicionario-dados}
\begin{tabular}{|>{\Centering}m{3cm}|>{\Centering}m{1.75cm}|>{\Centering}m{1.6cm}|>{\Centering}m{1.15cm}|>{\Centering}m{1.25cm}|m{4.5cm}|}
\hline
\thead{Atributo} & \thead{Tipo} & \thead{Tamanho} & \thead{Nulo} & \thead{Chave} & \thead{Descrição}\\ \hline

id\_pergunta & INT & 11 & N & PK & Chave primária da pergunta \\ \hline
id\_usuario & INT & 11 & N & FK  & Chave estrangeira de usuário \\ \hline
dt\_perg & DATE &  & N & & Data da pergunta \\ \hline
titulo\_perg & VARCHAR & 50 & N & & Título da pergunta \\ \hline
texto\_perg & TEXT & & N & & Descrição da pergunta \\ \hline
ativo & BOOLEAN & & N & & Pergunta ativa ou não \\ \hline
link\_img\_perg & TEXT & & S & & Link da imagem da pergunta \\ \hline
respondida & BOOLEAN & & N & & Se a pergunta já teve uma resposta útil a quem perguntou \\ \hline
visualizações & INT & 5 & N & & Quantidade de visualizações da pergunta \\ \hline

\end{tabular}\legend{Fonte: Os autores}
\end{quadro}
\FloatBarrier 


%Tag - Pergunta
\def\arraystretch{1.5}

\begin{quadro}[htb]
\centering
\ABNTEXfontereduzida
\caption[Tag\_Pergunta]{Tag\_Pergunta.}
\label{quadro-dicionario-dados}
\begin{tabular}{|>{\Centering}m{3cm}|>{\Centering}m{1.75cm}|>{\Centering}m{1.6cm}|>{\Centering}m{1.15cm}|>{\Centering}m{1.25cm}|m{4.5cm}|}
\hline
\thead{Atributo} & \thead{Tipo} & \thead{Tamanho} & \thead{Nulo} & \thead{Chave} & \thead{Descrição}\\ \hline

id\_assunto & INT & 11 & N & FK & Chave estrangeira no assunto \\ \hline
id\_pergunta & INT & 11 & N & FK & Chave estrangeira na pergunta \\ \hline

\end{tabular}\legend{Fonte: Os autores}
\end{quadro}
\FloatBarrier 
%\clearpage

%Tag - Evento
\def\arraystretch{1.5}

\begin{quadro}[htb]
\centering
\ABNTEXfontereduzida
\caption[Tag\_Evento]{Tag\_Evento.}
\label{quadro-dicionario-dados}
\begin{tabular}{|>{\Centering}m{3cm}|>{\Centering}m{1.75cm}|>{\Centering}m{1.6cm}|>{\Centering}m{1.15cm}|>{\Centering}m{1.25cm}|m{4.5cm}|}
\hline
\thead{Atributo} & \thead{Tipo} & \thead{Tamanho} & \thead{Nulo} & \thead{Chave} & \thead{Descrição}\\ \hline

id\_assunto & INT & 11 & N & FK & Chave estrangeira no assunto \\ \hline
id\_evento & INT & 11 & N & FK & Chave estrangeira no evento \\ \hline

\end{tabular}\legend{Fonte: Os autores}
\end{quadro}
\FloatBarrier 

%evento
\def\arraystretch{1.5}

\begin{quadro}[htb]
\centering
\ABNTEXfontereduzida
\caption[Evento]{Evento.}
\label{quadro-dicionario-dados}
\begin{tabular}{|>{\Centering}m{3cm}|>{\Centering}m{1.75cm}|>{\Centering}m{1.6cm}|>{\Centering}m{1.15cm}|>{\Centering}m{1.25cm}|m{4.5cm}|}
\hline
\thead{Atributo} & \thead{Tipo} & \thead{Tamanho} & \thead{Nulo} & \thead{Chave} & \thead{Descrição}\\
\hline

id\_evento & INT & 11 & N & PK & Chave primária do evento \\ \hline
id\_usuario & INT & 11 & S & FK  & Chave estrangeira de usuário \\ \hline
presencial & char & 1 & N & & Local do evento, sendo presencial, online ou ambos \\ \hline
nome\_evento & VARCHAR & 50 & N & & Nome do evento \\ \hline
texto\_evento & TEXT &  & N & & Descrição sobre o evento \\ \hline
dt\_evento & DATE & & N & & Data que o evento ocorrerá \\ \hline
img\_evento & TEXT &  & S & & Imagem do evento \\ \hline
local & TEXT & & N & & Local onde será realizado, tanto presencial como online \\ \hline
\end{tabular}\legend{Fonte: Os autores}
\end{quadro}
\FloatBarrier 

%Categoria
\def\arraystretch{1.5}

\begin{quadro}[htb]
\centering
\ABNTEXfontereduzida
\caption[Categoria]{Categoria.}
\label{quadro-dicionario-dados}
\begin{tabular}{|>{\Centering}m{3cm}|>{\Centering}m{1.75cm}|>{\Centering}m{1.6cm}|>{\Centering}m{1.15cm}|>{\Centering}m{1.25cm}|m{4.5cm}|}
\hline
\thead{Atributo} & \thead{Tipo} & \thead{Tamanho} & \thead{Nulo} & \thead{Chave} & \thead{Descrição}\\
\hline

id\_categoria & INT & 11 & N & PK & Chave primária da categoria \\ \hline
nome\_categoria & INT & 50 & N & FK & Nome da categoria \\ \hline

\end{tabular}\legend{Fonte: Os autores}
\end{quadro}
\FloatBarrier 

%\clearpage

\section{Prototipagem}
Segundo \citeonline{ferreira:2020}, ``prototipar é trazer, para o mundo real, o mundo palpável, as ideias de negócio construídas no mundo abstrato, na teoria''. Isto é, o autor comenta que um protótipo é um recurso utilizado para demonstrar e escolher a solução para representar uma ideia, podendo ser efetuado com entregas digitais, como telas de sistema. Dado isto, a próxima seção apresentará as telas prototipadas do projeto de sistema \gls{ifriends}.

Ainda, para auxiliar na prototipação das telas, foi elaborado um mapa mental de modo a representar melhor o fluxo do nosso projeto, que pode ser conferido na \autoref{Mapa mental}.

\begin{figure}[htb]
\centering
\caption{\label{Mapa mental} Mapa mental}
\includegraphics[width=1\textwidth]{anexos/Imagens_Prototipo/Mapa_Mental.png}
\fonte{os autores}
\end{figure}
\FloatBarrier

\subsection{Protótipos de alta fidelidade}
Nesta seção estão contidas as figuras que representam as principais telas do sistema em relação a \gls{POC} do projeto, cada tela apresenta uma breve contextualização sobre o seu conteúdo. De todo modo, a apresentação pode ser visualizada também pelo \href{https://www.figma.com/proto/GhIlybDubGmr3NkRU0a9GP/Protótipo---IFriends?node-id=73\%3A321}{Figma}.

A \autoref{Home Page} apresenta a página inicial do projeto, onde o usuário entra em contato pela primeira vez com o sistema. Nela o usuário pode navegar através de dois menus disponíveis na página: o lateral e o superior, usar a barra de pesquisa, \textit{logar} no seu perfil, acessar as suas configurações, entre outras ações disponibilizadas. Na página encontram as questões mais relevantes da comunidade, assim como os espaços destinados para a realização de uma pergunta ou de uma monitoria.

\begin{figure}[htb]
\centering
\caption{\label{Home Page} Página inicial}
\includesvg[inkscapelatex=false,width=0.7\textwidth]{anexos/Imagens_Prototipo/Home_Page.svg}
\fonte{os autores}
\end{figure}
\FloatBarrier

Quando o usuário clica em uma pergunta ou em ``Responder'' ele é direcionado à página dessa pergunta como mostra a \autoref{Pergunta e respostas}, nela ele pode encontrar as respostas já fornecidas por outros membros da comunidade, assim como também pode deixar a sua contribuição.

\begin{figure}[htb]
\centering
\caption{\label{Pergunta e respostas} Pergunta e respostas}
\includesvg[inkscapelatex=false,width=0.9\textwidth]{anexos/Imagens_Prototipo/Pergunta_Respostas.svg}
\fonte{os autores}
\end{figure}
\FloatBarrier

Já a \autoref{Cadastro de perguntas} corresponde a página de cadastro de perguntas, nessa tela são apresentados todos os elementos julgados necessários para a sua realização, nesta página ainda de encontra o manual de uma boa pergunta, tal foi elaborado com o intuito de ajudar e auxiliar o usuário na preparação de sua problemática. 

\begin{figure}[htb]
\centering
\caption{\label{Cadastro de perguntas} Cadastro de perguntas}
\includesvg[inkscapelatex=false,width=0.9\textwidth]{anexos/Imagens_Prototipo/Cadastro_Perguntas.svg}
\fonte{os autores}
\end{figure}
\FloatBarrier

As telas apresentadas até o momento são aquelas que se encontram relacionadas com a \gls{POC}, portanto vale salientar que esta seção será atualizada futuramente conforme o andamento e elaboração do projeto.


%---------------------------------------------------------------------------------------





