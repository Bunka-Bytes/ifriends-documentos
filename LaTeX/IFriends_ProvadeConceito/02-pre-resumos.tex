% ---
% RESUMOS
% ---

% resumo em português
\setlength{\absparsep}{18pt} % ajusta o espaçamento dos parágrafos do resumo
\begin{resumo}

O presente documento é o resultado obtido até a elaboração da \acs{POC} do projeto que visa a criação de uma comunidade virtual do \acs{ifsp} através do aprendizado adquirido na disciplina técnica de PDS no quarto ano do Curso Técnico de Informática, realizado no Instituto Federal de Educação, Ciência e Tecnologia de São Paulo, Campus de São Paulo. O objetivo central deste trabalho, desse modo, é apresentar a projeção e a implementação de uma prova de conceito de uma comunidade virtual qual visa a criação de um espaço de acolhimento de alunos para alunos. Propõe-se, assim, utilizar de um método ágil e de ferramentas de desenvolvimento para passar pelos processos de engenharia do sistema, além de estimular o trabalho em equipe. Sob essa perspectiva, o projeto pôde ser apresentado abordando seu tema principal e focando nas suas funcionalidades mais essenciais, como a gerenciamento das perguntas e a gerenciamento das respostas.


\textbf{Palavras-chaves}: Projeto. IFriends. Comunidade virtual.
\end{resumo}

% % resumo em inglês
% \begin{resumo}[Abstract]
% \begin{otherlanguage*}{english}
%   This is the english abstract.
% \todo[inline]{fazer tradução do resumo, não utilizar tradução automática}
%   \vspace{\onelineskip}

%   \noindent 
%   \textbf{Keywords}: Project. IFriends. Online Community.
%  \end{otherlanguage*}
% \end{resumo}