
% ----------------------------------------------------------
% Introdução (exemplo de capítulo sem numeração, mas presente no Sumário)
% ----------------------------------------------------------
\chapter{Introdução}
O presente documento é resultado da proposta de um projeto cujo objetivo é sustentado no planejamento e na execução de um sistema para a Web através do aprendizado obtido nas matérias técnicas do Curso Técnico de Informática, realizado no IFSP, e como forma de trabalho de conclusão de curso.

Tendo isto em vista e diante dos desafios presentes na lista de requisitos e orientações propostas para a iniciação deste projeto, os integrantes da equipe Bunka Bytes reuniram-se em busca de encontrar a solução que melhor se encaixasse em seus objetivos e nos da disciplina de \acs{pds}.

É pensando em soluções viáveis que a equipe voltou seu olhar para sistemas que contribuem para a criação de comunidades colaborativas na área de desenvolvimento de sistemas, que, segundo \citeonline{rosa2008identidade}, foi fortemente difundida pelas comunidades de Código Aberto (do inglês, \textsl{Open Source}), sendo primeiramente criada pela cultura \textsl{hacker}, na qual afirma que a paixão e o interesse dos \textsl{hackers} nas soluções foi uma das principais propulsoras do espírito colaborativo.

Além deles, foi trazido ao debate as possibilidades de aliar as principais dificuldades que os integrantes observaram durante sua vida acadêmica no \acs{ifsp}, a um sistema que pudesse suprir determinadas necessidades dos alunos, como os questionamentos que começam a surgir com mais frequência conforme o início dos estudos é dado, sendo eles em âmbitos diversos como: sobre a instituição de ensino, matérias e assuntos tratados no ensino médio, dúvidas sobre os conteúdos técnicos ou até mesmo a busca por um apoio educacional - como ocorrem nas monitoriais. 

O \gls{ifriends} surge nesse cenário, no qual a criação de uma comunidade de estudantes que colaborassem entre si, pudesse instigar o interesse dos alunos em ajudarem uns aos outros de maneira acessível e prática, onde uma dúvida estivesse a um palmo de distância.


\section{Objetivo}
%Nesta seção serão descritos os respectivos objetivos (principal e específico) pretendidos com a iniciação deste projeto. 
%\subsection{Objetivo Principal}

O objetivo deste projeto é tentar instigar o interesse dos estudantes que compreendem o \acs{ifsp} para poderem criar espaços colaborativos entre si, por um sistema onde os usuários interajam entre perguntas e respostas, fornecendo caminhos para o esclarecimento de suas dúvidas sobre a instituição de ensino, as áreas e disciplinas que a ela pertence.  

Dessa forma, o objetivo será aplicado através da construção de uma plataforma de perguntas, respostas e mentorias para a Web, em que qualquer usuário poderá submeter uma pergunta para ser respondida pelos outros membros da comunidade; além de possibilitar que estudantes possam escolher se tornar mentores sobre determinados assuntos, disponibilizando recursos para a criação de anúncios de eventos de monitorias (cuja localidade a eles deve competir) dentro de seus perfis de usuário. 

Tendo isso em vista e pensando numa melhor interatividade entre os usuários, o sistema deve passar por um processo de \gls{gamificação} em algumas de suas funcionalidades, como as votações para respostas e perguntas mais relevantes e os atributos dos usuários mais ativos  - assim como outros exemplos que devem ser adicionados durante o planejamento do projeto.


%\subsection{Objetivos Específicos}
%Dado o objetivo principal do projeto, o objetivo específico descrito nesta seção dá-se em forma de proposta de implementação do sistema citado anteriormente, na qual descrevemos um resumo da aplicação e seus propósitos, tendo em vista sua importância social e suas especificações técnicas. 


\section{Justificativa}
A reflexão com relação às formas complementares de aprendizagem é importante para a ampliação dos conteúdos interessados tanto aos alunos, quanto aos seus professores, pois permite que enxerguem, juntos, o ensino como um meio que evite a passagem de aprendizados de forma restrita e hierarquizada.

Por isso, \citeonline{fernandes2011redes} traz em sua pesquisa que o desafio da construção de sociedades de aprendizagem parte do pressuposto de que os recursos tecnológicos disponibilizados atualmente permitem aos estudantes aprenderem dentro e fora da escola e das mais variadas formas. Assim, para ele, a melhor forma se dá “construindo comunidades sustentadas pelo uso de tecnologias Web”.

O autor dá continuidade na exposição desse fenômeno ao atribuir o sucesso da potencialização da aprendizagem complementar e das relações sociais à ``Web 2.0''. Isto, pois, de acordo \citeonline{fernandes2011redes}, a mesma permitiu novas formas e possibilidades de criação de conteúdos e possibilitou o enfoque a uma aprendizagem motivada pelos interesses do aluno, em que ele deve assumir um papel exploratório nessa experiência, da qual poderá colher ensinamentos significativos, explica \citeonline{fernandes2011redes}.

Visando atrair atenção para o tema, o projeto tem como principal missão, permitir que os estudantes possam usufruir de uma ferramenta gratuita que proporcione a suavização do seu processo de aprendizagem, quando seus próprios colegas contribuirão com suas experiências passadas, além de deixarem um histórico para possibilitar um caminho menos árduo aos estudantes que virão. Por isso, espera-se que, com este projeto, a instituição de ensino também seja um agente na construção de uma comunidade propícia para estudantes, onde poderão unir-se em razão de dúvidas comuns, e assim incentivarem a disseminação de uma cultura colaborativa dentro de seus espaços.


